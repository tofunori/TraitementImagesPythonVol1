% Options for packages loaded elsewhere
\PassOptionsToPackage{unicode}{hyperref}
\PassOptionsToPackage{hyphens}{url}
\documentclass[
]{article}
\usepackage{xcolor}
\usepackage{amsmath,amssymb}
\setcounter{secnumdepth}{-\maxdimen} % remove section numbering
\usepackage{iftex}
\ifPDFTeX
  \usepackage[T1]{fontenc}
  \usepackage[utf8]{inputenc}
  \usepackage{textcomp} % provide euro and other symbols
\else % if luatex or xetex
  \usepackage{unicode-math} % this also loads fontspec
  \defaultfontfeatures{Scale=MatchLowercase}
  \defaultfontfeatures[\rmfamily]{Ligatures=TeX,Scale=1}
\fi
\usepackage{lmodern}
\ifPDFTeX\else
  % xetex/luatex font selection
\fi
% Use upquote if available, for straight quotes in verbatim environments
\IfFileExists{upquote.sty}{\usepackage{upquote}}{}
\IfFileExists{microtype.sty}{% use microtype if available
  \usepackage[]{microtype}
  \UseMicrotypeSet[protrusion]{basicmath} % disable protrusion for tt fonts
}{}
\makeatletter
\@ifundefined{KOMAClassName}{% if non-KOMA class
  \IfFileExists{parskip.sty}{%
    \usepackage{parskip}
  }{% else
    \setlength{\parindent}{0pt}
    \setlength{\parskip}{6pt plus 2pt minus 1pt}}
}{% if KOMA class
  \KOMAoptions{parskip=half}}
\makeatother
\usepackage{color}
\usepackage{fancyvrb}
\newcommand{\VerbBar}{|}
\newcommand{\VERB}{\Verb[commandchars=\\\{\}]}
\DefineVerbatimEnvironment{Highlighting}{Verbatim}{commandchars=\\\{\}}
% Add ',fontsize=\small' for more characters per line
\newenvironment{Shaded}{}{}
\newcommand{\AlertTok}[1]{\textcolor[rgb]{1.00,0.00,0.00}{\textbf{#1}}}
\newcommand{\AnnotationTok}[1]{\textcolor[rgb]{0.38,0.63,0.69}{\textbf{\textit{#1}}}}
\newcommand{\AttributeTok}[1]{\textcolor[rgb]{0.49,0.56,0.16}{#1}}
\newcommand{\BaseNTok}[1]{\textcolor[rgb]{0.25,0.63,0.44}{#1}}
\newcommand{\BuiltInTok}[1]{\textcolor[rgb]{0.00,0.50,0.00}{#1}}
\newcommand{\CharTok}[1]{\textcolor[rgb]{0.25,0.44,0.63}{#1}}
\newcommand{\CommentTok}[1]{\textcolor[rgb]{0.38,0.63,0.69}{\textit{#1}}}
\newcommand{\CommentVarTok}[1]{\textcolor[rgb]{0.38,0.63,0.69}{\textbf{\textit{#1}}}}
\newcommand{\ConstantTok}[1]{\textcolor[rgb]{0.53,0.00,0.00}{#1}}
\newcommand{\ControlFlowTok}[1]{\textcolor[rgb]{0.00,0.44,0.13}{\textbf{#1}}}
\newcommand{\DataTypeTok}[1]{\textcolor[rgb]{0.56,0.13,0.00}{#1}}
\newcommand{\DecValTok}[1]{\textcolor[rgb]{0.25,0.63,0.44}{#1}}
\newcommand{\DocumentationTok}[1]{\textcolor[rgb]{0.73,0.13,0.13}{\textit{#1}}}
\newcommand{\ErrorTok}[1]{\textcolor[rgb]{1.00,0.00,0.00}{\textbf{#1}}}
\newcommand{\ExtensionTok}[1]{#1}
\newcommand{\FloatTok}[1]{\textcolor[rgb]{0.25,0.63,0.44}{#1}}
\newcommand{\FunctionTok}[1]{\textcolor[rgb]{0.02,0.16,0.49}{#1}}
\newcommand{\ImportTok}[1]{\textcolor[rgb]{0.00,0.50,0.00}{\textbf{#1}}}
\newcommand{\InformationTok}[1]{\textcolor[rgb]{0.38,0.63,0.69}{\textbf{\textit{#1}}}}
\newcommand{\KeywordTok}[1]{\textcolor[rgb]{0.00,0.44,0.13}{\textbf{#1}}}
\newcommand{\NormalTok}[1]{#1}
\newcommand{\OperatorTok}[1]{\textcolor[rgb]{0.40,0.40,0.40}{#1}}
\newcommand{\OtherTok}[1]{\textcolor[rgb]{0.00,0.44,0.13}{#1}}
\newcommand{\PreprocessorTok}[1]{\textcolor[rgb]{0.74,0.48,0.00}{#1}}
\newcommand{\RegionMarkerTok}[1]{#1}
\newcommand{\SpecialCharTok}[1]{\textcolor[rgb]{0.25,0.44,0.63}{#1}}
\newcommand{\SpecialStringTok}[1]{\textcolor[rgb]{0.73,0.40,0.53}{#1}}
\newcommand{\StringTok}[1]{\textcolor[rgb]{0.25,0.44,0.63}{#1}}
\newcommand{\VariableTok}[1]{\textcolor[rgb]{0.10,0.09,0.49}{#1}}
\newcommand{\VerbatimStringTok}[1]{\textcolor[rgb]{0.25,0.44,0.63}{#1}}
\newcommand{\WarningTok}[1]{\textcolor[rgb]{0.38,0.63,0.69}{\textbf{\textit{#1}}}}
\usepackage{graphicx}
\makeatletter
\newsavebox\pandoc@box
\newcommand*\pandocbounded[1]{% scales image to fit in text height/width
  \sbox\pandoc@box{#1}%
  \Gscale@div\@tempa{\textheight}{\dimexpr\ht\pandoc@box+\dp\pandoc@box\relax}%
  \Gscale@div\@tempb{\linewidth}{\wd\pandoc@box}%
  \ifdim\@tempb\p@<\@tempa\p@\let\@tempa\@tempb\fi% select the smaller of both
  \ifdim\@tempa\p@<\p@\scalebox{\@tempa}{\usebox\pandoc@box}%
  \else\usebox{\pandoc@box}%
  \fi%
}
% Set default figure placement to htbp
\def\fps@figure{htbp}
\makeatother
\usepackage{svg}
\ifLuaTeX
\usepackage[bidi=basic]{babel}
\else
\usepackage[bidi=default]{babel}
\fi
\babelprovide[main,import]{french}
% get rid of language-specific shorthands (see #6817):
\let\LanguageShortHands\languageshorthands
\def\languageshorthands#1{}
\setlength{\emergencystretch}{3em} % prevent overfull lines
\providecommand{\tightlist}{%
  \setlength{\itemsep}{0pt}\setlength{\parskip}{0pt}}
\usepackage{bookmark}
\IfFileExists{xurl.sty}{\usepackage{xurl}}{} % add URL line breaks if available
\urlstyle{same}
\hypersetup{
  pdftitle={4~ Transformations spectrales -- Traitement d\textquotesingle images satellites avec Python},
  pdflang={fr},
  hidelinks,
  pdfcreator={LaTeX via pandoc}}

\title{4~ Transformations spectrales -- Traitement
d\textquotesingle images satellites avec Python}
\author{}
\date{}

\begin{document}
\maketitle

\phantomsection\label{quarto-document-content}
\phantomsection\label{title-block-header}
\section{\texorpdfstring{\protect\hypertarget{sec-chap03}{}{{4}~
{Transformations
spectrales}}}{4~ Transformations spectrales}}\label{transformations-spectrales}

\subsection{\texorpdfstring{{4.1} {🚀}
Préambule}{4.1 🚀 Préambule}}\label{pruxe9ambule}

Assurez-vous de lire ce préambule avant d'exécutez le reste du notebook.
\#\#\# {🎯} Objectifs Dans ce chapitre, nous abordons quelques
techniques de réhaussement et de visualisation d'images. Ce chapitre est
aussi disponible sous la forme d'un notebook Python:
\href{https://colab.research.google.com/github/sfoucher/TraitementImagesPythonVol1/blob/main/notebooks/03-TransformationSpectrales.ipynb}{\pandocbounded{\includesvg[keepaspectratio]{images/colab-badge.svg}}}

\subsubsection{\texorpdfstring{{4.1.1}
Librairies}{4.1.1 Librairies}}\label{librairies}

Les librairies qui vont être explorées dans ce chapitre sont les
suivantes:

\begin{itemize}
\item
  \href{https://scipy.org/}{SciPy -}
\item
  \href{https://numpy.org/}{NumPy -}
\item
  \href{https://pypi.org/project/opencv-python/}{opencv-python · PyPI}
\item
  \href{https://scikit-image.org/}{scikit-image}
\item
  \href{https://rasterio.readthedocs.io/en/stable/}{Rasterio}
\item
  \href{https://docs.xarray.dev/en/stable/}{Xarray}
\item
  \href{https://corteva.github.io/rioxarray/stable/index.html}{rioxarray}
\end{itemize}

Dans l'environnement Google Colab, seul \texttt{rioxarray} et GDAL
doivent être installés:

\phantomsection\label{3f7e33b2}
\phantomsection\label{cb1}
\begin{Shaded}
\begin{Highlighting}[]
\OperatorTok{\%\%}\NormalTok{capture}
\OperatorTok{!}\NormalTok{apt}\OperatorTok{{-}}\NormalTok{get update}
\OperatorTok{!}\NormalTok{apt}\OperatorTok{{-}}\NormalTok{get install gdal}\OperatorTok{{-}}\BuiltInTok{bin}\NormalTok{ libgdal}\OperatorTok{{-}}\NormalTok{dev}
\OperatorTok{!}\NormalTok{pip install }\OperatorTok{{-}}\NormalTok{q rioxarray}
\OperatorTok{!}\NormalTok{pip install }\OperatorTok{{-}}\NormalTok{qU }\StringTok{"geemap[workshop]"}
\end{Highlighting}
\end{Shaded}

Vérifier les importations:

\phantomsection\label{edab9829}
\phantomsection\label{cb2}
\begin{Shaded}
\begin{Highlighting}[]
\ImportTok{import}\NormalTok{ numpy }\ImportTok{as}\NormalTok{ np}
\ImportTok{import}\NormalTok{ rioxarray }\ImportTok{as}\NormalTok{ rxr}
\ImportTok{from}\NormalTok{ scipy }\ImportTok{import}\NormalTok{ signal}
\ImportTok{import}\NormalTok{ xarray }\ImportTok{as}\NormalTok{ xr}
\ImportTok{import}\NormalTok{ xrscipy}
\ImportTok{import}\NormalTok{ matplotlib.pyplot }\ImportTok{as}\NormalTok{ plt}
\end{Highlighting}
\end{Shaded}

\subsubsection{\texorpdfstring{{4.1.2} Images
utilisées}{4.1.2 Images utilisées}}\label{images-utilisuxe9es}

Nous allons utilisez les images suivantes dans ce chapitre:

\phantomsection\label{3035b64c}
\phantomsection\label{cb3}
\begin{Shaded}
\begin{Highlighting}[]
\OperatorTok{\%\%}\NormalTok{capture}
\OperatorTok{!}\NormalTok{wget https:}\OperatorTok{//}\NormalTok{github.com}\OperatorTok{/}\NormalTok{sfoucher}\OperatorTok{/}\NormalTok{TraitementImagesPythonVol1}\OperatorTok{/}\NormalTok{raw}\OperatorTok{/}\NormalTok{refs}\OperatorTok{/}\NormalTok{heads}\OperatorTok{/}\NormalTok{main}\OperatorTok{/}\NormalTok{data}\OperatorTok{/}\NormalTok{chapitre01}\OperatorTok{/}\NormalTok{subset\_RGBNIR\_of\_S2A\_MSIL2A\_20240625T153941\_N0510\_R011\_T18TYR\_20240625T221903.tif }\OperatorTok{{-}}\NormalTok{O RGBNIR\_of\_S2A.tif}
\OperatorTok{!}\NormalTok{wget https:}\OperatorTok{//}\NormalTok{github.com}\OperatorTok{/}\NormalTok{sfoucher}\OperatorTok{/}\NormalTok{opengeos}\OperatorTok{{-}}\NormalTok{data}\OperatorTok{/}\NormalTok{raw}\OperatorTok{/}\NormalTok{refs}\OperatorTok{/}\NormalTok{heads}\OperatorTok{/}\NormalTok{main}\OperatorTok{/}\NormalTok{raster}\OperatorTok{/}\NormalTok{landsat7.tif }\OperatorTok{{-}}\NormalTok{O landsat7.tif}
\OperatorTok{!}\NormalTok{wget https:}\OperatorTok{//}\NormalTok{github.com}\OperatorTok{/}\NormalTok{sfoucher}\OperatorTok{/}\NormalTok{opengeos}\OperatorTok{{-}}\NormalTok{data}\OperatorTok{/}\NormalTok{raw}\OperatorTok{/}\NormalTok{refs}\OperatorTok{/}\NormalTok{heads}\OperatorTok{/}\NormalTok{main}\OperatorTok{/}\NormalTok{images}\OperatorTok{/}\NormalTok{berkeley.jpg }\OperatorTok{{-}}\NormalTok{O berkeley.jpg}
\OperatorTok{!}\NormalTok{wget https:}\OperatorTok{//}\NormalTok{github.com}\OperatorTok{/}\NormalTok{sfoucher}\OperatorTok{/}\NormalTok{TraitementImagesPythonVol1}\OperatorTok{/}\NormalTok{raw}\OperatorTok{/}\NormalTok{refs}\OperatorTok{/}\NormalTok{heads}\OperatorTok{/}\NormalTok{main}\OperatorTok{/}\NormalTok{data}\OperatorTok{/}\NormalTok{chapitre01}\OperatorTok{/}\NormalTok{subset\_1\_of\_S2A\_MSIL2A\_20240625T153941\_N0510\_R011\_T18TYR\_20240625T221903\_resampled.tif }\OperatorTok{{-}}\NormalTok{O sentinel2.tif}
\end{Highlighting}
\end{Shaded}

Vérifiez que vous êtes capable de les lire :

\phantomsection\label{199b043d}
\phantomsection\label{cb4}
\begin{Shaded}
\begin{Highlighting}[]
\ControlFlowTok{with}\NormalTok{ rxr.open\_rasterio(}\StringTok{\textquotesingle{}berkeley.jpg\textquotesingle{}}\NormalTok{, mask\_and\_scale}\OperatorTok{=} \VariableTok{True}\NormalTok{) }\ImportTok{as}\NormalTok{ img\_rgb:}
    \BuiltInTok{print}\NormalTok{(img\_rgb)}
\ControlFlowTok{with}\NormalTok{ rxr.open\_rasterio(}\StringTok{\textquotesingle{}RGBNIR\_of\_S2A.tif\textquotesingle{}}\NormalTok{, mask\_and\_scale}\OperatorTok{=} \VariableTok{True}\NormalTok{) }\ImportTok{as}\NormalTok{ img\_rgbnir:}
    \BuiltInTok{print}\NormalTok{(img\_rgbnir)}
\ControlFlowTok{with}\NormalTok{ rxr.open\_rasterio(}\StringTok{\textquotesingle{}sentinel2.tif\textquotesingle{}}\NormalTok{, mask\_and\_scale}\OperatorTok{=} \VariableTok{True}\NormalTok{) }\ImportTok{as}\NormalTok{ img\_s2:}
    \BuiltInTok{print}\NormalTok{(img\_s2)}
\end{Highlighting}
\end{Shaded}

\subsection{\texorpdfstring{{4.2} Qu'est ce que l'information
spectrale?}{4.2 Qu'est ce que l'information spectrale?}}\label{quest-ce-que-linformation-spectrale}

L'information spectrale touche à l'exploitation de la dimension
spectrale des images (c.à.d le long des bandes spectrales de l'image).
La taille de cette dimension spectrale dépend du type de capteurs
considéré. Un capteur à très haute résolution spatiale par exemple aura
très peu de bandes (4 ou 5). Un capteur multispectral pourra contenir
une quinzaine de bande. À l'autre extrême, on trouvera les capteurs
hyperspectraux qui peuvent contenir des centaines de bandes spectrales.

Pour une surface donnée, la forme des valeurs le long de l'axe spectrale
caractérise le type de matériau observé ainsi que son état. On parle
souvent alors de signature spectrale. On peut voir celle-ci comme une
généralisation de la couleur d'un matériau au delà des bandes visibles
du spectre. L'exploitation de ces signatures spectrales est probablement
un des principes les plus importants en télédétection qui le distingue
de la vison par ordinateur.

\subsection{\texorpdfstring{{4.3} Indices
spectraux}{4.3 Indices spectraux}}\label{indices-spectraux}

Il existe une vaste littérature sur les indices spectraux, le choix d'un
indice plutôt qu'un autre dépend fortement de l'application visée, nous
allons simplement couvrir les principes de base ici.

Le principe d'un indice spectral consiste à mettre en valeur certaines
caractéristiques du spectre comme des pentes, des gradients, etc.

\subsection{\texorpdfstring{{4.4} Réduction de
dimension}{4.4 Réduction de dimension}}\label{ruxe9duction-de-dimension}

La réduction de dimension vise à ne retenir que l'information principale
d'un jeu de données. L'objectif est parfois d'éliminer le bruit d'un
capteur ou de faciliter la visualisation en ne retenant que 3 bandes
principales. Le degré d'information est souvent mesuré par la variance
d'une bande, c'est à dire son contraste. L'analyse en composante
principale vise alors à ranger l'information contenue dans une image en
ordre de variance décroissante.

\subsubsection{\texorpdfstring{{4.4.1} Analyses en composantes
principales}{4.4.1 Analyses en composantes principales}}\label{analyses-en-composantes-principales}

L'analyse en composantes principales (ACP) est probablement la plus
employée. En théorie, l'ACP n'est valide seulement que sur des données
Gaussiennes c'est à dire que le nuage de points des données a la forme
d'une ellipse à N dimensions. Cette ellipse est caractérisée par des
directions principales (grand axe versus petit axe). La première
composante est celle du grand axe de l'ellipse pour laquelle la donnée
présente le maximum de variation. L'ACP est une décomposition linéaire,
c'est à dire que les composantes principales sont des sommes pondérées
des valeurs originales.

\subsection{\texorpdfstring{{4.5} Exercices de
révision}{4.5 Exercices de révision}}\label{exercices-de-ruxe9vision}

\end{document}

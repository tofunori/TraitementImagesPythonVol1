\documentclass[11pt]{article}

    \usepackage[breakable]{tcolorbox}
    \usepackage{parskip} % Stop auto-indenting (to mimic markdown behaviour)
    

    % Basic figure setup, for now with no caption control since it's done
    % automatically by Pandoc (which extracts ![](path) syntax from Markdown).
    \usepackage{graphicx}
    % Keep aspect ratio if custom image width or height is specified
    \setkeys{Gin}{keepaspectratio}
    % Maintain compatibility with old templates. Remove in nbconvert 6.0
    \let\Oldincludegraphics\includegraphics
    % Ensure that by default, figures have no caption (until we provide a
    % proper Figure object with a Caption API and a way to capture that
    % in the conversion process - todo).
    \usepackage{caption}
    \DeclareCaptionFormat{nocaption}{}
    \captionsetup{format=nocaption,aboveskip=0pt,belowskip=0pt}

    \usepackage{float}
    \floatplacement{figure}{H} % forces figures to be placed at the correct location
    \usepackage{xcolor} % Allow colors to be defined
    \usepackage{enumerate} % Needed for markdown enumerations to work
    \usepackage{geometry} % Used to adjust the document margins
    \usepackage{amsmath} % Equations
    \usepackage{amssymb} % Equations
    \usepackage{textcomp} % defines textquotesingle
    % Hack from http://tex.stackexchange.com/a/47451/13684:
    \AtBeginDocument{%
        \def\PYZsq{\textquotesingle}% Upright quotes in Pygmentized code
    }
    \usepackage{upquote} % Upright quotes for verbatim code
    \usepackage{eurosym} % defines \euro

    \usepackage{iftex}
    \ifPDFTeX
        \usepackage[T1]{fontenc}
        \IfFileExists{alphabeta.sty}{
              \usepackage{alphabeta}
          }{
              \usepackage[mathletters]{ucs}
              \usepackage[utf8x]{inputenc}
          }
    \else
        \usepackage{fontspec}
        \usepackage{unicode-math}
    \fi

    \usepackage{fancyvrb} % verbatim replacement that allows latex
    \usepackage{grffile} % extends the file name processing of package graphics
                         % to support a larger range
    \makeatletter % fix for old versions of grffile with XeLaTeX
    \@ifpackagelater{grffile}{2019/11/01}
    {
      % Do nothing on new versions
    }
    {
      \def\Gread@@xetex#1{%
        \IfFileExists{"\Gin@base".bb}%
        {\Gread@eps{\Gin@base.bb}}%
        {\Gread@@xetex@aux#1}%
      }
    }
    \makeatother
    \usepackage[Export]{adjustbox} % Used to constrain images to a maximum size
    \adjustboxset{max size={0.9\linewidth}{0.9\paperheight}}

    % The hyperref package gives us a pdf with properly built
    % internal navigation ('pdf bookmarks' for the table of contents,
    % internal cross-reference links, web links for URLs, etc.)
    \usepackage{hyperref}
    % The default LaTeX title has an obnoxious amount of whitespace. By default,
    % titling removes some of it. It also provides customization options.
    \usepackage{titling}
    \usepackage{longtable} % longtable support required by pandoc >1.10
    \usepackage{booktabs}  % table support for pandoc > 1.12.2
    \usepackage{array}     % table support for pandoc >= 2.11.3
    \usepackage{calc}      % table minipage width calculation for pandoc >= 2.11.1
    \usepackage[inline]{enumitem} % IRkernel/repr support (it uses the enumerate* environment)
    \usepackage[normalem]{ulem} % ulem is needed to support strikethroughs (\sout)
                                % normalem makes italics be italics, not underlines
    \usepackage{soul}      % strikethrough (\st) support for pandoc >= 3.0.0
    \usepackage{mathrsfs}
    

    
    % Colors for the hyperref package
    \definecolor{urlcolor}{rgb}{0,.145,.698}
    \definecolor{linkcolor}{rgb}{.71,0.21,0.01}
    \definecolor{citecolor}{rgb}{.12,.54,.11}

    % ANSI colors
    \definecolor{ansi-black}{HTML}{3E424D}
    \definecolor{ansi-black-intense}{HTML}{282C36}
    \definecolor{ansi-red}{HTML}{E75C58}
    \definecolor{ansi-red-intense}{HTML}{B22B31}
    \definecolor{ansi-green}{HTML}{00A250}
    \definecolor{ansi-green-intense}{HTML}{007427}
    \definecolor{ansi-yellow}{HTML}{DDB62B}
    \definecolor{ansi-yellow-intense}{HTML}{B27D12}
    \definecolor{ansi-blue}{HTML}{208FFB}
    \definecolor{ansi-blue-intense}{HTML}{0065CA}
    \definecolor{ansi-magenta}{HTML}{D160C4}
    \definecolor{ansi-magenta-intense}{HTML}{A03196}
    \definecolor{ansi-cyan}{HTML}{60C6C8}
    \definecolor{ansi-cyan-intense}{HTML}{258F8F}
    \definecolor{ansi-white}{HTML}{C5C1B4}
    \definecolor{ansi-white-intense}{HTML}{A1A6B2}
    \definecolor{ansi-default-inverse-fg}{HTML}{FFFFFF}
    \definecolor{ansi-default-inverse-bg}{HTML}{000000}

    % common color for the border for error outputs.
    \definecolor{outerrorbackground}{HTML}{FFDFDF}

    % commands and environments needed by pandoc snippets
    % extracted from the output of `pandoc -s`
    \providecommand{\tightlist}{%
      \setlength{\itemsep}{0pt}\setlength{\parskip}{0pt}}
    \DefineVerbatimEnvironment{Highlighting}{Verbatim}{commandchars=\\\{\}}
    % Add ',fontsize=\small' for more characters per line
    \newenvironment{Shaded}{}{}
    \newcommand{\KeywordTok}[1]{\textcolor[rgb]{0.00,0.44,0.13}{\textbf{{#1}}}}
    \newcommand{\DataTypeTok}[1]{\textcolor[rgb]{0.56,0.13,0.00}{{#1}}}
    \newcommand{\DecValTok}[1]{\textcolor[rgb]{0.25,0.63,0.44}{{#1}}}
    \newcommand{\BaseNTok}[1]{\textcolor[rgb]{0.25,0.63,0.44}{{#1}}}
    \newcommand{\FloatTok}[1]{\textcolor[rgb]{0.25,0.63,0.44}{{#1}}}
    \newcommand{\CharTok}[1]{\textcolor[rgb]{0.25,0.44,0.63}{{#1}}}
    \newcommand{\StringTok}[1]{\textcolor[rgb]{0.25,0.44,0.63}{{#1}}}
    \newcommand{\CommentTok}[1]{\textcolor[rgb]{0.38,0.63,0.69}{\textit{{#1}}}}
    \newcommand{\OtherTok}[1]{\textcolor[rgb]{0.00,0.44,0.13}{{#1}}}
    \newcommand{\AlertTok}[1]{\textcolor[rgb]{1.00,0.00,0.00}{\textbf{{#1}}}}
    \newcommand{\FunctionTok}[1]{\textcolor[rgb]{0.02,0.16,0.49}{{#1}}}
    \newcommand{\RegionMarkerTok}[1]{{#1}}
    \newcommand{\ErrorTok}[1]{\textcolor[rgb]{1.00,0.00,0.00}{\textbf{{#1}}}}
    \newcommand{\NormalTok}[1]{{#1}}

    % Additional commands for more recent versions of Pandoc
    \newcommand{\ConstantTok}[1]{\textcolor[rgb]{0.53,0.00,0.00}{{#1}}}
    \newcommand{\SpecialCharTok}[1]{\textcolor[rgb]{0.25,0.44,0.63}{{#1}}}
    \newcommand{\VerbatimStringTok}[1]{\textcolor[rgb]{0.25,0.44,0.63}{{#1}}}
    \newcommand{\SpecialStringTok}[1]{\textcolor[rgb]{0.73,0.40,0.53}{{#1}}}
    \newcommand{\ImportTok}[1]{{#1}}
    \newcommand{\DocumentationTok}[1]{\textcolor[rgb]{0.73,0.13,0.13}{\textit{{#1}}}}
    \newcommand{\AnnotationTok}[1]{\textcolor[rgb]{0.38,0.63,0.69}{\textbf{\textit{{#1}}}}}
    \newcommand{\CommentVarTok}[1]{\textcolor[rgb]{0.38,0.63,0.69}{\textbf{\textit{{#1}}}}}
    \newcommand{\VariableTok}[1]{\textcolor[rgb]{0.10,0.09,0.49}{{#1}}}
    \newcommand{\ControlFlowTok}[1]{\textcolor[rgb]{0.00,0.44,0.13}{\textbf{{#1}}}}
    \newcommand{\OperatorTok}[1]{\textcolor[rgb]{0.40,0.40,0.40}{{#1}}}
    \newcommand{\BuiltInTok}[1]{{#1}}
    \newcommand{\ExtensionTok}[1]{{#1}}
    \newcommand{\PreprocessorTok}[1]{\textcolor[rgb]{0.74,0.48,0.00}{{#1}}}
    \newcommand{\AttributeTok}[1]{\textcolor[rgb]{0.49,0.56,0.16}{{#1}}}
    \newcommand{\InformationTok}[1]{\textcolor[rgb]{0.38,0.63,0.69}{\textbf{\textit{{#1}}}}}
    \newcommand{\WarningTok}[1]{\textcolor[rgb]{0.38,0.63,0.69}{\textbf{\textit{{#1}}}}}


    % Define a nice break command that doesn't care if a line doesn't already
    % exist.
    \def\br{\hspace*{\fill} \\* }
    % Math Jax compatibility definitions
    \def\gt{>}
    \def\lt{<}
    \let\Oldtex\TeX
    \let\Oldlatex\LaTeX
    \renewcommand{\TeX}{\textrm{\Oldtex}}
    \renewcommand{\LaTeX}{\textrm{\Oldlatex}}
    % Document parameters
    % Document title
    \title{04-TransformationSpatiales}
    
    
    
    
    
    
    
% Pygments definitions
\makeatletter
\def\PY@reset{\let\PY@it=\relax \let\PY@bf=\relax%
    \let\PY@ul=\relax \let\PY@tc=\relax%
    \let\PY@bc=\relax \let\PY@ff=\relax}
\def\PY@tok#1{\csname PY@tok@#1\endcsname}
\def\PY@toks#1+{\ifx\relax#1\empty\else%
    \PY@tok{#1}\expandafter\PY@toks\fi}
\def\PY@do#1{\PY@bc{\PY@tc{\PY@ul{%
    \PY@it{\PY@bf{\PY@ff{#1}}}}}}}
\def\PY#1#2{\PY@reset\PY@toks#1+\relax+\PY@do{#2}}

\@namedef{PY@tok@w}{\def\PY@tc##1{\textcolor[rgb]{0.73,0.73,0.73}{##1}}}
\@namedef{PY@tok@c}{\let\PY@it=\textit\def\PY@tc##1{\textcolor[rgb]{0.24,0.48,0.48}{##1}}}
\@namedef{PY@tok@cp}{\def\PY@tc##1{\textcolor[rgb]{0.61,0.40,0.00}{##1}}}
\@namedef{PY@tok@k}{\let\PY@bf=\textbf\def\PY@tc##1{\textcolor[rgb]{0.00,0.50,0.00}{##1}}}
\@namedef{PY@tok@kp}{\def\PY@tc##1{\textcolor[rgb]{0.00,0.50,0.00}{##1}}}
\@namedef{PY@tok@kt}{\def\PY@tc##1{\textcolor[rgb]{0.69,0.00,0.25}{##1}}}
\@namedef{PY@tok@o}{\def\PY@tc##1{\textcolor[rgb]{0.40,0.40,0.40}{##1}}}
\@namedef{PY@tok@ow}{\let\PY@bf=\textbf\def\PY@tc##1{\textcolor[rgb]{0.67,0.13,1.00}{##1}}}
\@namedef{PY@tok@nb}{\def\PY@tc##1{\textcolor[rgb]{0.00,0.50,0.00}{##1}}}
\@namedef{PY@tok@nf}{\def\PY@tc##1{\textcolor[rgb]{0.00,0.00,1.00}{##1}}}
\@namedef{PY@tok@nc}{\let\PY@bf=\textbf\def\PY@tc##1{\textcolor[rgb]{0.00,0.00,1.00}{##1}}}
\@namedef{PY@tok@nn}{\let\PY@bf=\textbf\def\PY@tc##1{\textcolor[rgb]{0.00,0.00,1.00}{##1}}}
\@namedef{PY@tok@ne}{\let\PY@bf=\textbf\def\PY@tc##1{\textcolor[rgb]{0.80,0.25,0.22}{##1}}}
\@namedef{PY@tok@nv}{\def\PY@tc##1{\textcolor[rgb]{0.10,0.09,0.49}{##1}}}
\@namedef{PY@tok@no}{\def\PY@tc##1{\textcolor[rgb]{0.53,0.00,0.00}{##1}}}
\@namedef{PY@tok@nl}{\def\PY@tc##1{\textcolor[rgb]{0.46,0.46,0.00}{##1}}}
\@namedef{PY@tok@ni}{\let\PY@bf=\textbf\def\PY@tc##1{\textcolor[rgb]{0.44,0.44,0.44}{##1}}}
\@namedef{PY@tok@na}{\def\PY@tc##1{\textcolor[rgb]{0.41,0.47,0.13}{##1}}}
\@namedef{PY@tok@nt}{\let\PY@bf=\textbf\def\PY@tc##1{\textcolor[rgb]{0.00,0.50,0.00}{##1}}}
\@namedef{PY@tok@nd}{\def\PY@tc##1{\textcolor[rgb]{0.67,0.13,1.00}{##1}}}
\@namedef{PY@tok@s}{\def\PY@tc##1{\textcolor[rgb]{0.73,0.13,0.13}{##1}}}
\@namedef{PY@tok@sd}{\let\PY@it=\textit\def\PY@tc##1{\textcolor[rgb]{0.73,0.13,0.13}{##1}}}
\@namedef{PY@tok@si}{\let\PY@bf=\textbf\def\PY@tc##1{\textcolor[rgb]{0.64,0.35,0.47}{##1}}}
\@namedef{PY@tok@se}{\let\PY@bf=\textbf\def\PY@tc##1{\textcolor[rgb]{0.67,0.36,0.12}{##1}}}
\@namedef{PY@tok@sr}{\def\PY@tc##1{\textcolor[rgb]{0.64,0.35,0.47}{##1}}}
\@namedef{PY@tok@ss}{\def\PY@tc##1{\textcolor[rgb]{0.10,0.09,0.49}{##1}}}
\@namedef{PY@tok@sx}{\def\PY@tc##1{\textcolor[rgb]{0.00,0.50,0.00}{##1}}}
\@namedef{PY@tok@m}{\def\PY@tc##1{\textcolor[rgb]{0.40,0.40,0.40}{##1}}}
\@namedef{PY@tok@gh}{\let\PY@bf=\textbf\def\PY@tc##1{\textcolor[rgb]{0.00,0.00,0.50}{##1}}}
\@namedef{PY@tok@gu}{\let\PY@bf=\textbf\def\PY@tc##1{\textcolor[rgb]{0.50,0.00,0.50}{##1}}}
\@namedef{PY@tok@gd}{\def\PY@tc##1{\textcolor[rgb]{0.63,0.00,0.00}{##1}}}
\@namedef{PY@tok@gi}{\def\PY@tc##1{\textcolor[rgb]{0.00,0.52,0.00}{##1}}}
\@namedef{PY@tok@gr}{\def\PY@tc##1{\textcolor[rgb]{0.89,0.00,0.00}{##1}}}
\@namedef{PY@tok@ge}{\let\PY@it=\textit}
\@namedef{PY@tok@gs}{\let\PY@bf=\textbf}
\@namedef{PY@tok@gp}{\let\PY@bf=\textbf\def\PY@tc##1{\textcolor[rgb]{0.00,0.00,0.50}{##1}}}
\@namedef{PY@tok@go}{\def\PY@tc##1{\textcolor[rgb]{0.44,0.44,0.44}{##1}}}
\@namedef{PY@tok@gt}{\def\PY@tc##1{\textcolor[rgb]{0.00,0.27,0.87}{##1}}}
\@namedef{PY@tok@err}{\def\PY@bc##1{{\setlength{\fboxsep}{\string -\fboxrule}\fcolorbox[rgb]{1.00,0.00,0.00}{1,1,1}{\strut ##1}}}}
\@namedef{PY@tok@kc}{\let\PY@bf=\textbf\def\PY@tc##1{\textcolor[rgb]{0.00,0.50,0.00}{##1}}}
\@namedef{PY@tok@kd}{\let\PY@bf=\textbf\def\PY@tc##1{\textcolor[rgb]{0.00,0.50,0.00}{##1}}}
\@namedef{PY@tok@kn}{\let\PY@bf=\textbf\def\PY@tc##1{\textcolor[rgb]{0.00,0.50,0.00}{##1}}}
\@namedef{PY@tok@kr}{\let\PY@bf=\textbf\def\PY@tc##1{\textcolor[rgb]{0.00,0.50,0.00}{##1}}}
\@namedef{PY@tok@bp}{\def\PY@tc##1{\textcolor[rgb]{0.00,0.50,0.00}{##1}}}
\@namedef{PY@tok@fm}{\def\PY@tc##1{\textcolor[rgb]{0.00,0.00,1.00}{##1}}}
\@namedef{PY@tok@vc}{\def\PY@tc##1{\textcolor[rgb]{0.10,0.09,0.49}{##1}}}
\@namedef{PY@tok@vg}{\def\PY@tc##1{\textcolor[rgb]{0.10,0.09,0.49}{##1}}}
\@namedef{PY@tok@vi}{\def\PY@tc##1{\textcolor[rgb]{0.10,0.09,0.49}{##1}}}
\@namedef{PY@tok@vm}{\def\PY@tc##1{\textcolor[rgb]{0.10,0.09,0.49}{##1}}}
\@namedef{PY@tok@sa}{\def\PY@tc##1{\textcolor[rgb]{0.73,0.13,0.13}{##1}}}
\@namedef{PY@tok@sb}{\def\PY@tc##1{\textcolor[rgb]{0.73,0.13,0.13}{##1}}}
\@namedef{PY@tok@sc}{\def\PY@tc##1{\textcolor[rgb]{0.73,0.13,0.13}{##1}}}
\@namedef{PY@tok@dl}{\def\PY@tc##1{\textcolor[rgb]{0.73,0.13,0.13}{##1}}}
\@namedef{PY@tok@s2}{\def\PY@tc##1{\textcolor[rgb]{0.73,0.13,0.13}{##1}}}
\@namedef{PY@tok@sh}{\def\PY@tc##1{\textcolor[rgb]{0.73,0.13,0.13}{##1}}}
\@namedef{PY@tok@s1}{\def\PY@tc##1{\textcolor[rgb]{0.73,0.13,0.13}{##1}}}
\@namedef{PY@tok@mb}{\def\PY@tc##1{\textcolor[rgb]{0.40,0.40,0.40}{##1}}}
\@namedef{PY@tok@mf}{\def\PY@tc##1{\textcolor[rgb]{0.40,0.40,0.40}{##1}}}
\@namedef{PY@tok@mh}{\def\PY@tc##1{\textcolor[rgb]{0.40,0.40,0.40}{##1}}}
\@namedef{PY@tok@mi}{\def\PY@tc##1{\textcolor[rgb]{0.40,0.40,0.40}{##1}}}
\@namedef{PY@tok@il}{\def\PY@tc##1{\textcolor[rgb]{0.40,0.40,0.40}{##1}}}
\@namedef{PY@tok@mo}{\def\PY@tc##1{\textcolor[rgb]{0.40,0.40,0.40}{##1}}}
\@namedef{PY@tok@ch}{\let\PY@it=\textit\def\PY@tc##1{\textcolor[rgb]{0.24,0.48,0.48}{##1}}}
\@namedef{PY@tok@cm}{\let\PY@it=\textit\def\PY@tc##1{\textcolor[rgb]{0.24,0.48,0.48}{##1}}}
\@namedef{PY@tok@cpf}{\let\PY@it=\textit\def\PY@tc##1{\textcolor[rgb]{0.24,0.48,0.48}{##1}}}
\@namedef{PY@tok@c1}{\let\PY@it=\textit\def\PY@tc##1{\textcolor[rgb]{0.24,0.48,0.48}{##1}}}
\@namedef{PY@tok@cs}{\let\PY@it=\textit\def\PY@tc##1{\textcolor[rgb]{0.24,0.48,0.48}{##1}}}

\def\PYZbs{\char`\\}
\def\PYZus{\char`\_}
\def\PYZob{\char`\{}
\def\PYZcb{\char`\}}
\def\PYZca{\char`\^}
\def\PYZam{\char`\&}
\def\PYZlt{\char`\<}
\def\PYZgt{\char`\>}
\def\PYZsh{\char`\#}
\def\PYZpc{\char`\%}
\def\PYZdl{\char`\$}
\def\PYZhy{\char`\-}
\def\PYZsq{\char`\'}
\def\PYZdq{\char`\"}
\def\PYZti{\char`\~}
% for compatibility with earlier versions
\def\PYZat{@}
\def\PYZlb{[}
\def\PYZrb{]}
\makeatother


    % For linebreaks inside Verbatim environment from package fancyvrb.
    \makeatletter
        \newbox\Wrappedcontinuationbox
        \newbox\Wrappedvisiblespacebox
        \newcommand*\Wrappedvisiblespace {\textcolor{red}{\textvisiblespace}}
        \newcommand*\Wrappedcontinuationsymbol {\textcolor{red}{\llap{\tiny$\m@th\hookrightarrow$}}}
        \newcommand*\Wrappedcontinuationindent {3ex }
        \newcommand*\Wrappedafterbreak {\kern\Wrappedcontinuationindent\copy\Wrappedcontinuationbox}
        % Take advantage of the already applied Pygments mark-up to insert
        % potential linebreaks for TeX processing.
        %        {, <, #, %, $, ' and ": go to next line.
        %        _, }, ^, &, >, - and ~: stay at end of broken line.
        % Use of \textquotesingle for straight quote.
        \newcommand*\Wrappedbreaksatspecials {%
            \def\PYGZus{\discretionary{\char`\_}{\Wrappedafterbreak}{\char`\_}}%
            \def\PYGZob{\discretionary{}{\Wrappedafterbreak\char`\{}{\char`\{}}%
            \def\PYGZcb{\discretionary{\char`\}}{\Wrappedafterbreak}{\char`\}}}%
            \def\PYGZca{\discretionary{\char`\^}{\Wrappedafterbreak}{\char`\^}}%
            \def\PYGZam{\discretionary{\char`\&}{\Wrappedafterbreak}{\char`\&}}%
            \def\PYGZlt{\discretionary{}{\Wrappedafterbreak\char`\<}{\char`\<}}%
            \def\PYGZgt{\discretionary{\char`\>}{\Wrappedafterbreak}{\char`\>}}%
            \def\PYGZsh{\discretionary{}{\Wrappedafterbreak\char`\#}{\char`\#}}%
            \def\PYGZpc{\discretionary{}{\Wrappedafterbreak\char`\%}{\char`\%}}%
            \def\PYGZdl{\discretionary{}{\Wrappedafterbreak\char`\$}{\char`\$}}%
            \def\PYGZhy{\discretionary{\char`\-}{\Wrappedafterbreak}{\char`\-}}%
            \def\PYGZsq{\discretionary{}{\Wrappedafterbreak\textquotesingle}{\textquotesingle}}%
            \def\PYGZdq{\discretionary{}{\Wrappedafterbreak\char`\"}{\char`\"}}%
            \def\PYGZti{\discretionary{\char`\~}{\Wrappedafterbreak}{\char`\~}}%
        }
        % Some characters . , ; ? ! / are not pygmentized.
        % This macro makes them "active" and they will insert potential linebreaks
        \newcommand*\Wrappedbreaksatpunct {%
            \lccode`\~`\.\lowercase{\def~}{\discretionary{\hbox{\char`\.}}{\Wrappedafterbreak}{\hbox{\char`\.}}}%
            \lccode`\~`\,\lowercase{\def~}{\discretionary{\hbox{\char`\,}}{\Wrappedafterbreak}{\hbox{\char`\,}}}%
            \lccode`\~`\;\lowercase{\def~}{\discretionary{\hbox{\char`\;}}{\Wrappedafterbreak}{\hbox{\char`\;}}}%
            \lccode`\~`\:\lowercase{\def~}{\discretionary{\hbox{\char`\:}}{\Wrappedafterbreak}{\hbox{\char`\:}}}%
            \lccode`\~`\?\lowercase{\def~}{\discretionary{\hbox{\char`\?}}{\Wrappedafterbreak}{\hbox{\char`\?}}}%
            \lccode`\~`\!\lowercase{\def~}{\discretionary{\hbox{\char`\!}}{\Wrappedafterbreak}{\hbox{\char`\!}}}%
            \lccode`\~`\/\lowercase{\def~}{\discretionary{\hbox{\char`\/}}{\Wrappedafterbreak}{\hbox{\char`\/}}}%
            \catcode`\.\active
            \catcode`\,\active
            \catcode`\;\active
            \catcode`\:\active
            \catcode`\?\active
            \catcode`\!\active
            \catcode`\/\active
            \lccode`\~`\~
        }
    \makeatother

    \let\OriginalVerbatim=\Verbatim
    \makeatletter
    \renewcommand{\Verbatim}[1][1]{%
        %\parskip\z@skip
        \sbox\Wrappedcontinuationbox {\Wrappedcontinuationsymbol}%
        \sbox\Wrappedvisiblespacebox {\FV@SetupFont\Wrappedvisiblespace}%
        \def\FancyVerbFormatLine ##1{\hsize\linewidth
            \vtop{\raggedright\hyphenpenalty\z@\exhyphenpenalty\z@
                \doublehyphendemerits\z@\finalhyphendemerits\z@
                \strut ##1\strut}%
        }%
        % If the linebreak is at a space, the latter will be displayed as visible
        % space at end of first line, and a continuation symbol starts next line.
        % Stretch/shrink are however usually zero for typewriter font.
        \def\FV@Space {%
            \nobreak\hskip\z@ plus\fontdimen3\font minus\fontdimen4\font
            \discretionary{\copy\Wrappedvisiblespacebox}{\Wrappedafterbreak}
            {\kern\fontdimen2\font}%
        }%

        % Allow breaks at special characters using \PYG... macros.
        \Wrappedbreaksatspecials
        % Breaks at punctuation characters . , ; ? ! and / need catcode=\active
        \OriginalVerbatim[#1,codes*=\Wrappedbreaksatpunct]%
    }
    \makeatother

    % Exact colors from NB
    \definecolor{incolor}{HTML}{303F9F}
    \definecolor{outcolor}{HTML}{D84315}
    \definecolor{cellborder}{HTML}{CFCFCF}
    \definecolor{cellbackground}{HTML}{F7F7F7}

    % prompt
    \makeatletter
    \newcommand{\boxspacing}{\kern\kvtcb@left@rule\kern\kvtcb@boxsep}
    \makeatother
    \newcommand{\prompt}[4]{
        {\ttfamily\llap{{\color{#2}[#3]:\hspace{3pt}#4}}\vspace{-\baselineskip}}
    }
    

    
    % Prevent overflowing lines due to hard-to-break entities
    \sloppy
    % Setup hyperref package
    \hypersetup{
      breaklinks=true,  % so long urls are correctly broken across lines
      colorlinks=true,
      urlcolor=urlcolor,
      linkcolor=linkcolor,
      citecolor=citecolor,
      }
    % Slightly bigger margins than the latex defaults
    
    \geometry{verbose,tmargin=1in,bmargin=1in,lmargin=1in,rmargin=1in}
    
    

\begin{document}
    
    \maketitle
    
    

    
    \hypertarget{sec-chap04}{%
\section{Transformations spatiales}\label{sec-chap04}}

\hypertarget{rocket-pruxe9ambule}{%
\subsection{:rocket: Préambule}\label{rocket-pruxe9ambule}}

Assurez-vous de lire ce préambule avant d'exécutez le reste du notebook.

\hypertarget{dart-objectifs}{%
\subsubsection{:dart: Objectifs}\label{dart-objectifs}}

Dans ce chapitre, nous abordons quelques techniques de traitement
d'images dans le domaine spatial uniquement. Ce chapitre est aussi
disponible sous la forme d'un notebook Python sur Google Colab:

\href{https://colab.research.google.com/github/sfoucher/TraitementImagesPythonVol1/blob/main/notebooks/04-TransformationSpatiales.ipynb}{\includegraphics{images/colab-badge.svg}}

\hypertarget{librairies}{%
\subsubsection{Librairies}\label{librairies}}

Les librairies qui vont être explorées dans ce chapitre sont les
suivantes:

\begin{itemize}
\item
  \href{https://scipy.org/}{SciPy -}
\item
  \href{https://numpy.org/}{NumPy -}
\item
  \href{https://pypi.org/project/opencv-python/}{opencv-python · PyPI}
\item
  \href{https://scikit-image.org/}{scikit-image}
\item
  \href{https://rasterio.readthedocs.io/en/stable/}{Rasterio}
\item
  \href{https://docs.xarray.dev/en/stable/}{Xarray}
\item
  \href{https://corteva.github.io/rioxarray/stable/index.html}{rioxarray}
\end{itemize}

Dans l'environnement Google Colab, seul \texttt{rioxarray} doit être
installés:

    \begin{tcolorbox}[breakable, size=fbox, boxrule=1pt, pad at break*=1mm,colback=cellbackground, colframe=cellborder]
\prompt{In}{incolor}{1}{\boxspacing}
\begin{Verbatim}[commandchars=\\\{\}]
\PY{o}{\PYZpc{}\PYZpc{}capture}
\PY{err}{!}\PY{n}{pip} \PY{n}{install} \PY{o}{\PYZhy{}}\PY{n}{qU} \PY{n}{matplotlib} \PY{n}{rioxarray} \PY{n}{xrscipy} \PY{n}{scikit}\PY{o}{\PYZhy{}}\PY{n}{image}
\end{Verbatim}
\end{tcolorbox}

    Vérifier les importations:

    \begin{tcolorbox}[breakable, size=fbox, boxrule=1pt, pad at break*=1mm,colback=cellbackground, colframe=cellborder]
\prompt{In}{incolor}{2}{\boxspacing}
\begin{Verbatim}[commandchars=\\\{\}]
\PY{c+c1}{\PYZsh{}| eval: true}
\PY{k+kn}{import} \PY{n+nn}{numpy} \PY{k}{as} \PY{n+nn}{np}
\PY{k+kn}{import} \PY{n+nn}{rioxarray} \PY{k}{as} \PY{n+nn}{rxr}
\PY{k+kn}{from} \PY{n+nn}{scipy} \PY{k+kn}{import} \PY{n}{signal}
\PY{k+kn}{import} \PY{n+nn}{xarray} \PY{k}{as} \PY{n+nn}{xr}
\PY{k+kn}{import} \PY{n+nn}{xrscipy}
\PY{k+kn}{import} \PY{n+nn}{matplotlib}\PY{n+nn}{.}\PY{n+nn}{pyplot} \PY{k}{as} \PY{n+nn}{plt}
\end{Verbatim}
\end{tcolorbox}

    \hypertarget{images-utilisuxe9es}{%
\subsubsection{Images utilisées}\label{images-utilisuxe9es}}

Nous allons utilisez les images suivantes dans ce chapitre:

    \begin{tcolorbox}[breakable, size=fbox, boxrule=1pt, pad at break*=1mm,colback=cellbackground, colframe=cellborder]
\prompt{In}{incolor}{3}{\boxspacing}
\begin{Verbatim}[commandchars=\\\{\}]
\PY{o}{\PYZpc{}\PYZpc{}capture}
\PY{err}{!}\PY{n}{wget} \PY{n}{https}\PY{p}{:}\PY{o}{/}\PY{o}{/}\PY{n}{github}\PY{o}{.}\PY{n}{com}\PY{o}{/}\PY{n}{sfoucher}\PY{o}{/}\PY{n}{TraitementImagesPythonVol1}\PY{o}{/}\PY{n}{raw}\PY{o}{/}\PY{n}{refs}\PY{o}{/}\PY{n}{heads}\PY{o}{/}\PY{n}{main}\PY{o}{/}\PY{n}{data}\PY{o}{/}\PY{n}{chapitre01}\PY{o}{/}\PY{n}{subset\PYZus{}RGBNIR\PYZus{}of\PYZus{}S2A\PYZus{}MSIL2A\PYZus{}20240625T153941\PYZus{}N0510\PYZus{}R011\PYZus{}T18TYR\PYZus{}20240625T221903}\PY{o}{.}\PY{n}{tif} \PY{o}{\PYZhy{}}\PY{n}{O} \PY{n}{RGBNIR\PYZus{}of\PYZus{}S2A}\PY{o}{.}\PY{n}{tif}
\PY{err}{!}\PY{n}{wget} \PY{n}{https}\PY{p}{:}\PY{o}{/}\PY{o}{/}\PY{n}{github}\PY{o}{.}\PY{n}{com}\PY{o}{/}\PY{n}{sfoucher}\PY{o}{/}\PY{n}{opengeos}\PY{o}{\PYZhy{}}\PY{n}{data}\PY{o}{/}\PY{n}{raw}\PY{o}{/}\PY{n}{refs}\PY{o}{/}\PY{n}{heads}\PY{o}{/}\PY{n}{main}\PY{o}{/}\PY{n}{raster}\PY{o}{/}\PY{n}{landsat7}\PY{o}{.}\PY{n}{tif} \PY{o}{\PYZhy{}}\PY{n}{O} \PY{n}{landsat7}\PY{o}{.}\PY{n}{tif}
\PY{err}{!}\PY{n}{wget} \PY{n}{https}\PY{p}{:}\PY{o}{/}\PY{o}{/}\PY{n}{github}\PY{o}{.}\PY{n}{com}\PY{o}{/}\PY{n}{sfoucher}\PY{o}{/}\PY{n}{opengeos}\PY{o}{\PYZhy{}}\PY{n}{data}\PY{o}{/}\PY{n}{raw}\PY{o}{/}\PY{n}{refs}\PY{o}{/}\PY{n}{heads}\PY{o}{/}\PY{n}{main}\PY{o}{/}\PY{n}{images}\PY{o}{/}\PY{n}{berkeley}\PY{o}{.}\PY{n}{jpg} \PY{o}{\PYZhy{}}\PY{n}{O} \PY{n}{berkeley}\PY{o}{.}\PY{n}{jpg}
\PY{err}{!}\PY{n}{wget} \PY{n}{https}\PY{p}{:}\PY{o}{/}\PY{o}{/}\PY{n}{github}\PY{o}{.}\PY{n}{com}\PY{o}{/}\PY{n}{sfoucher}\PY{o}{/}\PY{n}{TraitementImagesPythonVol1}\PY{o}{/}\PY{n}{raw}\PY{o}{/}\PY{n}{refs}\PY{o}{/}\PY{n}{heads}\PY{o}{/}\PY{n}{main}\PY{o}{/}\PY{n}{data}\PY{o}{/}\PY{n}{chapitre01}\PY{o}{/}\PY{n}{subset\PYZus{}0\PYZus{}of\PYZus{}S1A\PYZus{}split\PYZus{}NR\PYZus{}Cal\PYZus{}Deb\PYZus{}ML\PYZus{}Spk\PYZus{}SRGR}\PY{o}{.}\PY{n}{tif} \PY{o}{\PYZhy{}}\PY{n}{O} \PY{n}{SAR}\PY{o}{.}\PY{n}{tif}
\end{Verbatim}
\end{tcolorbox}

    Vérifiez que vous êtes capable de les lire :

    \begin{tcolorbox}[breakable, size=fbox, boxrule=1pt, pad at break*=1mm,colback=cellbackground, colframe=cellborder]
\prompt{In}{incolor}{4}{\boxspacing}
\begin{Verbatim}[commandchars=\\\{\}]
\PY{c+c1}{\PYZsh{}| eval: true}
\PY{c+c1}{\PYZsh{}| output: false}

\PY{k}{with} \PY{n}{rxr}\PY{o}{.}\PY{n}{open\PYZus{}rasterio}\PY{p}{(}\PY{l+s+s1}{\PYZsq{}}\PY{l+s+s1}{berkeley.jpg}\PY{l+s+s1}{\PYZsq{}}\PY{p}{,} \PY{n}{mask\PYZus{}and\PYZus{}scale}\PY{o}{=} \PY{k+kc}{True}\PY{p}{)} \PY{k}{as} \PY{n}{img\PYZus{}rgb}\PY{p}{:}
    \PY{n+nb}{print}\PY{p}{(}\PY{n}{img\PYZus{}rgb}\PY{p}{)}
\PY{k}{with} \PY{n}{rxr}\PY{o}{.}\PY{n}{open\PYZus{}rasterio}\PY{p}{(}\PY{l+s+s1}{\PYZsq{}}\PY{l+s+s1}{RGBNIR\PYZus{}of\PYZus{}S2A.tif}\PY{l+s+s1}{\PYZsq{}}\PY{p}{,} \PY{n}{mask\PYZus{}and\PYZus{}scale}\PY{o}{=} \PY{k+kc}{True}\PY{p}{)} \PY{k}{as} \PY{n}{img\PYZus{}rgbnir}\PY{p}{:}
    \PY{n+nb}{print}\PY{p}{(}\PY{n}{img\PYZus{}rgbnir}\PY{p}{)}
\PY{k}{with} \PY{n}{rxr}\PY{o}{.}\PY{n}{open\PYZus{}rasterio}\PY{p}{(}\PY{l+s+s1}{\PYZsq{}}\PY{l+s+s1}{SAR.tif}\PY{l+s+s1}{\PYZsq{}}\PY{p}{,} \PY{n}{mask\PYZus{}and\PYZus{}scale}\PY{o}{=} \PY{k+kc}{True}\PY{p}{)} \PY{k}{as} \PY{n}{img\PYZus{}SAR}\PY{p}{:}
    \PY{n+nb}{print}\PY{p}{(}\PY{n}{img\PYZus{}SAR}\PY{p}{)}
\end{Verbatim}
\end{tcolorbox}

    \begin{Verbatim}[commandchars=\\\{\}]
<xarray.DataArray (band: 3, y: 771, x: 1311)> Size: 12MB
[3032343 values with dtype=float32]
Coordinates:
  * band         (band) int64 24B 1 2 3
  * x            (x) float64 10kB 0.5 1.5 2.5 {\ldots} 1.308e+03 1.31e+03 1.31e+03
  * y            (y) float64 6kB 0.5 1.5 2.5 3.5 4.5 {\ldots} 767.5 768.5 769.5 770.5
    spatial\_ref  int64 8B 0
<xarray.DataArray (band: 4, y: 1926, x: 2074)> Size: 64MB
[15978096 values with dtype=float32]
Coordinates:
  * band         (band) int64 32B 1 2 3 4
  * x            (x) float64 17kB 7.318e+05 7.318e+05 {\ldots} 7.525e+05 7.525e+05
  * y            (y) float64 15kB 5.041e+06 5.041e+06 {\ldots} 5.022e+06 5.022e+06
    spatial\_ref  int64 8B 0
Attributes:
    TIFFTAG\_IMAGEDESCRIPTION:  subset\_RGBNIR\_of\_S2A\_MSIL2A\_20240625T153941\_N0{\ldots}
    TIFFTAG\_XRESOLUTION:       1
    TIFFTAG\_YRESOLUTION:       1
    TIFFTAG\_RESOLUTIONUNIT:    1 (unitless)
    AREA\_OR\_POINT:             Area
    STATISTICS\_MAXIMUM:        15104
    STATISTICS\_MEAN:           1426.6252674912
    STATISTICS\_MINIMUM:        86
    STATISTICS\_STDDEV:         306.56427126942
    STATISTICS\_VALID\_PERCENT:  100
<xarray.DataArray (band: 2, y: 1188, x: 1599)> Size: 15MB
[3799224 values with dtype=float32]
Coordinates:
  * band         (band) int64 16B 1 2
  * x            (x) float64 13kB -73.98 -73.98 -73.98 {\ldots} -73.51 -73.51 -73.51
  * y            (y) float64 10kB 45.27 45.27 45.27 45.27 {\ldots} 44.93 44.93 44.93
    spatial\_ref  int64 8B 0
Attributes:
    TIFFTAG\_XRESOLUTION:     1
    TIFFTAG\_YRESOLUTION:     1
    TIFFTAG\_RESOLUTIONUNIT:  1 (unitless)
    AREA\_OR\_POINT:           Area
    \end{Verbatim}

    \begin{Verbatim}[commandchars=\\\{\}]
/home/sfoucher/miniconda3/lib/python3.10/site-packages/rioxarray/\_io.py:1143:
NotGeoreferencedWarning: Dataset has no geotransform, gcps, or rpcs. The
identity matrix will be returned.
  warnings.warn(str(rio\_warning.message), type(rio\_warning.message))  \# type:
ignore
Warning 1: TIFFReadDirectory:Sum of Photometric type-related color channels and
ExtraSamples doesn't match SamplesPerPixel. Defining non-color channels as
ExtraSamples.
Warning 1: TIFFReadDirectory:Sum of Photometric type-related color channels and
ExtraSamples doesn't match SamplesPerPixel. Defining non-color channels as
ExtraSamples.
    \end{Verbatim}

    \hypertarget{analyse-fruxe9quentielle}{%
\subsection{Analyse fréquentielle}\label{analyse-fruxe9quentielle}}

L'analyse fréquentielle, issue du traitement du signal, permet d'avoir
un autre point de vue sur les données à partir de ses composantes
harmoniques. La modifications de ces composantes de Fourier modifie
l'ensemble de l'image et permet de corriger des problèmes systématiques
comme des artefacts ou du bruit de capteur. Bien que ce domaine soit un
peu éloigné de la télédétection, les images fourniment par les capteurs
sont tous sujets à des étapes de traitement du signal et il faut donc en
connaître les grands principes afin de pouvoir comprendre certains
enjeux lors des traitements.

\hypertarget{la-transformuxe9e-de-fourier}{%
\subsubsection{La transformée de
Fourier}\label{la-transformuxe9e-de-fourier}}

La transformée de Fourier permet de transformer une image dans un espace
fréquentielle. Cette transformée est complètement reversible. Dans le
cas des images numériques, on parle de \texttt{2D-DFT}
(\emph{2D-Discrete Fourier Transform}) qui est un algorithme optimisé
pour le calcul fréquentiel {[}@Cooley-1965{]}. La \emph{1D-DFT} peu
s'écrire simplement comme une projection sur une série d'exponentielles
complexes:

\[X[k] = \sum_{n=0 \ldots N-1} x[n] \times \exp(-j \times 2\pi \times k \times n/N))\]
\{\#eq-dft\}

La transformée inverse prend une forme similaire:

\[x[k] = \frac{1}{N}\sum_{n=0 \ldots N-1} X[n] \times \exp(j \times 2\pi \times k \times n/N))\]
\{\#eq-idft\}

Le signal d'origine est donc reconstruit à partir d'une somme de
sinusoïde complexe \(\exp(j2\pi \frac{k}{N}n))\) de fréquence \(k/N\).
Noter qu'à partir de \(k=N/2\), les sinusoïdes se répètent à un signe
près et forme un miroir des composantes, la convention est lors de
mettre ces composantes dans une espace négatif \([-N/2,\ldots,-1]\).

Dans le cas d'un simple signal périodique à une dimension avec une
fréquence de 4/16 (donc 4 périodes sur 16) on obtient deux pics de
fréquence à la position de 4 cycles observés sur \(N=16\) observations.
Les puissances de Fourier sont affichés dans un espace fréquentiel en
cycles par unité d'espacement de l'échantillon (avec zéro au début)
variant entre -1 et +1. Par exemple, si l'espacement des échantillons
est en secondes, l'unité de fréquence est cycles/seconde (ou Hz). Dans
le cas de N échantillons, le pic sera observé à la fréquence
\(+/- 4/16=0.25\) cycles/secondes. La fréquence d'échantillonnage
\(F_s\) du signal a aussi beaucoup d'importance aussi et doit être au
moins a deux fois la plus haute fréquence observée (ici \(F_s > 0.5\))
sinon un phénomène de repliement appelé aliasing sera observé.

    \begin{tcolorbox}[breakable, size=fbox, boxrule=1pt, pad at break*=1mm,colback=cellbackground, colframe=cellborder]
\prompt{In}{incolor}{5}{\boxspacing}
\begin{Verbatim}[commandchars=\\\{\}]
\PY{c+c1}{\PYZsh{}| eval: true}

\PY{k+kn}{import} \PY{n+nn}{math}
\PY{n}{Fs}\PY{o}{=} \PY{l+m+mf}{2.0}
\PY{n}{Ts}\PY{o}{=} \PY{l+m+mi}{1}\PY{o}{/}\PY{n}{Fs}
\PY{n}{N}\PY{o}{=} \PY{l+m+mi}{16}
\PY{n}{arr} \PY{o}{=} \PY{n}{xr}\PY{o}{.}\PY{n}{DataArray}\PY{p}{(}\PY{n}{np}\PY{o}{.}\PY{n}{sin}\PY{p}{(}\PY{l+m+mi}{2}\PY{o}{*}\PY{n}{math}\PY{o}{.}\PY{n}{pi}\PY{o}{*}\PY{n}{np}\PY{o}{.}\PY{n}{arange}\PY{p}{(}\PY{l+m+mi}{0}\PY{p}{,}\PY{n}{N}\PY{p}{,}\PY{n}{Ts}\PY{p}{)}\PY{o}{*}\PY{l+m+mi}{4}\PY{o}{/}\PY{l+m+mi}{16}\PY{p}{)}\PY{p}{,}
                   \PY{n}{dims}\PY{o}{=}\PY{p}{(}\PY{l+s+s1}{\PYZsq{}}\PY{l+s+s1}{x}\PY{l+s+s1}{\PYZsq{}}\PY{p}{)}\PY{p}{,} \PY{n}{coords}\PY{o}{=}\PY{p}{\PYZob{}}\PY{l+s+s1}{\PYZsq{}}\PY{l+s+s1}{x}\PY{l+s+s1}{\PYZsq{}}\PY{p}{:} \PY{n}{np}\PY{o}{.}\PY{n}{arange}\PY{p}{(}\PY{l+m+mi}{0}\PY{p}{,}\PY{n}{N}\PY{p}{,}\PY{n}{Ts}\PY{p}{)}\PY{p}{\PYZcb{}}\PY{p}{)}
\PY{n}{fourier} \PY{o}{=} \PY{n}{np}\PY{o}{.}\PY{n}{fft}\PY{o}{.}\PY{n}{fft}\PY{p}{(}\PY{n}{arr}\PY{p}{)}
\PY{n}{freq} \PY{o}{=} \PY{n}{np}\PY{o}{.}\PY{n}{fft}\PY{o}{.}\PY{n}{fftfreq}\PY{p}{(}\PY{n}{fourier}\PY{o}{.}\PY{n}{size}\PY{p}{,} \PY{n}{d}\PY{o}{=}\PY{n}{Ts}\PY{p}{)}
\PY{n}{fourier} \PY{o}{=} \PY{n}{xr}\PY{o}{.}\PY{n}{DataArray}\PY{p}{(}\PY{n}{fourier}\PY{p}{,}
                   \PY{n}{dims}\PY{o}{=}\PY{p}{(}\PY{l+s+s1}{\PYZsq{}}\PY{l+s+s1}{f}\PY{l+s+s1}{\PYZsq{}}\PY{p}{)}\PY{p}{,} \PY{n}{coords}\PY{o}{=}\PY{p}{\PYZob{}}\PY{l+s+s1}{\PYZsq{}}\PY{l+s+s1}{f}\PY{l+s+s1}{\PYZsq{}}\PY{p}{:} \PY{n}{freq}\PY{p}{\PYZcb{}}\PY{p}{)}

\PY{n}{fig}\PY{p}{,} \PY{n}{axes} \PY{o}{=} \PY{n}{plt}\PY{o}{.}\PY{n}{subplots}\PY{p}{(}\PY{n}{nrows}\PY{o}{=}\PY{l+m+mi}{1}\PY{p}{,} \PY{n}{ncols}\PY{o}{=}\PY{l+m+mi}{2}\PY{p}{,} \PY{n}{figsize}\PY{o}{=}\PY{p}{(}\PY{l+m+mi}{10}\PY{p}{,} \PY{l+m+mi}{4}\PY{p}{)}\PY{p}{)}
\PY{n}{plt}\PY{o}{.}\PY{n}{subplot}\PY{p}{(}\PY{l+m+mi}{1}\PY{p}{,} \PY{l+m+mi}{2}\PY{p}{,} \PY{l+m+mi}{1}\PY{p}{)}
\PY{n}{arr}\PY{o}{.}\PY{n}{plot}\PY{o}{.}\PY{n}{line}\PY{p}{(}\PY{n}{color}\PY{o}{=}\PY{l+s+s1}{\PYZsq{}}\PY{l+s+s1}{red}\PY{l+s+s1}{\PYZsq{}}\PY{p}{,} \PY{n}{linestyle}\PY{o}{=}\PY{l+s+s1}{\PYZsq{}}\PY{l+s+s1}{dashed}\PY{l+s+s1}{\PYZsq{}}\PY{p}{,} \PY{n}{marker}\PY{o}{=}\PY{l+s+s1}{\PYZsq{}}\PY{l+s+s1}{o}\PY{l+s+s1}{\PYZsq{}}\PY{p}{,} \PY{n}{markerfacecolor}\PY{o}{=}\PY{l+s+s1}{\PYZsq{}}\PY{l+s+s1}{blue}\PY{l+s+s1}{\PYZsq{}}\PY{p}{)}
\PY{n}{axes}\PY{p}{[}\PY{l+m+mi}{0}\PY{p}{]}\PY{o}{.}\PY{n}{set\PYZus{}title}\PY{p}{(}\PY{l+s+s2}{\PYZdq{}}\PY{l+s+s2}{Signal périodique}\PY{l+s+s2}{\PYZdq{}}\PY{p}{)}
\PY{n}{plt}\PY{o}{.}\PY{n}{subplot}\PY{p}{(}\PY{l+m+mi}{1}\PY{p}{,} \PY{l+m+mi}{2}\PY{p}{,} \PY{l+m+mi}{2}\PY{p}{)}
\PY{n}{np}\PY{o}{.}\PY{n}{abs}\PY{p}{(}\PY{n}{fourier}\PY{p}{)}\PY{o}{.}\PY{n}{plot}\PY{o}{.}\PY{n}{line}\PY{p}{(}\PY{n}{color}\PY{o}{=}\PY{l+s+s1}{\PYZsq{}}\PY{l+s+s1}{red}\PY{l+s+s1}{\PYZsq{}}\PY{p}{,} \PY{n}{linestyle}\PY{o}{=}\PY{l+s+s1}{\PYZsq{}}\PY{l+s+s1}{dashed}\PY{l+s+s1}{\PYZsq{}}\PY{p}{,} \PY{n}{marker}\PY{o}{=}\PY{l+s+s1}{\PYZsq{}}\PY{l+s+s1}{o}\PY{l+s+s1}{\PYZsq{}}\PY{p}{,} \PY{n}{markerfacecolor}\PY{o}{=}\PY{l+s+s1}{\PYZsq{}}\PY{l+s+s1}{blue}\PY{l+s+s1}{\PYZsq{}}\PY{p}{)}
\PY{n}{axes}\PY{p}{[}\PY{l+m+mi}{1}\PY{p}{]}\PY{o}{.}\PY{n}{set\PYZus{}title}\PY{p}{(}\PY{l+s+s2}{\PYZdq{}}\PY{l+s+s2}{Composantes de Fourier (amplitude)}\PY{l+s+s2}{\PYZdq{}}\PY{p}{)}
\PY{n}{plt}\PY{o}{.}\PY{n}{show}\PY{p}{(}\PY{p}{)}
\end{Verbatim}
\end{tcolorbox}

    \begin{center}
    \adjustimage{max size={0.9\linewidth}{0.9\paperheight}}{04-TransformationSpatiales_files/04-TransformationSpatiales_9_0.png}
    \end{center}
    { \hspace*{\fill} \\}
    
    \hypertarget{filtrage-fruxe9quentielle}{%
\subsubsection{Filtrage fréquentielle}\label{filtrage-fruxe9quentielle}}

Un filtrage fréquentielle consiste à modifier le spectre de Fourier afin
d'éliminer ou de réduire certaines composantes fréquentielles. On peut
distinguer trois grandes catégories de filtres fréquentielles:

\begin{enumerate}
\def\labelenumi{\arabic{enumi}.}
\item
  Les filtres passe-bas qui ne préservent que les basses fréquences
  pour, par exemple, lisser une image.
\item
  Les filtres passe-haut qui ne préservent que les hautes fréquences
  pour ne préserver que les détails.
\item
  Les filtres passe-bandes qui vont préserver les fréquences dans une
  bandes particulières.
\end{enumerate}

La librairie Scipy contient différents filtres fréquentielles. Notez,
qu'un filtrage fréquentielle est une simple multiplication de la réponse
du filtre \(F[k]\) par les composantes fréquentielles du signal à
filtrer \(X[k]\):

\[
X_f[k] = F[k] \times X[k]
\] \{\#eq-fourier-filter\}

À noter que cette multiplication dans l'espace de Fourier est
équivalente à une opération de convolution dans l'espace originale du
signal \(x\):

\[
x_f = IDFT^{-1}[F]*x
\] \{\#eq-convolve\}

    \begin{tcolorbox}[breakable, size=fbox, boxrule=1pt, pad at break*=1mm,colback=cellbackground, colframe=cellborder]
\prompt{In}{incolor}{6}{\boxspacing}
\begin{Verbatim}[commandchars=\\\{\}]
\PY{c+c1}{\PYZsh{}| eval: true}

\PY{k+kn}{from} \PY{n+nn}{scipy} \PY{k+kn}{import} \PY{n}{ndimage}
\PY{k+kn}{import} \PY{n+nn}{numpy}\PY{n+nn}{.}\PY{n+nn}{fft}

\PY{n}{fig}\PY{p}{,} \PY{p}{(}\PY{n}{ax1}\PY{p}{,} \PY{n}{ax2}\PY{p}{)} \PY{o}{=} \PY{n}{plt}\PY{o}{.}\PY{n}{subplots}\PY{p}{(}\PY{l+m+mi}{1}\PY{p}{,} \PY{l+m+mi}{2}\PY{p}{,} \PY{n}{figsize}\PY{o}{=}\PY{p}{(}\PY{l+m+mi}{10}\PY{p}{,} \PY{l+m+mi}{4}\PY{p}{)}\PY{p}{)}
\PY{n}{input\PYZus{}} \PY{o}{=} \PY{n}{numpy}\PY{o}{.}\PY{n}{fft}\PY{o}{.}\PY{n}{fft2}\PY{p}{(}\PY{n}{img\PYZus{}rgb}\PY{o}{.}\PY{n}{to\PYZus{}numpy}\PY{p}{(}\PY{p}{)}\PY{p}{)} 
\PY{n}{result} \PY{o}{=} \PY{p}{[}\PY{n}{ndimage}\PY{o}{.}\PY{n}{fourier\PYZus{}gaussian}\PY{p}{(}\PY{n}{input\PYZus{}}\PY{p}{[}\PY{n}{b}\PY{p}{]}\PY{p}{,} \PY{n}{sigma}\PY{o}{=}\PY{l+m+mi}{4}\PY{p}{)} \PY{k}{for} \PY{n}{b} \PY{o+ow}{in} \PY{n+nb}{range}\PY{p}{(}\PY{l+m+mi}{3}\PY{p}{)}\PY{p}{]} \PY{c+c1}{\PYZsh{} on filtre chaque bande avec un filtre Gaussien}
\PY{n}{result} \PY{o}{=} \PY{n}{numpy}\PY{o}{.}\PY{n}{fft}\PY{o}{.}\PY{n}{ifft2}\PY{p}{(}\PY{n}{result}\PY{p}{)}
\PY{n}{ax1}\PY{o}{.}\PY{n}{imshow}\PY{p}{(}\PY{n}{img\PYZus{}rgb}\PY{o}{.}\PY{n}{to\PYZus{}numpy}\PY{p}{(}\PY{p}{)}\PY{o}{.}\PY{n}{transpose}\PY{p}{(}\PY{l+m+mi}{1}\PY{p}{,} \PY{l+m+mi}{2}\PY{p}{,} \PY{l+m+mi}{0}\PY{p}{)}\PY{o}{.}\PY{n}{astype}\PY{p}{(}\PY{l+s+s1}{\PYZsq{}}\PY{l+s+s1}{uint8}\PY{l+s+s1}{\PYZsq{}}\PY{p}{)}\PY{p}{)}
\PY{n}{ax1}\PY{o}{.}\PY{n}{set\PYZus{}title}\PY{p}{(}\PY{l+s+s1}{\PYZsq{}}\PY{l+s+s1}{Originale}\PY{l+s+s1}{\PYZsq{}}\PY{p}{)}
\PY{n}{ax2}\PY{o}{.}\PY{n}{imshow}\PY{p}{(}\PY{n}{result}\PY{o}{.}\PY{n}{real}\PY{o}{.}\PY{n}{transpose}\PY{p}{(}\PY{l+m+mi}{1}\PY{p}{,} \PY{l+m+mi}{2}\PY{p}{,} \PY{l+m+mi}{0}\PY{p}{)}\PY{o}{.}\PY{n}{astype}\PY{p}{(}\PY{l+s+s1}{\PYZsq{}}\PY{l+s+s1}{uint8}\PY{l+s+s1}{\PYZsq{}}\PY{p}{)}\PY{p}{)}  \PY{c+c1}{\PYZsh{} La partie imaginaire n\PYZsq{}est pas utile ici}
\PY{n}{ax2}\PY{o}{.}\PY{n}{set\PYZus{}title}\PY{p}{(}\PY{l+s+s1}{\PYZsq{}}\PY{l+s+s1}{Filtrage Gaussien}\PY{l+s+s1}{\PYZsq{}}\PY{p}{)}
\PY{n}{plt}\PY{o}{.}\PY{n}{show}\PY{p}{(}\PY{p}{)}
\end{Verbatim}
\end{tcolorbox}

    \begin{Verbatim}[commandchars=\\\{\}]
/home/sfoucher/miniconda3/lib/python3.10/site-packages/rasterio/\_\_init\_\_.py:356:
NotGeoreferencedWarning: Dataset has no geotransform, gcps, or rpcs. The
identity matrix will be returned.
  dataset = DatasetReader(path, driver=driver, sharing=sharing, **kwargs)
    \end{Verbatim}

    \begin{center}
    \adjustimage{max size={0.9\linewidth}{0.9\paperheight}}{04-TransformationSpatiales_files/04-TransformationSpatiales_11_1.png}
    \end{center}
    { \hspace*{\fill} \\}
    
    \hypertarget{laliasing}{%
\subsubsection{L'aliasing}\label{laliasing}}

L'aliasing est un problème fréquent en traitement du signal. Il résulte
d'une fréquence d'échantillonnage trop faible par rapport au contenu
fréquentielle du signal. Ceci peut se produire lorsque vous
sous-échantillonner fortement une image avec un facteur de décimation
(par exemple 1 pixel sur 2). En prenant un pixel sur 2, on réduit la
fréquence d'échantillonnage d'un facteur 2 ce qui nous impose de réduire
le contenu fréquentielle de l'image et donc les fréquences maximales de
l'image. L'image présente alors un aspect faussement texturée avec
beaucoup de haute fréquences:

    \begin{tcolorbox}[breakable, size=fbox, boxrule=1pt, pad at break*=1mm,colback=cellbackground, colframe=cellborder]
\prompt{In}{incolor}{7}{\boxspacing}
\begin{Verbatim}[commandchars=\\\{\}]
\PY{c+c1}{\PYZsh{}| eval: true}
\PY{n}{fig}\PY{p}{,} \PY{n}{axes} \PY{o}{=} \PY{n}{plt}\PY{o}{.}\PY{n}{subplots}\PY{p}{(}\PY{n}{nrows}\PY{o}{=}\PY{l+m+mi}{1}\PY{p}{,} \PY{n}{ncols}\PY{o}{=}\PY{l+m+mi}{2}\PY{p}{,} \PY{n}{figsize}\PY{o}{=}\PY{p}{(}\PY{l+m+mi}{10}\PY{p}{,} \PY{l+m+mi}{4}\PY{p}{)}\PY{p}{)}
\PY{n}{plt}\PY{o}{.}\PY{n}{subplot}\PY{p}{(}\PY{l+m+mi}{1}\PY{p}{,} \PY{l+m+mi}{2}\PY{p}{,} \PY{l+m+mi}{1}\PY{p}{)}
\PY{n}{img\PYZus{}rgb}\PY{o}{.}\PY{n}{astype}\PY{p}{(}\PY{l+s+s1}{\PYZsq{}}\PY{l+s+s1}{int}\PY{l+s+s1}{\PYZsq{}}\PY{p}{)}\PY{o}{.}\PY{n}{plot}\PY{o}{.}\PY{n}{imshow}\PY{p}{(}\PY{n}{rgb}\PY{o}{=}\PY{l+s+s2}{\PYZdq{}}\PY{l+s+s2}{band}\PY{l+s+s2}{\PYZdq{}}\PY{p}{)}
\PY{n}{axes}\PY{p}{[}\PY{l+m+mi}{0}\PY{p}{]}\PY{o}{.}\PY{n}{set\PYZus{}title}\PY{p}{(}\PY{l+s+s2}{\PYZdq{}}\PY{l+s+s2}{Originale}\PY{l+s+s2}{\PYZdq{}}\PY{p}{)}
\PY{n}{plt}\PY{o}{.}\PY{n}{subplot}\PY{p}{(}\PY{l+m+mi}{1}\PY{p}{,} \PY{l+m+mi}{2}\PY{p}{,} \PY{l+m+mi}{2}\PY{p}{)}
\PY{n}{img\PYZus{}rgb}\PY{p}{[}\PY{p}{:}\PY{p}{,}\PY{p}{:}\PY{p}{:}\PY{l+m+mi}{4}\PY{p}{,}\PY{p}{:}\PY{p}{:}\PY{l+m+mi}{4}\PY{p}{]}\PY{o}{.}\PY{n}{astype}\PY{p}{(}\PY{l+s+s1}{\PYZsq{}}\PY{l+s+s1}{int}\PY{l+s+s1}{\PYZsq{}}\PY{p}{)}\PY{o}{.}\PY{n}{plot}\PY{o}{.}\PY{n}{imshow}\PY{p}{(}\PY{n}{rgb}\PY{o}{=}\PY{l+s+s2}{\PYZdq{}}\PY{l+s+s2}{band}\PY{l+s+s2}{\PYZdq{}}\PY{p}{)}
\PY{n}{axes}\PY{p}{[}\PY{l+m+mi}{1}\PY{p}{]}\PY{o}{.}\PY{n}{set\PYZus{}title}\PY{p}{(}\PY{l+s+s2}{\PYZdq{}}\PY{l+s+s2}{Décimée par un facteur 4}\PY{l+s+s2}{\PYZdq{}}\PY{p}{)}
\PY{n}{plt}\PY{o}{.}\PY{n}{show}\PY{p}{(}\PY{p}{)}
\end{Verbatim}
\end{tcolorbox}

    \begin{center}
    \adjustimage{max size={0.9\linewidth}{0.9\paperheight}}{04-TransformationSpatiales_files/04-TransformationSpatiales_13_0.png}
    \end{center}
    { \hspace*{\fill} \\}
    
    Une façon de réduire le contenu fréquentiel est de filtrer par un filtre
passe-bas pour réduire les hautes fréquences par exemple avec un filtre
Gaussien:

    \begin{tcolorbox}[breakable, size=fbox, boxrule=1pt, pad at break*=1mm,colback=cellbackground, colframe=cellborder]
\prompt{In}{incolor}{8}{\boxspacing}
\begin{Verbatim}[commandchars=\\\{\}]
\PY{c+c1}{\PYZsh{}| eval: true}
\PY{k+kn}{from} \PY{n+nn}{scipy}\PY{n+nn}{.}\PY{n+nn}{ndimage} \PY{k+kn}{import} \PY{n}{gaussian\PYZus{}filter}

\PY{n}{q}\PY{o}{=} \PY{l+m+mi}{4}
\PY{n}{sigma}\PY{o}{=} \PY{n}{q}\PY{o}{*}\PY{l+m+mf}{1.1774}\PY{o}{/}\PY{n}{math}\PY{o}{.}\PY{n}{pi}
\PY{n}{arr} \PY{o}{=} \PY{n}{xr}\PY{o}{.}\PY{n}{DataArray}\PY{p}{(}\PY{n}{gaussian\PYZus{}filter}\PY{p}{(}\PY{n}{img\PYZus{}rgb}\PY{o}{.}\PY{n}{to\PYZus{}numpy}\PY{p}{(}\PY{p}{)}\PY{p}{,} \PY{n}{sigma}\PY{o}{=} \PY{p}{(}\PY{l+m+mi}{0}\PY{p}{,}\PY{n}{sigma}\PY{p}{,}\PY{n}{sigma}\PY{p}{)}\PY{p}{)}\PY{p}{,} \PY{n}{dims}\PY{o}{=}\PY{p}{(}\PY{l+s+s1}{\PYZsq{}}\PY{l+s+s1}{band}\PY{l+s+s1}{\PYZsq{}}\PY{p}{,}\PY{l+s+s2}{\PYZdq{}}\PY{l+s+s2}{y}\PY{l+s+s2}{\PYZdq{}}\PY{p}{,} \PY{l+s+s2}{\PYZdq{}}\PY{l+s+s2}{x}\PY{l+s+s2}{\PYZdq{}}\PY{p}{)}\PY{p}{,} \PY{n}{coords}\PY{o}{=} \PY{p}{\PYZob{}}\PY{l+s+s1}{\PYZsq{}}\PY{l+s+s1}{x}\PY{l+s+s1}{\PYZsq{}}\PY{p}{:} \PY{n}{img\PYZus{}rgb}\PY{o}{.}\PY{n}{coords}\PY{p}{[}\PY{l+s+s1}{\PYZsq{}}\PY{l+s+s1}{x}\PY{l+s+s1}{\PYZsq{}}\PY{p}{]}\PY{p}{,} \PY{l+s+s1}{\PYZsq{}}\PY{l+s+s1}{y}\PY{l+s+s1}{\PYZsq{}}\PY{p}{:} \PY{n}{img\PYZus{}rgb}\PY{o}{.}\PY{n}{coords}\PY{p}{[}\PY{l+s+s1}{\PYZsq{}}\PY{l+s+s1}{y}\PY{l+s+s1}{\PYZsq{}}\PY{p}{]}\PY{p}{,} \PY{l+s+s1}{\PYZsq{}}\PY{l+s+s1}{spatial\PYZus{}ref}\PY{l+s+s1}{\PYZsq{}}\PY{p}{:} \PY{l+m+mi}{0}\PY{p}{\PYZcb{}}\PY{p}{)}

\PY{n}{fig}\PY{p}{,} \PY{n}{axes} \PY{o}{=} \PY{n}{plt}\PY{o}{.}\PY{n}{subplots}\PY{p}{(}\PY{n}{nrows}\PY{o}{=}\PY{l+m+mi}{1}\PY{p}{,} \PY{n}{ncols}\PY{o}{=}\PY{l+m+mi}{2}\PY{p}{,} \PY{n}{figsize}\PY{o}{=}\PY{p}{(}\PY{l+m+mi}{10}\PY{p}{,} \PY{l+m+mi}{4}\PY{p}{)}\PY{p}{)}
\PY{n}{plt}\PY{o}{.}\PY{n}{subplot}\PY{p}{(}\PY{l+m+mi}{1}\PY{p}{,} \PY{l+m+mi}{2}\PY{p}{,} \PY{l+m+mi}{1}\PY{p}{)}
\PY{n}{img\PYZus{}rgb}\PY{o}{.}\PY{n}{astype}\PY{p}{(}\PY{l+s+s1}{\PYZsq{}}\PY{l+s+s1}{int}\PY{l+s+s1}{\PYZsq{}}\PY{p}{)}\PY{o}{.}\PY{n}{plot}\PY{o}{.}\PY{n}{imshow}\PY{p}{(}\PY{n}{rgb}\PY{o}{=}\PY{l+s+s2}{\PYZdq{}}\PY{l+s+s2}{band}\PY{l+s+s2}{\PYZdq{}}\PY{p}{)}
\PY{n}{axes}\PY{p}{[}\PY{l+m+mi}{0}\PY{p}{]}\PY{o}{.}\PY{n}{set\PYZus{}title}\PY{p}{(}\PY{l+s+s2}{\PYZdq{}}\PY{l+s+s2}{Originale}\PY{l+s+s2}{\PYZdq{}}\PY{p}{)}
\PY{n}{plt}\PY{o}{.}\PY{n}{subplot}\PY{p}{(}\PY{l+m+mi}{1}\PY{p}{,} \PY{l+m+mi}{2}\PY{p}{,} \PY{l+m+mi}{2}\PY{p}{)}
\PY{n}{arr}\PY{p}{[}\PY{p}{:}\PY{p}{,}\PY{p}{:}\PY{p}{:}\PY{n}{q}\PY{p}{,}\PY{p}{:}\PY{p}{:}\PY{n}{q}\PY{p}{]}\PY{o}{.}\PY{n}{astype}\PY{p}{(}\PY{l+s+s1}{\PYZsq{}}\PY{l+s+s1}{int}\PY{l+s+s1}{\PYZsq{}}\PY{p}{)}\PY{o}{.}\PY{n}{plot}\PY{o}{.}\PY{n}{imshow}\PY{p}{(}\PY{n}{rgb}\PY{o}{=}\PY{l+s+s2}{\PYZdq{}}\PY{l+s+s2}{band}\PY{l+s+s2}{\PYZdq{}}\PY{p}{)}
\PY{n}{axes}\PY{p}{[}\PY{l+m+mi}{1}\PY{p}{]}\PY{o}{.}\PY{n}{set\PYZus{}title}\PY{p}{(}\PY{l+s+s2}{\PYZdq{}}\PY{l+s+s2}{Décimée par un facteur 4}\PY{l+s+s2}{\PYZdq{}}\PY{p}{)}
\PY{n}{plt}\PY{o}{.}\PY{n}{show}\PY{p}{(}\PY{p}{)}
\end{Verbatim}
\end{tcolorbox}

    \begin{center}
    \adjustimage{max size={0.9\linewidth}{0.9\paperheight}}{04-TransformationSpatiales_files/04-TransformationSpatiales_15_0.png}
    \end{center}
    { \hspace*{\fill} \\}
    
    \begin{tcolorbox}[breakable, size=fbox, boxrule=1pt, pad at break*=1mm,colback=cellbackground, colframe=cellborder]
\prompt{In}{incolor}{9}{\boxspacing}
\begin{Verbatim}[commandchars=\\\{\}]
\PY{c+c1}{\PYZsh{}| eval: true}
\PY{k+kn}{import} \PY{n+nn}{xrscipy}\PY{n+nn}{.}\PY{n+nn}{signal} \PY{k}{as} \PY{n+nn}{dsp}

\PY{n}{fig}\PY{p}{,} \PY{n}{axes} \PY{o}{=} \PY{n}{plt}\PY{o}{.}\PY{n}{subplots}\PY{p}{(}\PY{n}{nrows}\PY{o}{=}\PY{l+m+mi}{1}\PY{p}{,} \PY{n}{ncols}\PY{o}{=}\PY{l+m+mi}{2}\PY{p}{,} \PY{n}{figsize}\PY{o}{=}\PY{p}{(}\PY{l+m+mi}{10}\PY{p}{,} \PY{l+m+mi}{4}\PY{p}{)}\PY{p}{)}
\PY{n}{plt}\PY{o}{.}\PY{n}{subplot}\PY{p}{(}\PY{l+m+mi}{1}\PY{p}{,} \PY{l+m+mi}{2}\PY{p}{,} \PY{l+m+mi}{1}\PY{p}{)}
\PY{n}{img\PYZus{}rgb}\PY{o}{.}\PY{n}{astype}\PY{p}{(}\PY{l+s+s1}{\PYZsq{}}\PY{l+s+s1}{int}\PY{l+s+s1}{\PYZsq{}}\PY{p}{)}\PY{o}{.}\PY{n}{plot}\PY{o}{.}\PY{n}{imshow}\PY{p}{(}\PY{n}{rgb}\PY{o}{=}\PY{l+s+s2}{\PYZdq{}}\PY{l+s+s2}{band}\PY{l+s+s2}{\PYZdq{}}\PY{p}{)}
\PY{n}{axes}\PY{p}{[}\PY{l+m+mi}{0}\PY{p}{]}\PY{o}{.}\PY{n}{set\PYZus{}title}\PY{p}{(}\PY{l+s+s2}{\PYZdq{}}\PY{l+s+s2}{Originale}\PY{l+s+s2}{\PYZdq{}}\PY{p}{)}
\PY{n}{plt}\PY{o}{.}\PY{n}{subplot}\PY{p}{(}\PY{l+m+mi}{1}\PY{p}{,} \PY{l+m+mi}{2}\PY{p}{,} \PY{l+m+mi}{2}\PY{p}{)}
\PY{n}{dsp}\PY{o}{.}\PY{n}{decimate}\PY{p}{(}\PY{n}{img\PYZus{}rgb}\PY{p}{,} \PY{n}{q}\PY{o}{=}\PY{l+m+mi}{4}\PY{p}{,} \PY{n}{dim}\PY{o}{=}\PY{l+s+s1}{\PYZsq{}}\PY{l+s+s1}{x}\PY{l+s+s1}{\PYZsq{}}\PY{p}{)}\PY{o}{.}\PY{n}{astype}\PY{p}{(}\PY{l+s+s1}{\PYZsq{}}\PY{l+s+s1}{int}\PY{l+s+s1}{\PYZsq{}}\PY{p}{)}\PY{o}{.}\PY{n}{plot}\PY{o}{.}\PY{n}{imshow}\PY{p}{(}\PY{n}{rgb}\PY{o}{=}\PY{l+s+s2}{\PYZdq{}}\PY{l+s+s2}{band}\PY{l+s+s2}{\PYZdq{}}\PY{p}{)}
\PY{n}{axes}\PY{p}{[}\PY{l+m+mi}{1}\PY{p}{]}\PY{o}{.}\PY{n}{set\PYZus{}title}\PY{p}{(}\PY{l+s+s2}{\PYZdq{}}\PY{l+s+s2}{Décimée par un facteur 4}\PY{l+s+s2}{\PYZdq{}}\PY{p}{)}
\end{Verbatim}
\end{tcolorbox}

    \begin{Verbatim}[commandchars=\\\{\}]
Clipping input data to the valid range for imshow with RGB data ([0..1] for
floats or [0..255] for integers). Got range [3..265].
    \end{Verbatim}

            \begin{tcolorbox}[breakable, size=fbox, boxrule=.5pt, pad at break*=1mm, opacityfill=0]
\prompt{Out}{outcolor}{9}{\boxspacing}
\begin{Verbatim}[commandchars=\\\{\}]
Text(0.5, 1.0, 'Décimée par un facteur 4')
\end{Verbatim}
\end{tcolorbox}
        
    \begin{center}
    \adjustimage{max size={0.9\linewidth}{0.9\paperheight}}{04-TransformationSpatiales_files/04-TransformationSpatiales_16_2.png}
    \end{center}
    { \hspace*{\fill} \\}
    
    \hypertarget{filtrage-dimage}{%
\subsection{Filtrage d'image}\label{filtrage-dimage}}

Le filtrage d'image a plusieurs objectifs en télédétection:

\begin{enumerate}
\def\labelenumi{\arabic{enumi}.}
\item
  La réduction du bruit afin d'améliorer la résolution radiométrique et
  améliorer la lisibilité de l'image.
\item
  Le réhaussement de l'image afin d'améliorer le contraste ou faire
  ressortir les contours.
\item
  La production de nouvelles caractéristiques: c.à.d dériver de
  nouvelles images mettant en valeur certaines informations dans l'image
  comme la texture, les contours, etc.
\end{enumerate}

Il existe de nombreuses méthodes de filtrage dans la littérature, on
peut rassembler ces filtres en quatre grandes catégories:

\begin{enumerate}
\def\labelenumi{\arabic{enumi}.}
\item
  Le filtrage peut-être global ou local, c.à.d prendre en compte toute
  l'image pour filtrer (ex: filtrage par Fourier) ou seulement
  localement avec une fenêtre ou un voisinage local.
\item
  La fonction de filtrage peut-être linéaire ou non linéaire.
\item
  La fonction de filtrage peut être stationnaire ou adaptative
\item
  Le filtrage peut-être mono-échelle ou multi-échelles
\end{enumerate}

La librairie Scipy
(\href{https://docs.scipy.org/doc/scipy/reference/ndimage.html}{Multidimensional
image processing (scipy.ndimage)}) contient une panoplie complète de
filtres.

\hypertarget{filtrage-linuxe9aire-stationnaire}{%
\subsubsection{Filtrage linéaire
stationnaire}\label{filtrage-linuxe9aire-stationnaire}}

Un filtrage linéaire stationnaire consiste à appliquer une même
pondération locale des valeurs des pixels dans une fenêtre glissante. La
taille de cette fenêtre est généralement impaire (3,5, etc.) afin de
définir une position centrale et une fenêtre symétrique.

Mettre une figure ici

Le filtre le plus simple est certainement le filtre moyen qui consiste à
appliquer le même poids uniforme dans la fenêtre glissante.

\[
F= \frac{1}{25}\left[
\begin{array}{c|c|c|c|c}
1 & 1 & 1 & 1 & 1 \\
\hline
1 & 1 & 1 & 1 & 1 \\
\hline
1 & 1 & 1 & 1 & 1 \\
\hline
1 & 1 & 1 & 1 & 1 \\
\hline
1 & 1 & 1 & 1 & 1
\end{array}
\right]
\]\{\#eq-boxfilter\}

En python, on dispose des fonctions \texttt{rolling} et
\texttt{sliding\_window} définis dans la librairie numpy. Par exemple
pour le cas du filtre moyen on peut construire une nouvelle vue de
l'image avec deux nouvelles dimensions \texttt{x\_win} et
\texttt{y\_win}:

    \begin{tcolorbox}[breakable, size=fbox, boxrule=1pt, pad at break*=1mm,colback=cellbackground, colframe=cellborder]
\prompt{In}{incolor}{10}{\boxspacing}
\begin{Verbatim}[commandchars=\\\{\}]
\PY{c+c1}{\PYZsh{}| eval: true}
\PY{k+kn}{import} \PY{n+nn}{rioxarray} \PY{k}{as} \PY{n+nn}{rxr}
\PY{n}{rolling\PYZus{}win} \PY{o}{=} \PY{n}{img\PYZus{}rgb}\PY{o}{.}\PY{n}{rolling}\PY{p}{(}\PY{n}{x}\PY{o}{=}\PY{l+m+mi}{5}\PY{p}{,} \PY{n}{y}\PY{o}{=}\PY{l+m+mi}{5}\PY{p}{,}  \PY{n}{min\PYZus{}periods}\PY{o}{=} \PY{l+m+mi}{3}\PY{p}{,} \PY{n}{center}\PY{o}{=} \PY{k+kc}{True}\PY{p}{)}\PY{o}{.}\PY{n}{construct}\PY{p}{(}\PY{n}{x}\PY{o}{=}\PY{l+s+s2}{\PYZdq{}}\PY{l+s+s2}{x\PYZus{}win}\PY{l+s+s2}{\PYZdq{}}\PY{p}{,} \PY{n}{y}\PY{o}{=}\PY{l+s+s2}{\PYZdq{}}\PY{l+s+s2}{y\PYZus{}win}\PY{l+s+s2}{\PYZdq{}}\PY{p}{,} \PY{n}{keep\PYZus{}attrs}\PY{o}{=} \PY{k+kc}{True}\PY{p}{)}
\PY{n+nb}{print}\PY{p}{(}\PY{n}{rolling\PYZus{}win}\PY{p}{[}\PY{l+m+mi}{0}\PY{p}{,}\PY{l+m+mi}{0}\PY{p}{,}\PY{l+m+mi}{1}\PY{p}{,}\PY{o}{.}\PY{o}{.}\PY{o}{.}\PY{p}{]}\PY{p}{)}
\PY{n+nb}{print}\PY{p}{(}\PY{n}{rolling\PYZus{}win}\PY{o}{.}\PY{n}{shape}\PY{p}{)}
\end{Verbatim}
\end{tcolorbox}

    \begin{Verbatim}[commandchars=\\\{\}]
<xarray.DataArray (x\_win: 5, y\_win: 5)> Size: 100B
array([[ nan,  nan,  nan,  nan,  nan],
       [ nan,  nan, 209., 210., 209.],
       [ nan,  nan, 213., 214., 212.],
       [ nan,  nan, 213., 212., 210.],
       [ nan,  nan, 210., 209., 206.]], dtype=float32)
Coordinates:
    band         int64 8B 1
    x            float64 8B 1.5
    y            float64 8B 0.5
    spatial\_ref  int64 8B 0
Dimensions without coordinates: x\_win, y\_win
(3, 771, 1311, 5, 5)
    \end{Verbatim}

    L'avantage de cette approche est qu'il n'y a pas d'utilisation inutile
de la mémoire. Noter les \texttt{nan} sur les bords de l'image car la
fenêtre déborde sur les bordures de l'image. Par la suite un opérateur
moyenne peut être appliqué.

    \begin{tcolorbox}[breakable, size=fbox, boxrule=1pt, pad at break*=1mm,colback=cellbackground, colframe=cellborder]
\prompt{In}{incolor}{11}{\boxspacing}
\begin{Verbatim}[commandchars=\\\{\}]
\PY{c+c1}{\PYZsh{}| eval: true}
\PY{n}{filtre\PYZus{}moyen}\PY{o}{=} \PY{n}{rolling\PYZus{}win}\PY{o}{.}\PY{n}{mean}\PY{p}{(}\PY{n}{dim}\PY{o}{=} \PY{p}{[}\PY{l+s+s1}{\PYZsq{}}\PY{l+s+s1}{x\PYZus{}win}\PY{l+s+s1}{\PYZsq{}}\PY{p}{,} \PY{l+s+s1}{\PYZsq{}}\PY{l+s+s1}{y\PYZus{}win}\PY{l+s+s1}{\PYZsq{}}\PY{p}{]}\PY{p}{,} \PY{n}{skipna}\PY{o}{=} \PY{k+kc}{True}\PY{p}{)}
\PY{n}{fig}\PY{p}{,} \PY{n}{ax} \PY{o}{=} \PY{n}{plt}\PY{o}{.}\PY{n}{subplots}\PY{p}{(}\PY{n}{nrows}\PY{o}{=}\PY{l+m+mi}{1}\PY{p}{,} \PY{n}{ncols}\PY{o}{=}\PY{l+m+mi}{1}\PY{p}{,} \PY{n}{figsize}\PY{o}{=}\PY{p}{(}\PY{l+m+mi}{8}\PY{p}{,} \PY{l+m+mi}{4}\PY{p}{)}\PY{p}{)}
\PY{n}{filtre\PYZus{}moyen}\PY{o}{.}\PY{n}{astype}\PY{p}{(}\PY{l+s+s1}{\PYZsq{}}\PY{l+s+s1}{int}\PY{l+s+s1}{\PYZsq{}}\PY{p}{)}\PY{o}{.}\PY{n}{plot}\PY{o}{.}\PY{n}{imshow}\PY{p}{(}\PY{n}{rgb}\PY{o}{=}\PY{l+s+s2}{\PYZdq{}}\PY{l+s+s2}{band}\PY{l+s+s2}{\PYZdq{}}\PY{p}{)}
\PY{n}{ax}\PY{o}{.}\PY{n}{set\PYZus{}title}\PY{p}{(}\PY{l+s+s2}{\PYZdq{}}\PY{l+s+s2}{Filtre moyen 5x5}\PY{l+s+s2}{\PYZdq{}}\PY{p}{)}
\end{Verbatim}
\end{tcolorbox}

            \begin{tcolorbox}[breakable, size=fbox, boxrule=.5pt, pad at break*=1mm, opacityfill=0]
\prompt{Out}{outcolor}{11}{\boxspacing}
\begin{Verbatim}[commandchars=\\\{\}]
Text(0.5, 1.0, 'Filtre moyen 5x5')
\end{Verbatim}
\end{tcolorbox}
        
    \begin{center}
    \adjustimage{max size={0.9\linewidth}{0.9\paperheight}}{04-TransformationSpatiales_files/04-TransformationSpatiales_20_1.png}
    \end{center}
    { \hspace*{\fill} \\}
    
    Filtre de Sobel, filtre Prewitt

\hypertarget{filtrage-par-convolution}{%
\paragraph{Filtrage par convolution}\label{filtrage-par-convolution}}

La façon la plus efficace d'appliquer un filtre linéaire est d'appliquer
une convolution. La convolution est généralement très efficace car elle
est peut être calculée dans le domaine fréquentielle. Prenons l'exemple
du filtre de Scharr {[}@Scharr1999{]}, ce filtre permet de détecter les
contours horizontaux et verticaux:

\[
F= \left[
\begin{array}{ccc}
-3-3j & 0-10j & +3-3j \\
-10+0j & 0+0j & +10+0j \\
-3+3j & 0+10j & +3+3j
\end{array}
\right]
\]\{\#eq-scharr-filter\}

Remarquez l'utilisation de chiffres complexes afin de passer deux
filtres différents sur la partie réelle et imaginaire.

    \begin{tcolorbox}[breakable, size=fbox, boxrule=1pt, pad at break*=1mm,colback=cellbackground, colframe=cellborder]
\prompt{In}{incolor}{12}{\boxspacing}
\begin{Verbatim}[commandchars=\\\{\}]
\PY{c+c1}{\PYZsh{}| eval: true}

\PY{n}{scharr} \PY{o}{=} \PY{n}{np}\PY{o}{.}\PY{n}{array}\PY{p}{(}\PY{p}{[}\PY{p}{[} \PY{o}{\PYZhy{}}\PY{l+m+mi}{3}\PY{o}{\PYZhy{}}\PY{l+m+mi}{3}\PY{n}{j}\PY{p}{,} \PY{l+m+mi}{0}\PY{o}{\PYZhy{}}\PY{l+m+mi}{10}\PY{n}{j}\PY{p}{,}  \PY{o}{+}\PY{l+m+mi}{3} \PY{o}{\PYZhy{}}\PY{l+m+mi}{3}\PY{n}{j}\PY{p}{]}\PY{p}{,}
                   \PY{p}{[}\PY{o}{\PYZhy{}}\PY{l+m+mi}{10}\PY{o}{+}\PY{l+m+mi}{0}\PY{n}{j}\PY{p}{,} \PY{l+m+mi}{0}\PY{o}{+} \PY{l+m+mi}{0}\PY{n}{j}\PY{p}{,} \PY{o}{+}\PY{l+m+mi}{10} \PY{o}{+}\PY{l+m+mi}{0}\PY{n}{j}\PY{p}{]}\PY{p}{,}
                   \PY{p}{[} \PY{o}{\PYZhy{}}\PY{l+m+mi}{3}\PY{o}{+}\PY{l+m+mi}{3}\PY{n}{j}\PY{p}{,} \PY{l+m+mi}{0}\PY{o}{+}\PY{l+m+mi}{10}\PY{n}{j}\PY{p}{,}  \PY{o}{+}\PY{l+m+mi}{3} \PY{o}{+}\PY{l+m+mi}{3}\PY{n}{j}\PY{p}{]}\PY{p}{]}\PY{p}{)} \PY{c+c1}{\PYZsh{} Gx + j*Gy}
\PY{n+nb}{print}\PY{p}{(}\PY{n}{img\PYZus{}rgb}\PY{o}{.}\PY{n}{isel}\PY{p}{(}\PY{n}{band}\PY{o}{=}\PY{l+m+mi}{0}\PY{p}{)}\PY{o}{.}\PY{n}{shape}\PY{p}{)}
\PY{n}{grad} \PY{o}{=} \PY{n}{signal}\PY{o}{.}\PY{n}{convolve2d}\PY{p}{(}\PY{n}{img\PYZus{}rgb}\PY{o}{.}\PY{n}{isel}\PY{p}{(}\PY{n}{band}\PY{o}{=}\PY{l+m+mi}{0}\PY{p}{)}\PY{p}{,} \PY{n}{scharr}\PY{p}{,} \PY{n}{boundary}\PY{o}{=}\PY{l+s+s1}{\PYZsq{}}\PY{l+s+s1}{symm}\PY{l+s+s1}{\PYZsq{}}\PY{p}{,} \PY{n}{mode}\PY{o}{=}\PY{l+s+s1}{\PYZsq{}}\PY{l+s+s1}{same}\PY{l+s+s1}{\PYZsq{}}\PY{p}{)}
\PY{c+c1}{\PYZsh{} on reconstruit un xarray à partir du résultat:}
\PY{n}{arr} \PY{o}{=} \PY{n}{xr}\PY{o}{.}\PY{n}{DataArray}\PY{p}{(}\PY{n}{np}\PY{o}{.}\PY{n}{abs}\PY{p}{(}\PY{n}{grad}\PY{p}{)}\PY{p}{,} \PY{n}{dims}\PY{o}{=}\PY{p}{(}\PY{l+s+s2}{\PYZdq{}}\PY{l+s+s2}{y}\PY{l+s+s2}{\PYZdq{}}\PY{p}{,} \PY{l+s+s2}{\PYZdq{}}\PY{l+s+s2}{x}\PY{l+s+s2}{\PYZdq{}}\PY{p}{)}\PY{p}{,} \PY{n}{coords}\PY{o}{=} \PY{p}{\PYZob{}}\PY{l+s+s1}{\PYZsq{}}\PY{l+s+s1}{x}\PY{l+s+s1}{\PYZsq{}}\PY{p}{:} \PY{n}{img\PYZus{}rgb}\PY{o}{.}\PY{n}{coords}\PY{p}{[}\PY{l+s+s1}{\PYZsq{}}\PY{l+s+s1}{x}\PY{l+s+s1}{\PYZsq{}}\PY{p}{]}\PY{p}{,} \PY{l+s+s1}{\PYZsq{}}\PY{l+s+s1}{y}\PY{l+s+s1}{\PYZsq{}}\PY{p}{:} \PY{n}{img\PYZus{}rgb}\PY{o}{.}\PY{n}{coords}\PY{p}{[}\PY{l+s+s1}{\PYZsq{}}\PY{l+s+s1}{y}\PY{l+s+s1}{\PYZsq{}}\PY{p}{]}\PY{p}{,} \PY{l+s+s1}{\PYZsq{}}\PY{l+s+s1}{spatial\PYZus{}ref}\PY{l+s+s1}{\PYZsq{}}\PY{p}{:} \PY{l+m+mi}{0}\PY{p}{\PYZcb{}}\PY{p}{)}
\PY{n+nb}{print}\PY{p}{(}\PY{n}{arr}\PY{p}{)}
\PY{n}{fig}\PY{p}{,} \PY{n}{ax} \PY{o}{=} \PY{n}{plt}\PY{o}{.}\PY{n}{subplots}\PY{p}{(}\PY{n}{nrows}\PY{o}{=}\PY{l+m+mi}{1}\PY{p}{,} \PY{n}{ncols}\PY{o}{=}\PY{l+m+mi}{1}\PY{p}{,} \PY{n}{figsize}\PY{o}{=}\PY{p}{(}\PY{l+m+mi}{8}\PY{p}{,} \PY{l+m+mi}{4}\PY{p}{)}\PY{p}{)}
\PY{n}{arr}\PY{o}{.}\PY{n}{plot}\PY{o}{.}\PY{n}{imshow}\PY{p}{(}\PY{p}{)}
\PY{n}{ax}\PY{o}{.}\PY{n}{set\PYZus{}title}\PY{p}{(}\PY{l+s+s2}{\PYZdq{}}\PY{l+s+s2}{Amplitude du filtre de Scharr}\PY{l+s+s2}{\PYZdq{}}\PY{p}{)}
\end{Verbatim}
\end{tcolorbox}

    \begin{Verbatim}[commandchars=\\\{\}]
(771, 1311)
<xarray.DataArray (y: 771, x: 1311)> Size: 8MB
array([[  65.96969001,   58.85575588,   54.91812087, {\ldots}, 1474.        ,
        1037.01205393,  389.99487176],
       [  61.07372594,   39.8246155 ,   89.18520057, {\ldots}, 1763.79647352,
         864.92543031,  270.20362692],
       [  98.48857802,  112.44554237,  168.10710871, {\ldots}, 2110.61365484,
         870.36658943,  204.40156555],
       {\ldots},
       [ 143.17821063,  597.00753764, 2479.42977315, {\ldots},  216.00925906,
         248.33847869,  200.89798406],
       [ 106.07544485,  393.67245268, 2188.78824924, {\ldots},  124.96399481,
         159.90622252,  346.34087255],
       [  41.59326869,  229.05894438, 1845.1216762 , {\ldots},  175.16278143,
          33.37663854,  414.3911196 ]], shape=(771, 1311))
Coordinates:
  * x            (x) float64 10kB 0.5 1.5 2.5 {\ldots} 1.308e+03 1.31e+03 1.31e+03
  * y            (y) float64 6kB 0.5 1.5 2.5 3.5 4.5 {\ldots} 767.5 768.5 769.5 770.5
    spatial\_ref  int64 8B 0
    \end{Verbatim}

            \begin{tcolorbox}[breakable, size=fbox, boxrule=.5pt, pad at break*=1mm, opacityfill=0]
\prompt{Out}{outcolor}{12}{\boxspacing}
\begin{Verbatim}[commandchars=\\\{\}]
Text(0.5, 1.0, 'Amplitude du filtre de Scharr')
\end{Verbatim}
\end{tcolorbox}
        
    \begin{center}
    \adjustimage{max size={0.9\linewidth}{0.9\paperheight}}{04-TransformationSpatiales_files/04-TransformationSpatiales_22_2.png}
    \end{center}
    { \hspace*{\fill} \\}
    
    \hypertarget{gestion-des-bordures}{%
\subparagraph{Gestion des bordures}\label{gestion-des-bordures}}

L'application de filtres à l'intérieur de fenêtres glissantes implique
de gérer les bords de l'image car la fenêtre de traitement va
nécessairement déborder de quelques pixels en dehors de l'image
(généralement la moitié de la fenêtre déborde). On peut soit décider
d'ignorer les valeurs en dehors de l'image en imposant une valeur
\texttt{nan}, prolonger l'image de quelques lignes et colonnes avec des
valeurs mirroirs ou constantes.

\hypertarget{filtrage-par-une-couche-convolutionnelle}{%
\paragraph{Filtrage par une couche
convolutionnelle}\label{filtrage-par-une-couche-convolutionnelle}}

Cette section nécessite la librairie Pytorch avec un GPU et ne
fonctionnera que sur Colab.

Une couche convolutionnelle est simplement un ensemble de filtres
appliqués sur la donnée d'entrée. Ce type de filtrage est à la base des
réseaux dits convolutionnels qui seront abordés dans le tome 2. On peut
ici imposer les mêmes filtres de gradient dans la couche
convolutionnelle:

    \begin{tcolorbox}[breakable, size=fbox, boxrule=1pt, pad at break*=1mm,colback=cellbackground, colframe=cellborder]
\prompt{In}{incolor}{13}{\boxspacing}
\begin{Verbatim}[commandchars=\\\{\}]
\PY{c+c1}{\PYZsh{}| eval: true}
\PY{k+kn}{import} \PY{n+nn}{torch}
\PY{k+kn}{import} \PY{n+nn}{torch}\PY{n+nn}{.}\PY{n+nn}{nn} \PY{k}{as} \PY{n+nn}{nn}
\PY{k+kn}{import} \PY{n+nn}{numpy} \PY{k}{as} \PY{n+nn}{np}
\PY{k+kn}{import} \PY{n+nn}{matplotlib}\PY{n+nn}{.}\PY{n+nn}{pyplot} \PY{k}{as} \PY{n+nn}{plt}
\PY{n}{normalized\PYZus{}img}\PY{o}{=} \PY{n}{torch}\PY{o}{.}\PY{n}{tensor}\PY{p}{(}\PY{n}{img\PYZus{}rgb}\PY{o}{.}\PY{n}{to\PYZus{}numpy}\PY{p}{(}\PY{p}{)}\PY{p}{)}
\PY{n}{nchannels}\PY{o}{=} \PY{n}{normalized\PYZus{}img}\PY{o}{.}\PY{n}{size}\PY{p}{(}\PY{p}{)}\PY{p}{[}\PY{l+m+mi}{0}\PY{p}{]} \PY{c+c1}{\PYZsh{} nombre de canaux de l\PYZsq{}image}

\PY{c+c1}{\PYZsh{} Define a conv2d layer}
\PY{n}{conv\PYZus{}layer} \PY{o}{=} \PY{n}{nn}\PY{o}{.}\PY{n}{Conv2d}\PY{p}{(}\PY{n}{in\PYZus{}channels}\PY{o}{=} \PY{n}{nchannels}\PY{p}{,} \PY{n}{out\PYZus{}channels}\PY{o}{=}\PY{l+m+mi}{2}\PY{p}{,} \PY{n}{kernel\PYZus{}size}\PY{o}{=}\PY{l+m+mi}{3}\PY{p}{,} \PY{n}{padding}\PY{o}{=}\PY{l+m+mi}{1}\PY{p}{,} \PY{n}{stride}\PY{o}{=}\PY{l+m+mi}{1}\PY{p}{,} \PY{n}{dilation}\PY{o}{=} \PY{l+m+mi}{1}\PY{p}{)}

\PY{c+c1}{\PYZsh{} Filtre de Sobel}
\PY{n}{sobel\PYZus{}x} \PY{o}{=} \PY{n}{np}\PY{o}{.}\PY{n}{array}\PY{p}{(}\PY{p}{[}\PY{p}{[}\PY{o}{\PYZhy{}}\PY{l+m+mi}{3}\PY{p}{,} \PY{l+m+mi}{0}\PY{p}{,} \PY{l+m+mi}{3}\PY{p}{]}\PY{p}{,} \PY{p}{[}\PY{o}{\PYZhy{}}\PY{l+m+mi}{10}\PY{p}{,} \PY{l+m+mi}{0}\PY{p}{,} \PY{l+m+mi}{10}\PY{p}{]}\PY{p}{,} \PY{p}{[}\PY{o}{\PYZhy{}}\PY{l+m+mi}{3}\PY{p}{,} \PY{l+m+mi}{0}\PY{p}{,} \PY{l+m+mi}{3}\PY{p}{]}\PY{p}{]}\PY{p}{)}
\PY{n}{sobel\PYZus{}y} \PY{o}{=} \PY{n}{np}\PY{o}{.}\PY{n}{array}\PY{p}{(}\PY{p}{[}\PY{p}{[}\PY{o}{\PYZhy{}}\PY{l+m+mi}{3}\PY{p}{,} \PY{o}{\PYZhy{}}\PY{l+m+mi}{10}\PY{p}{,} \PY{o}{\PYZhy{}}\PY{l+m+mi}{3}\PY{p}{]}\PY{p}{,} \PY{p}{[}\PY{l+m+mi}{0}\PY{p}{,} \PY{l+m+mi}{0}\PY{p}{,} \PY{l+m+mi}{0}\PY{p}{]}\PY{p}{,} \PY{p}{[}\PY{l+m+mi}{3}\PY{p}{,} \PY{l+m+mi}{10}\PY{p}{,} \PY{l+m+mi}{3}\PY{p}{]}\PY{p}{]}\PY{p}{)}

\PY{n}{kernel} \PY{o}{=} \PY{n}{np}\PY{o}{.}\PY{n}{stack}\PY{p}{(}\PY{p}{[}\PY{n}{sobel\PYZus{}x}\PY{p}{,} \PY{n}{sobel\PYZus{}y}\PY{p}{]}\PY{p}{)}
\PY{n}{kernel} \PY{o}{=} \PY{n}{kernel}\PY{o}{.}\PY{n}{reshape}\PY{p}{(}\PY{l+m+mi}{2}\PY{p}{,} \PY{l+m+mi}{1}\PY{p}{,} \PY{l+m+mi}{3}\PY{p}{,} \PY{l+m+mi}{3}\PY{p}{)}

\PY{n}{kernel} \PY{o}{=} \PY{n}{np}\PY{o}{.}\PY{n}{tile}\PY{p}{(}\PY{n}{kernel}\PY{p}{,}\PY{p}{(}\PY{l+m+mi}{1}\PY{p}{,}\PY{n}{nchannels}\PY{p}{,}\PY{l+m+mi}{1}\PY{p}{,}\PY{l+m+mi}{1}\PY{p}{)}\PY{p}{)}
\PY{n+nb}{print}\PY{p}{(}\PY{n}{kernel}\PY{o}{.}\PY{n}{shape}\PY{p}{)}
\PY{n}{kernel} \PY{o}{=} \PY{n}{torch}\PY{o}{.}\PY{n}{as\PYZus{}tensor}\PY{p}{(}\PY{n}{kernel}\PY{p}{,}\PY{n}{dtype}\PY{o}{=}\PY{n}{torch}\PY{o}{.}\PY{n}{float32}\PY{p}{)}
\PY{n}{conv\PYZus{}layer}\PY{o}{.}\PY{n}{weight} \PY{o}{=} \PY{n}{nn}\PY{o}{.}\PY{n}{Parameter}\PY{p}{(}\PY{n}{kernel}\PY{p}{)}
\PY{n}{conv\PYZus{}layer}\PY{o}{.}\PY{n}{bias} \PY{o}{=} \PY{n}{nn}\PY{o}{.}\PY{n}{Parameter}\PY{p}{(}\PY{n}{torch}\PY{o}{.}\PY{n}{zeros}\PY{p}{(}\PY{l+m+mi}{2}\PY{p}{,}\PY{p}{)}\PY{p}{)}

\PY{n+nb}{input}\PY{o}{=} \PY{n}{normalized\PYZus{}img}\PY{o}{.}\PY{n}{unsqueeze}\PY{p}{(}\PY{l+m+mi}{0}\PY{p}{)} \PY{c+c1}{\PYZsh{} il faut ajouter une dimension pour le nombre d\PYZsq{}échantillons}
\PY{n+nb}{print}\PY{p}{(}\PY{n+nb}{input}\PY{o}{.}\PY{n}{shape}\PY{p}{)}
\PY{c+c1}{\PYZsh{} Visualize the filters}
\PY{n}{fig}\PY{p}{,} \PY{n}{axs} \PY{o}{=} \PY{n}{plt}\PY{o}{.}\PY{n}{subplots}\PY{p}{(}\PY{l+m+mi}{1}\PY{p}{,} \PY{l+m+mi}{2}\PY{p}{,} \PY{n}{figsize}\PY{o}{=}\PY{p}{(}\PY{l+m+mi}{8}\PY{p}{,} \PY{l+m+mi}{5}\PY{p}{)}\PY{p}{)}
\PY{k}{for} \PY{n}{i} \PY{o+ow}{in} \PY{n+nb}{range}\PY{p}{(}\PY{l+m+mi}{2}\PY{p}{)}\PY{p}{:}
    \PY{n}{axs}\PY{p}{[}\PY{n}{i}\PY{p}{]}\PY{o}{.}\PY{n}{imshow}\PY{p}{(}\PY{n}{conv\PYZus{}layer}\PY{o}{.}\PY{n}{weight}\PY{o}{.}\PY{n}{data}\PY{o}{.}\PY{n}{numpy}\PY{p}{(}\PY{p}{)}\PY{p}{[}\PY{n}{i}\PY{p}{,} \PY{l+m+mi}{0}\PY{p}{]}\PY{p}{)}
    \PY{n}{axs}\PY{p}{[}\PY{n}{i}\PY{p}{]}\PY{o}{.}\PY{n}{set\PYZus{}title}\PY{p}{(}\PY{l+s+sa}{f}\PY{l+s+s1}{\PYZsq{}}\PY{l+s+s1}{Filtre }\PY{l+s+si}{\PYZob{}}\PY{n}{i}\PY{o}{+}\PY{l+m+mi}{1}\PY{l+s+si}{\PYZcb{}}\PY{l+s+s1}{\PYZsq{}}\PY{p}{)}
\PY{n}{plt}\PY{o}{.}\PY{n}{show}\PY{p}{(}\PY{p}{)}
\end{Verbatim}
\end{tcolorbox}

    \begin{Verbatim}[commandchars=\\\{\}]
(2, 3, 3, 3)
torch.Size([1, 3, 771, 1311])
    \end{Verbatim}

    \begin{center}
    \adjustimage{max size={0.9\linewidth}{0.9\paperheight}}{04-TransformationSpatiales_files/04-TransformationSpatiales_24_1.png}
    \end{center}
    { \hspace*{\fill} \\}
    
    Le résultat est alors calculé sur GPU (si disponible):

    \begin{tcolorbox}[breakable, size=fbox, boxrule=1pt, pad at break*=1mm,colback=cellbackground, colframe=cellborder]
\prompt{In}{incolor}{14}{\boxspacing}
\begin{Verbatim}[commandchars=\\\{\}]
\PY{c+c1}{\PYZsh{}| eval: true}
\PY{k+kn}{import} \PY{n+nn}{torch}
\PY{k+kn}{import} \PY{n+nn}{matplotlib}\PY{n+nn}{.}\PY{n+nn}{pyplot} \PY{k}{as} \PY{n+nn}{plt}

\PY{n}{output} \PY{o}{=} \PY{n}{conv\PYZus{}layer}\PY{p}{(}\PY{n+nb}{input}\PY{p}{)}
\PY{n+nb}{print}\PY{p}{(}\PY{l+s+sa}{f}\PY{l+s+s1}{\PYZsq{}}\PY{l+s+s1}{Image (BxCxHxW): }\PY{l+s+si}{\PYZob{}}\PY{n+nb}{input}\PY{o}{.}\PY{n}{shape}\PY{l+s+si}{\PYZcb{}}\PY{l+s+s1}{\PYZsq{}}\PY{p}{)}
\PY{n+nb}{print}\PY{p}{(}\PY{l+s+sa}{f}\PY{l+s+s1}{\PYZsq{}}\PY{l+s+s1}{Sortie (BxFxHxW): }\PY{l+s+si}{\PYZob{}}\PY{n}{output}\PY{o}{.}\PY{n}{shape}\PY{l+s+si}{\PYZcb{}}\PY{l+s+s1}{\PYZsq{}}\PY{p}{)}

\PY{n}{fig}\PY{p}{,} \PY{n}{axs} \PY{o}{=} \PY{n}{plt}\PY{o}{.}\PY{n}{subplots}\PY{p}{(}\PY{l+m+mi}{1}\PY{p}{,} \PY{l+m+mi}{2}\PY{p}{,} \PY{n}{figsize}\PY{o}{=}\PY{p}{(}\PY{l+m+mi}{20}\PY{p}{,} \PY{l+m+mi}{5}\PY{p}{)}\PY{p}{)}
\PY{k}{for} \PY{n}{i} \PY{o+ow}{in} \PY{n+nb}{range}\PY{p}{(}\PY{l+m+mi}{2}\PY{p}{)}\PY{p}{:}
    \PY{n}{axs}\PY{p}{[}\PY{n}{i}\PY{p}{]}\PY{o}{.}\PY{n}{imshow}\PY{p}{(}\PY{n}{output}\PY{o}{.}\PY{n}{detach}\PY{p}{(}\PY{p}{)}\PY{o}{.}\PY{n}{data}\PY{o}{.}\PY{n}{numpy}\PY{p}{(}\PY{p}{)}\PY{p}{[}\PY{l+m+mi}{0}\PY{p}{,}\PY{n}{i}\PY{p}{]}\PY{p}{,} \PY{n}{vmin}\PY{o}{=}\PY{o}{\PYZhy{}}\PY{l+m+mi}{5000}\PY{p}{,} \PY{n}{vmax}\PY{o}{=}\PY{l+m+mi}{5000}\PY{p}{,} \PY{n}{cmap}\PY{o}{=} \PY{l+s+s1}{\PYZsq{}}\PY{l+s+s1}{gray}\PY{l+s+s1}{\PYZsq{}}\PY{p}{)}
    \PY{n}{axs}\PY{p}{[}\PY{n}{i}\PY{p}{]}\PY{o}{.}\PY{n}{set\PYZus{}title}\PY{p}{(}\PY{l+s+sa}{f}\PY{l+s+s1}{\PYZsq{}}\PY{l+s+s1}{Filtrage }\PY{l+s+si}{\PYZob{}}\PY{n}{i}\PY{o}{+}\PY{l+m+mi}{1}\PY{l+s+si}{\PYZcb{}}\PY{l+s+s1}{\PYZsq{}}\PY{p}{)}
\PY{n}{plt}\PY{o}{.}\PY{n}{show}\PY{p}{(}\PY{p}{)}
\end{Verbatim}
\end{tcolorbox}

    \begin{Verbatim}[commandchars=\\\{\}]
Image (BxCxHxW): torch.Size([1, 3, 771, 1311])
Sortie (BxFxHxW): torch.Size([1, 2, 771, 1311])
    \end{Verbatim}

    \begin{center}
    \adjustimage{max size={0.9\linewidth}{0.9\paperheight}}{04-TransformationSpatiales_files/04-TransformationSpatiales_26_1.png}
    \end{center}
    { \hspace*{\fill} \\}
    
    \hypertarget{filtrage-adaptatif}{%
\subsubsection{Filtrage adaptatif}\label{filtrage-adaptatif}}

Les filtrages adaptatifs consistent à appliquer un traitement en
fonction du contenu local d'une image. Le filtre n'est alors plus
stationnaire et sa réponse peut varier en fonction du contenu local. Ce
type de filtre est très utilisé pour filtrer les images SAR (Synthetic
Aperture Radar) qui sont dégradées par un bruit multiplicatif que l'on
appelle \emph{speckle}. On peut voir un exemple d'une image Sentinel-1
(bande HH) sur la région de Montréal, remarquée que l'image est affichée
en dB en appliquant la fonction \texttt{log10}.

    \begin{tcolorbox}[breakable, size=fbox, boxrule=1pt, pad at break*=1mm,colback=cellbackground, colframe=cellborder]
\prompt{In}{incolor}{15}{\boxspacing}
\begin{Verbatim}[commandchars=\\\{\}]
\PY{c+c1}{\PYZsh{}| eval: true}
\PY{n+nb}{print}\PY{p}{(}\PY{n}{img\PYZus{}SAR}\PY{o}{.}\PY{n}{rio}\PY{o}{.}\PY{n}{resolution}\PY{p}{(}\PY{p}{)}\PY{p}{)}
\PY{n+nb}{print}\PY{p}{(}\PY{n}{img\PYZus{}SAR}\PY{o}{.}\PY{n}{rio}\PY{o}{.}\PY{n}{crs}\PY{p}{)}
\PY{n}{fig}\PY{p}{,} \PY{n}{axs} \PY{o}{=} \PY{n}{plt}\PY{o}{.}\PY{n}{subplots}\PY{p}{(}\PY{l+m+mi}{1}\PY{p}{,} \PY{l+m+mi}{1}\PY{p}{,} \PY{n}{figsize}\PY{o}{=}\PY{p}{(}\PY{l+m+mi}{6}\PY{p}{,} \PY{l+m+mi}{4}\PY{p}{)}\PY{p}{)}
\PY{n}{xr}\PY{o}{.}\PY{n}{ufuncs}\PY{o}{.}\PY{n}{log10}\PY{p}{(}\PY{n}{img\PYZus{}SAR}\PY{o}{.}\PY{n}{sel}\PY{p}{(}\PY{n}{band}\PY{o}{=}\PY{l+m+mi}{1}\PY{p}{)}\PY{o}{.}\PY{n}{drop}\PY{p}{(}\PY{l+s+s2}{\PYZdq{}}\PY{l+s+s2}{band}\PY{l+s+s2}{\PYZdq{}}\PY{p}{)}\PY{p}{)}\PY{o}{.}\PY{n}{plot}\PY{p}{(}\PY{p}{)}
\PY{n}{axs}\PY{o}{.}\PY{n}{set\PYZus{}title}\PY{p}{(}\PY{l+s+s2}{\PYZdq{}}\PY{l+s+s2}{Image SAR Sentinel\PYZhy{}1 (dB)}\PY{l+s+s2}{\PYZdq{}}\PY{p}{)}
\end{Verbatim}
\end{tcolorbox}

    \begin{Verbatim}[commandchars=\\\{\}]
(0.00029254428869762705, -0.000287092818453516)
EPSG:4326
    \end{Verbatim}

    \begin{Verbatim}[commandchars=\\\{\}]
/tmp/ipykernel\_5972/2357378870.py:5: DeprecationWarning: dropping variables
using `drop` is deprecated; use drop\_vars.
  xr.ufuncs.log10(img\_SAR.sel(band=1).drop("band")).plot()
Warning 1: TIFFReadDirectory:Sum of Photometric type-related color channels and
ExtraSamples doesn't match SamplesPerPixel. Defining non-color channels as
ExtraSamples.
    \end{Verbatim}

            \begin{tcolorbox}[breakable, size=fbox, boxrule=.5pt, pad at break*=1mm, opacityfill=0]
\prompt{Out}{outcolor}{15}{\boxspacing}
\begin{Verbatim}[commandchars=\\\{\}]
Text(0.5, 1.0, 'Image SAR Sentinel-1 (dB)')
\end{Verbatim}
\end{tcolorbox}
        
    \begin{center}
    \adjustimage{max size={0.9\linewidth}{0.9\paperheight}}{04-TransformationSpatiales_files/04-TransformationSpatiales_28_3.png}
    \end{center}
    { \hspace*{\fill} \\}
    
    Un des filtres les plus simples pour réduire le bruit est d'appliquer un
filtre moyenne, par exemple un \(5x5\) ci dessous:

    \begin{tcolorbox}[breakable, size=fbox, boxrule=1pt, pad at break*=1mm,colback=cellbackground, colframe=cellborder]
\prompt{In}{incolor}{16}{\boxspacing}
\begin{Verbatim}[commandchars=\\\{\}]
\PY{c+c1}{\PYZsh{}| eval: true}
\PY{n}{rolling\PYZus{}win} \PY{o}{=} \PY{n}{img\PYZus{}SAR}\PY{o}{.}\PY{n}{sel}\PY{p}{(}\PY{n}{band}\PY{o}{=}\PY{l+m+mi}{2}\PY{p}{)}\PY{o}{.}\PY{n}{rolling}\PY{p}{(}\PY{n}{x}\PY{o}{=}\PY{l+m+mi}{5}\PY{p}{,} \PY{n}{y}\PY{o}{=}\PY{l+m+mi}{5}\PY{p}{,}  \PY{n}{min\PYZus{}periods}\PY{o}{=} \PY{l+m+mi}{3}\PY{p}{,} \PY{n}{center}\PY{o}{=} \PY{k+kc}{True}\PY{p}{)}\PY{o}{.}\PY{n}{construct}\PY{p}{(}\PY{n}{x}\PY{o}{=}\PY{l+s+s2}{\PYZdq{}}\PY{l+s+s2}{x\PYZus{}win}\PY{l+s+s2}{\PYZdq{}}\PY{p}{,} \PY{n}{y}\PY{o}{=}\PY{l+s+s2}{\PYZdq{}}\PY{l+s+s2}{y\PYZus{}win}\PY{l+s+s2}{\PYZdq{}}\PY{p}{,} \PY{n}{keep\PYZus{}attrs}\PY{o}{=} \PY{k+kc}{True}\PY{p}{)}
\PY{n}{filtre\PYZus{}moyen}\PY{o}{=} \PY{n}{rolling\PYZus{}win}\PY{o}{.}\PY{n}{mean}\PY{p}{(}\PY{n}{dim}\PY{o}{=} \PY{p}{[}\PY{l+s+s1}{\PYZsq{}}\PY{l+s+s1}{x\PYZus{}win}\PY{l+s+s1}{\PYZsq{}}\PY{p}{,} \PY{l+s+s1}{\PYZsq{}}\PY{l+s+s1}{y\PYZus{}win}\PY{l+s+s1}{\PYZsq{}}\PY{p}{]}\PY{p}{,} \PY{n}{skipna}\PY{o}{=} \PY{k+kc}{True}\PY{p}{)}
\PY{n}{fig}\PY{p}{,} \PY{n}{axs} \PY{o}{=} \PY{n}{plt}\PY{o}{.}\PY{n}{subplots}\PY{p}{(}\PY{l+m+mi}{1}\PY{p}{,} \PY{l+m+mi}{1}\PY{p}{,} \PY{n}{figsize}\PY{o}{=}\PY{p}{(}\PY{l+m+mi}{6}\PY{p}{,} \PY{l+m+mi}{4}\PY{p}{)}\PY{p}{)}
\PY{n}{xr}\PY{o}{.}\PY{n}{ufuncs}\PY{o}{.}\PY{n}{log10}\PY{p}{(}\PY{n}{filtre\PYZus{}moyen}\PY{p}{)}\PY{o}{.}\PY{n}{plot}\PY{o}{.}\PY{n}{imshow}\PY{p}{(}\PY{p}{)}
\PY{n}{axs}\PY{o}{.}\PY{n}{set\PYZus{}title}\PY{p}{(}\PY{l+s+s2}{\PYZdq{}}\PY{l+s+s2}{Filtrage moyen 5x5 (dB)}\PY{l+s+s2}{\PYZdq{}}\PY{p}{)}
\end{Verbatim}
\end{tcolorbox}

            \begin{tcolorbox}[breakable, size=fbox, boxrule=.5pt, pad at break*=1mm, opacityfill=0]
\prompt{Out}{outcolor}{16}{\boxspacing}
\begin{Verbatim}[commandchars=\\\{\}]
Text(0.5, 1.0, 'Filtrage moyen 5x5 (dB)')
\end{Verbatim}
\end{tcolorbox}
        
    \begin{center}
    \adjustimage{max size={0.9\linewidth}{0.9\paperheight}}{04-TransformationSpatiales_files/04-TransformationSpatiales_30_1.png}
    \end{center}
    { \hspace*{\fill} \\}
    
    Au lieu d'appliquer un filtre moyen de manière indiscriminée, le filtre
de Lee {[}@Lee-1986{]} applique une pondération en fonction du contenu
local de l'image \(I\) dans sa forme la plus simple:

\[ 
\begin{aligned}
I_F & = I_M + K \times (I - I_M) \\
K & = \frac{\sigma^2_I}{\sigma^2_I + \sigma^2_{bruit}}
\end{aligned}
\]\{\#eq-lee-filter\}

Ainsi si la variance locale est élevée \(K\) s'approche de \(1\)
préservant ainsi les détails de l'image sinon l'image moyenne \(I_M\)
est appliquée.

    \begin{tcolorbox}[breakable, size=fbox, boxrule=1pt, pad at break*=1mm,colback=cellbackground, colframe=cellborder]
\prompt{In}{incolor}{17}{\boxspacing}
\begin{Verbatim}[commandchars=\\\{\}]
\PY{c+c1}{\PYZsh{}| eval: true}
\PY{n}{rolling\PYZus{}win} \PY{o}{=} \PY{n}{img\PYZus{}SAR}\PY{o}{.}\PY{n}{sel}\PY{p}{(}\PY{n}{band}\PY{o}{=}\PY{l+m+mi}{2}\PY{p}{)}\PY{o}{.}\PY{n}{rolling}\PY{p}{(}\PY{n}{x}\PY{o}{=}\PY{l+m+mi}{5}\PY{p}{,} \PY{n}{y}\PY{o}{=}\PY{l+m+mi}{5}\PY{p}{,}  \PY{n}{min\PYZus{}periods}\PY{o}{=} \PY{l+m+mi}{3}\PY{p}{,} \PY{n}{center}\PY{o}{=} \PY{k+kc}{True}\PY{p}{)}\PY{o}{.}\PY{n}{construct}\PY{p}{(}\PY{n}{x}\PY{o}{=}\PY{l+s+s2}{\PYZdq{}}\PY{l+s+s2}{x\PYZus{}win}\PY{l+s+s2}{\PYZdq{}}\PY{p}{,} \PY{n}{y}\PY{o}{=}\PY{l+s+s2}{\PYZdq{}}\PY{l+s+s2}{y\PYZus{}win}\PY{l+s+s2}{\PYZdq{}}\PY{p}{,} \PY{n}{keep\PYZus{}attrs}\PY{o}{=} \PY{k+kc}{True}\PY{p}{)}
\PY{n}{filtre\PYZus{}moyen}\PY{o}{=} \PY{n}{rolling\PYZus{}win}\PY{o}{.}\PY{n}{mean}\PY{p}{(}\PY{n}{dim}\PY{o}{=} \PY{p}{[}\PY{l+s+s1}{\PYZsq{}}\PY{l+s+s1}{x\PYZus{}win}\PY{l+s+s1}{\PYZsq{}}\PY{p}{,} \PY{l+s+s1}{\PYZsq{}}\PY{l+s+s1}{y\PYZus{}win}\PY{l+s+s1}{\PYZsq{}}\PY{p}{]}\PY{p}{,} \PY{n}{skipna}\PY{o}{=} \PY{k+kc}{True}\PY{p}{)}
\PY{n}{ecart\PYZus{}type}\PY{o}{=} \PY{n}{rolling\PYZus{}win}\PY{o}{.}\PY{n}{std}\PY{p}{(}\PY{n}{dim}\PY{o}{=} \PY{p}{[}\PY{l+s+s1}{\PYZsq{}}\PY{l+s+s1}{x\PYZus{}win}\PY{l+s+s1}{\PYZsq{}}\PY{p}{,} \PY{l+s+s1}{\PYZsq{}}\PY{l+s+s1}{y\PYZus{}win}\PY{l+s+s1}{\PYZsq{}}\PY{p}{]}\PY{p}{,} \PY{n}{skipna}\PY{o}{=} \PY{k+kc}{True}\PY{p}{)}
\PY{n}{cv}\PY{o}{=} \PY{n}{ecart\PYZus{}type}\PY{o}{/}\PY{n}{filtre\PYZus{}moyen}
\PY{n}{ponderation} \PY{o}{=} \PY{p}{(}\PY{n}{cv} \PY{o}{\PYZhy{}} \PY{l+m+mf}{0.25}\PY{p}{)} \PY{o}{/} \PY{n}{cv}

\PY{n}{fig}\PY{p}{,} \PY{n}{axes} \PY{o}{=} \PY{n}{plt}\PY{o}{.}\PY{n}{subplots}\PY{p}{(}\PY{n}{nrows}\PY{o}{=}\PY{l+m+mi}{1}\PY{p}{,} \PY{n}{ncols}\PY{o}{=}\PY{l+m+mi}{2}\PY{p}{,} \PY{n}{figsize}\PY{o}{=}\PY{p}{(}\PY{l+m+mi}{12}\PY{p}{,} \PY{l+m+mi}{4}\PY{p}{)}\PY{p}{)}
\PY{n}{plt}\PY{o}{.}\PY{n}{subplot}\PY{p}{(}\PY{l+m+mi}{1}\PY{p}{,} \PY{l+m+mi}{2}\PY{p}{,} \PY{l+m+mi}{1}\PY{p}{)}
\PY{n}{cv}\PY{o}{.}\PY{n}{plot}\PY{o}{.}\PY{n}{imshow}\PY{p}{(} \PY{n}{vmin}\PY{o}{=}\PY{l+m+mi}{0}\PY{p}{,} \PY{n}{vmax}\PY{o}{=}\PY{l+m+mi}{2}\PY{p}{)}
\PY{n}{axes}\PY{p}{[}\PY{l+m+mi}{0}\PY{p}{]}\PY{o}{.}\PY{n}{set\PYZus{}title}\PY{p}{(}\PY{l+s+s2}{\PYZdq{}}\PY{l+s+s2}{CV}\PY{l+s+s2}{\PYZdq{}}\PY{p}{)}
\PY{n}{plt}\PY{o}{.}\PY{n}{subplot}\PY{p}{(}\PY{l+m+mi}{1}\PY{p}{,} \PY{l+m+mi}{2}\PY{p}{,} \PY{l+m+mi}{2}\PY{p}{)}
\PY{n}{ponderation}\PY{o}{.}\PY{n}{plot}\PY{o}{.}\PY{n}{imshow}\PY{p}{(} \PY{n}{vmin}\PY{o}{=}\PY{l+m+mi}{0}\PY{p}{,} \PY{n}{vmax}\PY{o}{=}\PY{l+m+mi}{1}\PY{p}{)} 
\PY{n}{axes}\PY{p}{[}\PY{l+m+mi}{1}\PY{p}{]}\PY{o}{.}\PY{n}{set\PYZus{}title}\PY{p}{(}\PY{l+s+s2}{\PYZdq{}}\PY{l+s+s2}{Pondération}\PY{l+s+s2}{\PYZdq{}}\PY{p}{)}
\end{Verbatim}
\end{tcolorbox}

            \begin{tcolorbox}[breakable, size=fbox, boxrule=.5pt, pad at break*=1mm, opacityfill=0]
\prompt{Out}{outcolor}{17}{\boxspacing}
\begin{Verbatim}[commandchars=\\\{\}]
Text(0.5, 1.0, 'Pondération')
\end{Verbatim}
\end{tcolorbox}
        
    \begin{center}
    \adjustimage{max size={0.9\linewidth}{0.9\paperheight}}{04-TransformationSpatiales_files/04-TransformationSpatiales_32_1.png}
    \end{center}
    { \hspace*{\fill} \\}
    
    On zoomant sur l'image on peut clairement voir que les détails de
l'image sont mieux préservés:

    \begin{tcolorbox}[breakable, size=fbox, boxrule=1pt, pad at break*=1mm,colback=cellbackground, colframe=cellborder]
\prompt{In}{incolor}{18}{\boxspacing}
\begin{Verbatim}[commandchars=\\\{\}]
\PY{c+c1}{\PYZsh{}| eval: true}
\PY{c+c1}{\PYZsh{}| echo: false}

\PY{n}{filtered}\PY{o}{=} \PY{n}{filtre\PYZus{}moyen} \PY{o}{+} \PY{n}{ponderation} \PY{o}{*} \PY{p}{(}\PY{n}{img\PYZus{}SAR}\PY{o}{.}\PY{n}{sel}\PY{p}{(}\PY{n}{band}\PY{o}{=}\PY{l+m+mi}{1}\PY{p}{)}\PY{o}{.}\PY{n}{drop}\PY{p}{(}\PY{l+s+s2}{\PYZdq{}}\PY{l+s+s2}{band}\PY{l+s+s2}{\PYZdq{}}\PY{p}{)} \PY{o}{\PYZhy{}} \PY{n}{filtre\PYZus{}moyen}\PY{p}{)}
\PY{n}{fig}\PY{p}{,} \PY{n}{axes} \PY{o}{=} \PY{n}{plt}\PY{o}{.}\PY{n}{subplots}\PY{p}{(}\PY{n}{nrows}\PY{o}{=}\PY{l+m+mi}{1}\PY{p}{,} \PY{n}{ncols}\PY{o}{=}\PY{l+m+mi}{2}\PY{p}{,} \PY{n}{figsize}\PY{o}{=}\PY{p}{(}\PY{l+m+mi}{10}\PY{p}{,} \PY{l+m+mi}{4}\PY{p}{)}\PY{p}{)}
\PY{n}{plt}\PY{o}{.}\PY{n}{subplot}\PY{p}{(}\PY{l+m+mi}{1}\PY{p}{,} \PY{l+m+mi}{2}\PY{p}{,} \PY{l+m+mi}{1}\PY{p}{)}
\PY{n}{xr}\PY{o}{.}\PY{n}{ufuncs}\PY{o}{.}\PY{n}{log10}\PY{p}{(}\PY{n}{filtre\PYZus{}moyen}\PY{p}{)}\PY{o}{.}\PY{n}{isel}\PY{p}{(}\PY{n}{x}\PY{o}{=}\PY{n+nb}{slice}\PY{p}{(}\PY{k+kc}{None}\PY{p}{,} \PY{l+m+mi}{250}\PY{p}{)}\PY{p}{,}\PY{n}{y}\PY{o}{=}\PY{n+nb}{slice}\PY{p}{(}\PY{k+kc}{None}\PY{p}{,} \PY{l+m+mi}{250}\PY{p}{)}\PY{p}{)}\PY{o}{.}\PY{n}{plot}\PY{o}{.}\PY{n}{imshow}\PY{p}{(}\PY{p}{)}
\PY{n}{axes}\PY{p}{[}\PY{l+m+mi}{0}\PY{p}{]}\PY{o}{.}\PY{n}{set\PYZus{}title}\PY{p}{(}\PY{l+s+s2}{\PYZdq{}}\PY{l+s+s2}{Filtre moyen}\PY{l+s+s2}{\PYZdq{}}\PY{p}{)}
\PY{n}{plt}\PY{o}{.}\PY{n}{subplot}\PY{p}{(}\PY{l+m+mi}{1}\PY{p}{,} \PY{l+m+mi}{2}\PY{p}{,} \PY{l+m+mi}{2}\PY{p}{)}
\PY{n}{xr}\PY{o}{.}\PY{n}{ufuncs}\PY{o}{.}\PY{n}{log10}\PY{p}{(}\PY{n}{filtered}\PY{p}{)}\PY{o}{.}\PY{n}{isel}\PY{p}{(}\PY{n}{x}\PY{o}{=}\PY{n+nb}{slice}\PY{p}{(}\PY{k+kc}{None}\PY{p}{,} \PY{l+m+mi}{250}\PY{p}{)}\PY{p}{,}\PY{n}{y}\PY{o}{=}\PY{n+nb}{slice}\PY{p}{(}\PY{k+kc}{None}\PY{p}{,} \PY{l+m+mi}{250}\PY{p}{)}\PY{p}{)}\PY{o}{.}\PY{n}{plot}\PY{o}{.}\PY{n}{imshow}\PY{p}{(}\PY{p}{)} \PY{c+c1}{\PYZsh{}cmap=plt.get\PYZus{}cmap(\PYZsq{}hot\PYZsq{}),}
\PY{n}{axes}\PY{p}{[}\PY{l+m+mi}{1}\PY{p}{]}\PY{o}{.}\PY{n}{set\PYZus{}title}\PY{p}{(}\PY{l+s+s2}{\PYZdq{}}\PY{l+s+s2}{Filtre de Lee}\PY{l+s+s2}{\PYZdq{}}\PY{p}{)}
\end{Verbatim}
\end{tcolorbox}

    \begin{Verbatim}[commandchars=\\\{\}]
/tmp/ipykernel\_5972/2599367426.py:4: DeprecationWarning: dropping variables
using `drop` is deprecated; use drop\_vars.
  filtered= filtre\_moyen + ponderation * (img\_SAR.sel(band=1).drop("band") -
filtre\_moyen)
    \end{Verbatim}

            \begin{tcolorbox}[breakable, size=fbox, boxrule=.5pt, pad at break*=1mm, opacityfill=0]
\prompt{Out}{outcolor}{18}{\boxspacing}
\begin{Verbatim}[commandchars=\\\{\}]
Text(0.5, 1.0, 'Filtre de Lee')
\end{Verbatim}
\end{tcolorbox}
        
    \begin{center}
    \adjustimage{max size={0.9\linewidth}{0.9\paperheight}}{04-TransformationSpatiales_files/04-TransformationSpatiales_34_2.png}
    \end{center}
    { \hspace*{\fill} \\}
    
    \hypertarget{segmentation}{%
\subsection{Segmentation}\label{segmentation}}

La segmentation d'image consiste à séparer une image en régions
homogènes spatialement connexes (segments) où les valeurs sont uniformes
selon un certain critère (couleurs, texture, etc.). Une image présente
généralement beaucoup de pixels redondants, l'intérêt de ce type de
méthode est essentiellement de réduire la quantité de pxiels nécessaire.
En télédétection, on parle souvent d'approche objet. En vision par
ordinateur, on parle parfois de super-pixel. Il existe de nombreuses
méthodes de segmentation, la librairie \texttt{sickit-image} rend
disponible plusieurs implémentations sur des images RVB
(\href{https://scikit-image.org/docs/stable/auto_examples/segmentation/plot_segmentations.html\#sphx-glr-auto-examples-segmentation-plot-segmentations-py}{Comparison
of segmentation and superpixel algorithms --- skimage 0.25.0
documentation}).

\hypertarget{super-pixel}{%
\subsubsection{Super-pixel}\label{super-pixel}}

Ce type de méthode cherche à former des régions homogènes et compactes
dans l'image {[}@Achanta-2012{]}. Une des méthodes les plus simples est
la méthode SLIC (\emph{Simple Linear Iterative Clustering}), elle
combine un regroupement de type K-moyenne avec une distance hybride qui
prend en compte les différences de couleur entre pixels mais aussi leur
distance par rapport centre du super-pixel:

\begin{enumerate}
\def\labelenumi{\arabic{enumi}.}
\item
  Décomposer l'image en N régions régulières de taille \(S \times S\)
\item
  Initialiser les centres \(C_k\) de chaque segment \(k\)
\item
  Rechercher les pixels qui ont la distance la plus petite dans une
  région \(2S \times 2S\):
\end{enumerate}

\[
D_{SLIC}= d_{couleur} + \frac{m}{S}d_{xy}
\]

\begin{enumerate}
\def\labelenumi{\arabic{enumi}.}
\setcounter{enumi}{2}
\tightlist
\item
  Mettre à jour les centre \(C_k\) de chaque segment \(k\), retourner à
  l'étape 3
\end{enumerate}

Les régions évoluent rapidement avec les itérations, plus le poids \(m\)
est élevé, plus la forme du super-pixel est contrainte et ne suivra pas
vraiment le contenu de l'image:

    \begin{tcolorbox}[breakable, size=fbox, boxrule=1pt, pad at break*=1mm,colback=cellbackground, colframe=cellborder]
\prompt{In}{incolor}{19}{\boxspacing}
\begin{Verbatim}[commandchars=\\\{\}]
\PY{c+c1}{\PYZsh{}| eval: true}

\PY{k+kn}{from} \PY{n+nn}{skimage}\PY{n+nn}{.}\PY{n+nn}{color} \PY{k+kn}{import} \PY{n}{rgb2gray}

\PY{k+kn}{from} \PY{n+nn}{skimage}\PY{n+nn}{.}\PY{n+nn}{segmentation} \PY{k+kn}{import} \PY{n}{slic}\PY{p}{,} \PY{n}{mark\PYZus{}boundaries}

\PY{n}{img} \PY{o}{=} \PY{n}{img\PYZus{}rgb}\PY{o}{.}\PY{n}{to\PYZus{}numpy}\PY{p}{(}\PY{p}{)}\PY{o}{.}\PY{n}{astype}\PY{p}{(}\PY{l+s+s1}{\PYZsq{}}\PY{l+s+s1}{uint8}\PY{l+s+s1}{\PYZsq{}}\PY{p}{)}\PY{o}{.}\PY{n}{transpose}\PY{p}{(}\PY{l+m+mi}{1}\PY{p}{,}\PY{l+m+mi}{2}\PY{p}{,}\PY{l+m+mi}{0}\PY{p}{)} 

\PY{n}{segments\PYZus{}slic1} \PY{o}{=} \PY{n}{slic}\PY{p}{(}\PY{n}{img}\PY{p}{,} \PY{n}{n\PYZus{}segments}\PY{o}{=}\PY{l+m+mi}{250}\PY{p}{,} \PY{n}{compactness}\PY{o}{=}\PY{l+m+mi}{10}\PY{p}{,} \PY{n}{sigma}\PY{o}{=}\PY{l+m+mi}{1}\PY{p}{,} \PY{n}{start\PYZus{}label}\PY{o}{=}\PY{l+m+mi}{1}\PY{p}{,} \PY{n}{max\PYZus{}num\PYZus{}iter}\PY{o}{=}\PY{l+m+mi}{1}\PY{p}{)}
\PY{n}{segments\PYZus{}slic2} \PY{o}{=} \PY{n}{slic}\PY{p}{(}\PY{n}{img}\PY{p}{,} \PY{n}{n\PYZus{}segments}\PY{o}{=}\PY{l+m+mi}{250}\PY{p}{,} \PY{n}{compactness}\PY{o}{=}\PY{l+m+mi}{10}\PY{p}{,} \PY{n}{sigma}\PY{o}{=}\PY{l+m+mi}{1}\PY{p}{,} \PY{n}{start\PYZus{}label}\PY{o}{=}\PY{l+m+mi}{1}\PY{p}{,} \PY{n}{max\PYZus{}num\PYZus{}iter}\PY{o}{=}\PY{l+m+mi}{2}\PY{p}{)}
\PY{n}{segments\PYZus{}slic100} \PY{o}{=} \PY{n}{slic}\PY{p}{(}\PY{n}{img}\PY{p}{,} \PY{n}{n\PYZus{}segments}\PY{o}{=}\PY{l+m+mi}{250}\PY{p}{,} \PY{n}{compactness}\PY{o}{=}\PY{l+m+mi}{100}\PY{p}{,} \PY{n}{sigma}\PY{o}{=}\PY{l+m+mi}{1}\PY{p}{,} \PY{n}{start\PYZus{}label}\PY{o}{=}\PY{l+m+mi}{1}\PY{p}{,} \PY{n}{max\PYZus{}num\PYZus{}iter}\PY{o}{=}\PY{l+m+mi}{10}\PY{p}{)}
\PY{n}{segments\PYZus{}slic100b} \PY{o}{=} \PY{n}{slic}\PY{p}{(}\PY{n}{img}\PY{p}{,} \PY{n}{n\PYZus{}segments}\PY{o}{=}\PY{l+m+mi}{250}\PY{p}{,} \PY{n}{compactness}\PY{o}{=}\PY{l+m+mi}{10}\PY{p}{,} \PY{n}{sigma}\PY{o}{=}\PY{l+m+mi}{1}\PY{p}{,} \PY{n}{start\PYZus{}label}\PY{o}{=}\PY{l+m+mi}{1}\PY{p}{,} \PY{n}{max\PYZus{}num\PYZus{}iter}\PY{o}{=}\PY{l+m+mi}{10}\PY{p}{)}

\PY{n+nb}{print}\PY{p}{(}\PY{l+s+sa}{f}\PY{l+s+s1}{\PYZsq{}}\PY{l+s+s1}{SLIC nombre de segments: }\PY{l+s+si}{\PYZob{}}\PY{n+nb}{len}\PY{p}{(}\PY{n}{np}\PY{o}{.}\PY{n}{unique}\PY{p}{(}\PY{n}{segments\PYZus{}slic1}\PY{p}{)}\PY{p}{)}\PY{l+s+si}{\PYZcb{}}\PY{l+s+s1}{\PYZsq{}}\PY{p}{)}

\PY{n}{fig}\PY{p}{,} \PY{n}{ax} \PY{o}{=} \PY{n}{plt}\PY{o}{.}\PY{n}{subplots}\PY{p}{(}\PY{l+m+mi}{2}\PY{p}{,} \PY{l+m+mi}{2}\PY{p}{,} \PY{n}{figsize}\PY{o}{=}\PY{p}{(}\PY{l+m+mi}{10}\PY{p}{,} \PY{l+m+mi}{6}\PY{p}{)}\PY{p}{,} \PY{n}{sharex}\PY{o}{=}\PY{k+kc}{True}\PY{p}{,} \PY{n}{sharey}\PY{o}{=}\PY{k+kc}{True}\PY{p}{)}

\PY{n}{ax}\PY{p}{[}\PY{l+m+mi}{0}\PY{p}{,} \PY{l+m+mi}{0}\PY{p}{]}\PY{o}{.}\PY{n}{imshow}\PY{p}{(}\PY{n}{mark\PYZus{}boundaries}\PY{p}{(}\PY{n}{img}\PY{p}{,} \PY{n}{segments\PYZus{}slic1}\PY{p}{)}\PY{p}{)}
\PY{n}{ax}\PY{p}{[}\PY{l+m+mi}{0}\PY{p}{,} \PY{l+m+mi}{0}\PY{p}{]}\PY{o}{.}\PY{n}{set\PYZus{}title}\PY{p}{(}\PY{l+s+s2}{\PYZdq{}}\PY{l+s+s2}{Initialisation}\PY{l+s+s2}{\PYZdq{}}\PY{p}{)}
\PY{n}{ax}\PY{p}{[}\PY{l+m+mi}{0}\PY{p}{,} \PY{l+m+mi}{1}\PY{p}{]}\PY{o}{.}\PY{n}{imshow}\PY{p}{(}\PY{n}{mark\PYZus{}boundaries}\PY{p}{(}\PY{n}{img}\PY{p}{,} \PY{n}{segments\PYZus{}slic2}\PY{p}{)}\PY{p}{)}
\PY{n}{ax}\PY{p}{[}\PY{l+m+mi}{0}\PY{p}{,} \PY{l+m+mi}{1}\PY{p}{]}\PY{o}{.}\PY{n}{set\PYZus{}title}\PY{p}{(}\PY{l+s+s1}{\PYZsq{}}\PY{l+s+s1}{2 itérations}\PY{l+s+s1}{\PYZsq{}}\PY{p}{)}
\PY{n}{ax}\PY{p}{[}\PY{l+m+mi}{1}\PY{p}{,} \PY{l+m+mi}{0}\PY{p}{]}\PY{o}{.}\PY{n}{imshow}\PY{p}{(}\PY{n}{mark\PYZus{}boundaries}\PY{p}{(}\PY{n}{img}\PY{p}{,} \PY{n}{segments\PYZus{}slic100}\PY{p}{)}\PY{p}{)}
\PY{n}{ax}\PY{p}{[}\PY{l+m+mi}{1}\PY{p}{,} \PY{l+m+mi}{0}\PY{p}{]}\PY{o}{.}\PY{n}{set\PYZus{}title}\PY{p}{(}\PY{l+s+s1}{\PYZsq{}}\PY{l+s+s1}{10 itérations avec m=100}\PY{l+s+s1}{\PYZsq{}}\PY{p}{)}
\PY{n}{ax}\PY{p}{[}\PY{l+m+mi}{1}\PY{p}{,} \PY{l+m+mi}{1}\PY{p}{]}\PY{o}{.}\PY{n}{imshow}\PY{p}{(}\PY{n}{mark\PYZus{}boundaries}\PY{p}{(}\PY{n}{img}\PY{p}{,} \PY{n}{segments\PYZus{}slic100b}\PY{p}{)}\PY{p}{)}
\PY{n}{ax}\PY{p}{[}\PY{l+m+mi}{1}\PY{p}{,} \PY{l+m+mi}{1}\PY{p}{]}\PY{o}{.}\PY{n}{set\PYZus{}title}\PY{p}{(}\PY{l+s+s1}{\PYZsq{}}\PY{l+s+s1}{10 itérations avec m=10}\PY{l+s+s1}{\PYZsq{}}\PY{p}{)}

\PY{k}{for} \PY{n}{a} \PY{o+ow}{in} \PY{n}{ax}\PY{o}{.}\PY{n}{ravel}\PY{p}{(}\PY{p}{)}\PY{p}{:}
    \PY{n}{a}\PY{o}{.}\PY{n}{set\PYZus{}axis\PYZus{}off}\PY{p}{(}\PY{p}{)}

\PY{n}{plt}\PY{o}{.}\PY{n}{tight\PYZus{}layout}\PY{p}{(}\PY{p}{)}
\PY{n}{plt}\PY{o}{.}\PY{n}{show}\PY{p}{(}\PY{p}{)}
\end{Verbatim}
\end{tcolorbox}

    \begin{Verbatim}[commandchars=\\\{\}]
SLIC nombre de segments: 240
    \end{Verbatim}

    \begin{center}
    \adjustimage{max size={0.9\linewidth}{0.9\paperheight}}{04-TransformationSpatiales_files/04-TransformationSpatiales_36_1.png}
    \end{center}
    { \hspace*{\fill} \\}
    
    Le nombre de segments initial est probablement le paramètre le plus
important. Une manière de l'estimer est d'évaluer l'échelle moyenne des
segments homogènes dans l'image à analyser. On peut observer ci-dessous
l'impact de passer d'une échelle \(40 \times 40\) à \(20 \times 20\). En
prenant la moyenne de chaque segment, on peut voir tout de suite que
\(40 \times 40\) résulte en des segments trop grands mélangeant
différentes classes.

    \begin{tcolorbox}[breakable, size=fbox, boxrule=1pt, pad at break*=1mm,colback=cellbackground, colframe=cellborder]
\prompt{In}{incolor}{20}{\boxspacing}
\begin{Verbatim}[commandchars=\\\{\}]
\PY{c+c1}{\PYZsh{}| eval: true}
\PY{k+kn}{from} \PY{n+nn}{skimage} \PY{k+kn}{import} \PY{n}{color}\PY{p}{,} \PY{n}{segmentation}
\PY{n}{n\PYZus{}regions} \PY{o}{=} \PY{n+nb}{int}\PY{p}{(}\PY{p}{(}\PY{n}{img}\PY{o}{.}\PY{n}{shape}\PY{p}{[}\PY{l+m+mi}{0}\PY{p}{]} \PY{o}{*} \PY{n}{img}\PY{o}{.}\PY{n}{shape}\PY{p}{[}\PY{l+m+mi}{1}\PY{p}{]}\PY{p}{)}\PY{o}{/}\PY{p}{(}\PY{l+m+mi}{40}\PY{o}{*}\PY{l+m+mi}{40}\PY{p}{)}\PY{p}{)}
\PY{n+nb}{print}\PY{p}{(}\PY{l+s+s1}{\PYZsq{}}\PY{l+s+s1}{Nb segments: }\PY{l+s+s1}{\PYZsq{}}\PY{p}{,}\PY{n}{n\PYZus{}regions}\PY{p}{)}
\PY{n}{segments\PYZus{}slic\PYZus{}40} \PY{o}{=} \PY{n}{slic}\PY{p}{(}\PY{n}{img}\PY{p}{,} \PY{n}{n\PYZus{}segments}\PY{o}{=}\PY{n}{n\PYZus{}regions}\PY{p}{,} \PY{n}{compactness}\PY{o}{=}\PY{l+m+mi}{10}\PY{p}{,} \PY{n}{sigma}\PY{o}{=}\PY{l+m+mi}{1}\PY{p}{,} \PY{n}{start\PYZus{}label}\PY{o}{=}\PY{l+m+mi}{1}\PY{p}{,} \PY{n}{max\PYZus{}num\PYZus{}iter}\PY{o}{=}\PY{l+m+mi}{10}\PY{p}{)}
\PY{n+nb}{print}\PY{p}{(}\PY{l+s+sa}{f}\PY{l+s+s1}{\PYZsq{}}\PY{l+s+s1}{SLIC nombre de segments: }\PY{l+s+si}{\PYZob{}}\PY{n+nb}{len}\PY{p}{(}\PY{n}{np}\PY{o}{.}\PY{n}{unique}\PY{p}{(}\PY{n}{segments\PYZus{}slic\PYZus{}40}\PY{p}{)}\PY{p}{)}\PY{l+s+si}{\PYZcb{}}\PY{l+s+s1}{\PYZsq{}}\PY{p}{)}
\PY{n}{out} \PY{o}{=} \PY{n}{color}\PY{o}{.}\PY{n}{label2rgb}\PY{p}{(}\PY{n}{segments\PYZus{}slic\PYZus{}40}\PY{p}{,} \PY{n}{img}\PY{p}{,} \PY{n}{kind}\PY{o}{=}\PY{l+s+s1}{\PYZsq{}}\PY{l+s+s1}{avg}\PY{l+s+s1}{\PYZsq{}}\PY{p}{,} \PY{n}{bg\PYZus{}label}\PY{o}{=}\PY{l+m+mi}{0}\PY{p}{)}
\PY{n}{out\PYZus{}40} \PY{o}{=} \PY{n}{segmentation}\PY{o}{.}\PY{n}{mark\PYZus{}boundaries}\PY{p}{(}\PY{n}{out}\PY{p}{,} \PY{n}{segments\PYZus{}slic\PYZus{}40}\PY{p}{,} \PY{p}{(}\PY{l+m+mi}{0}\PY{p}{,} \PY{l+m+mi}{0}\PY{p}{,} \PY{l+m+mi}{0}\PY{p}{)}\PY{p}{)}

\PY{n}{n\PYZus{}regions} \PY{o}{=} \PY{n+nb}{int}\PY{p}{(}\PY{p}{(}\PY{n}{img}\PY{o}{.}\PY{n}{shape}\PY{p}{[}\PY{l+m+mi}{0}\PY{p}{]} \PY{o}{*} \PY{n}{img}\PY{o}{.}\PY{n}{shape}\PY{p}{[}\PY{l+m+mi}{1}\PY{p}{]}\PY{p}{)}\PY{o}{/}\PY{p}{(}\PY{l+m+mi}{20}\PY{o}{*}\PY{l+m+mi}{20}\PY{p}{)}\PY{p}{)}
\PY{n+nb}{print}\PY{p}{(}\PY{l+s+s1}{\PYZsq{}}\PY{l+s+s1}{Nb segments: }\PY{l+s+s1}{\PYZsq{}}\PY{p}{,}\PY{n}{n\PYZus{}regions}\PY{p}{)}
\PY{n}{segments\PYZus{}slic\PYZus{}20} \PY{o}{=} \PY{n}{slic}\PY{p}{(}\PY{n}{img}\PY{p}{,} \PY{n}{n\PYZus{}segments}\PY{o}{=}\PY{n}{n\PYZus{}regions}\PY{p}{,} \PY{n}{compactness}\PY{o}{=}\PY{l+m+mi}{10}\PY{p}{,} \PY{n}{sigma}\PY{o}{=}\PY{l+m+mi}{1}\PY{p}{,} \PY{n}{start\PYZus{}label}\PY{o}{=}\PY{l+m+mi}{1}\PY{p}{,} \PY{n}{max\PYZus{}num\PYZus{}iter}\PY{o}{=}\PY{l+m+mi}{10}\PY{p}{)}
\PY{n+nb}{print}\PY{p}{(}\PY{l+s+sa}{f}\PY{l+s+s1}{\PYZsq{}}\PY{l+s+s1}{SLIC nombre de segments: }\PY{l+s+si}{\PYZob{}}\PY{n+nb}{len}\PY{p}{(}\PY{n}{np}\PY{o}{.}\PY{n}{unique}\PY{p}{(}\PY{n}{segments\PYZus{}slic\PYZus{}20}\PY{p}{)}\PY{p}{)}\PY{l+s+si}{\PYZcb{}}\PY{l+s+s1}{\PYZsq{}}\PY{p}{)}
\PY{n}{out} \PY{o}{=} \PY{n}{color}\PY{o}{.}\PY{n}{label2rgb}\PY{p}{(}\PY{n}{segments\PYZus{}slic\PYZus{}20}\PY{p}{,} \PY{n}{img}\PY{p}{,} \PY{n}{kind}\PY{o}{=}\PY{l+s+s1}{\PYZsq{}}\PY{l+s+s1}{avg}\PY{l+s+s1}{\PYZsq{}}\PY{p}{,} \PY{n}{bg\PYZus{}label}\PY{o}{=}\PY{l+m+mi}{0}\PY{p}{)}
\PY{n}{out\PYZus{}20} \PY{o}{=} \PY{n}{segmentation}\PY{o}{.}\PY{n}{mark\PYZus{}boundaries}\PY{p}{(}\PY{n}{out}\PY{p}{,} \PY{n}{segments\PYZus{}slic\PYZus{}20}\PY{p}{,} \PY{p}{(}\PY{l+m+mi}{0}\PY{p}{,} \PY{l+m+mi}{0}\PY{p}{,} \PY{l+m+mi}{0}\PY{p}{)}\PY{p}{)}

\PY{n}{fig}\PY{p}{,} \PY{n}{ax} \PY{o}{=} \PY{n}{plt}\PY{o}{.}\PY{n}{subplots}\PY{p}{(}\PY{l+m+mi}{1}\PY{p}{,} \PY{l+m+mi}{2}\PY{p}{,} \PY{n}{figsize}\PY{o}{=}\PY{p}{(}\PY{l+m+mi}{10}\PY{p}{,} \PY{l+m+mi}{6}\PY{p}{)}\PY{p}{,} \PY{n}{sharex}\PY{o}{=}\PY{k+kc}{True}\PY{p}{,} \PY{n}{sharey}\PY{o}{=}\PY{k+kc}{True}\PY{p}{)}

\PY{n}{ax}\PY{p}{[}\PY{l+m+mi}{0}\PY{p}{]}\PY{o}{.}\PY{n}{imshow}\PY{p}{(}\PY{n}{out\PYZus{}40}\PY{p}{)}
\PY{n}{ax}\PY{p}{[}\PY{l+m+mi}{0}\PY{p}{]}\PY{o}{.}\PY{n}{set\PYZus{}title}\PY{p}{(}\PY{l+s+s2}{\PYZdq{}}\PY{l+s+s2}{Initialisation avec 631 segments}\PY{l+s+s2}{\PYZdq{}}\PY{p}{)}
\PY{n}{ax}\PY{p}{[}\PY{l+m+mi}{1}\PY{p}{]}\PY{o}{.}\PY{n}{imshow}\PY{p}{(}\PY{n}{out\PYZus{}20}\PY{p}{)}
\PY{n}{ax}\PY{p}{[}\PY{l+m+mi}{1}\PY{p}{]}\PY{o}{.}\PY{n}{set\PYZus{}title}\PY{p}{(}\PY{l+s+s1}{\PYZsq{}}\PY{l+s+s1}{Initialisation avec 2526 segments}\PY{l+s+s1}{\PYZsq{}}\PY{p}{)}
\PY{k}{for} \PY{n}{a} \PY{o+ow}{in} \PY{n}{ax}\PY{o}{.}\PY{n}{ravel}\PY{p}{(}\PY{p}{)}\PY{p}{:}
    \PY{n}{a}\PY{o}{.}\PY{n}{set\PYZus{}axis\PYZus{}off}\PY{p}{(}\PY{p}{)}
\PY{n}{plt}\PY{o}{.}\PY{n}{tight\PYZus{}layout}\PY{p}{(}\PY{p}{)}
\PY{n}{plt}\PY{o}{.}\PY{n}{show}\PY{p}{(}\PY{p}{)}
\end{Verbatim}
\end{tcolorbox}

    \begin{Verbatim}[commandchars=\\\{\}]
Nb segments:  631
    \end{Verbatim}

    \begin{Verbatim}[commandchars=\\\{\}]
SLIC nombre de segments: 459
    \end{Verbatim}

    \begin{Verbatim}[commandchars=\\\{\}]
Nb segments:  2526
    \end{Verbatim}

    \begin{Verbatim}[commandchars=\\\{\}]
SLIC nombre de segments: 2201
    \end{Verbatim}

    \begin{center}
    \adjustimage{max size={0.9\linewidth}{0.9\paperheight}}{04-TransformationSpatiales_files/04-TransformationSpatiales_38_4.png}
    \end{center}
    { \hspace*{\fill} \\}
    
    \hypertarget{vectorisation-et-rasterisation}{%
\subsection{Vectorisation et
rasterisation}\label{vectorisation-et-rasterisation}}

\hypertarget{analyse-de-terrain}{%
\subsection{Analyse de terrain}\label{analyse-de-terrain}}

\hypertarget{uxe9luxe9vation}{%
\subsubsection{Élévation}\label{uxe9luxe9vation}}

\hypertarget{pente}{%
\subsubsection{Pente}\label{pente}}

\hypertarget{ombrage}{%
\subsubsection{Ombrage}\label{ombrage}}

\hypertarget{visibilituxe9}{%
\subsubsection{Visibilité}\label{visibilituxe9}}

\hypertarget{exercices-de-ruxe9vision}{%
\subsection{Exercices de révision}\label{exercices-de-ruxe9vision}}


    % Add a bibliography block to the postdoc
    
    
    
\end{document}

\documentclass[11pt]{article}

    \usepackage[breakable]{tcolorbox}
    \usepackage{parskip} % Stop auto-indenting (to mimic markdown behaviour)
    

    % Basic figure setup, for now with no caption control since it's done
    % automatically by Pandoc (which extracts ![](path) syntax from Markdown).
    \usepackage{graphicx}
    % Keep aspect ratio if custom image width or height is specified
    \setkeys{Gin}{keepaspectratio}
    % Maintain compatibility with old templates. Remove in nbconvert 6.0
    \let\Oldincludegraphics\includegraphics
    % Ensure that by default, figures have no caption (until we provide a
    % proper Figure object with a Caption API and a way to capture that
    % in the conversion process - todo).
    \usepackage{caption}
    \DeclareCaptionFormat{nocaption}{}
    \captionsetup{format=nocaption,aboveskip=0pt,belowskip=0pt}

    \usepackage{float}
    \floatplacement{figure}{H} % forces figures to be placed at the correct location
    \usepackage{xcolor} % Allow colors to be defined
    \usepackage{enumerate} % Needed for markdown enumerations to work
    \usepackage{geometry} % Used to adjust the document margins
    \usepackage{amsmath} % Equations
    \usepackage{amssymb} % Equations
    \usepackage{textcomp} % defines textquotesingle
    % Hack from http://tex.stackexchange.com/a/47451/13684:
    \AtBeginDocument{%
        \def\PYZsq{\textquotesingle}% Upright quotes in Pygmentized code
    }
    \usepackage{upquote} % Upright quotes for verbatim code
    \usepackage{eurosym} % defines \euro

    \usepackage{iftex}
    \ifPDFTeX
        \usepackage[T1]{fontenc}
        \IfFileExists{alphabeta.sty}{
              \usepackage{alphabeta}
          }{
              \usepackage[mathletters]{ucs}
              \usepackage[utf8x]{inputenc}
          }
    \else
        \usepackage{fontspec}
        \usepackage{unicode-math}
    \fi

    \usepackage{fancyvrb} % verbatim replacement that allows latex
    \usepackage{grffile} % extends the file name processing of package graphics
                         % to support a larger range
    \makeatletter % fix for old versions of grffile with XeLaTeX
    \@ifpackagelater{grffile}{2019/11/01}
    {
      % Do nothing on new versions
    }
    {
      \def\Gread@@xetex#1{%
        \IfFileExists{"\Gin@base".bb}%
        {\Gread@eps{\Gin@base.bb}}%
        {\Gread@@xetex@aux#1}%
      }
    }
    \makeatother
    \usepackage[Export]{adjustbox} % Used to constrain images to a maximum size
    \adjustboxset{max size={0.9\linewidth}{0.9\paperheight}}

    % The hyperref package gives us a pdf with properly built
    % internal navigation ('pdf bookmarks' for the table of contents,
    % internal cross-reference links, web links for URLs, etc.)
    \usepackage{hyperref}
    % The default LaTeX title has an obnoxious amount of whitespace. By default,
    % titling removes some of it. It also provides customization options.
    \usepackage{titling}
    \usepackage{longtable} % longtable support required by pandoc >1.10
    \usepackage{booktabs}  % table support for pandoc > 1.12.2
    \usepackage{array}     % table support for pandoc >= 2.11.3
    \usepackage{calc}      % table minipage width calculation for pandoc >= 2.11.1
    \usepackage[inline]{enumitem} % IRkernel/repr support (it uses the enumerate* environment)
    \usepackage[normalem]{ulem} % ulem is needed to support strikethroughs (\sout)
                                % normalem makes italics be italics, not underlines
    \usepackage{soul}      % strikethrough (\st) support for pandoc >= 3.0.0
    \usepackage{mathrsfs}
    

    
    % Colors for the hyperref package
    \definecolor{urlcolor}{rgb}{0,.145,.698}
    \definecolor{linkcolor}{rgb}{.71,0.21,0.01}
    \definecolor{citecolor}{rgb}{.12,.54,.11}

    % ANSI colors
    \definecolor{ansi-black}{HTML}{3E424D}
    \definecolor{ansi-black-intense}{HTML}{282C36}
    \definecolor{ansi-red}{HTML}{E75C58}
    \definecolor{ansi-red-intense}{HTML}{B22B31}
    \definecolor{ansi-green}{HTML}{00A250}
    \definecolor{ansi-green-intense}{HTML}{007427}
    \definecolor{ansi-yellow}{HTML}{DDB62B}
    \definecolor{ansi-yellow-intense}{HTML}{B27D12}
    \definecolor{ansi-blue}{HTML}{208FFB}
    \definecolor{ansi-blue-intense}{HTML}{0065CA}
    \definecolor{ansi-magenta}{HTML}{D160C4}
    \definecolor{ansi-magenta-intense}{HTML}{A03196}
    \definecolor{ansi-cyan}{HTML}{60C6C8}
    \definecolor{ansi-cyan-intense}{HTML}{258F8F}
    \definecolor{ansi-white}{HTML}{C5C1B4}
    \definecolor{ansi-white-intense}{HTML}{A1A6B2}
    \definecolor{ansi-default-inverse-fg}{HTML}{FFFFFF}
    \definecolor{ansi-default-inverse-bg}{HTML}{000000}

    % common color for the border for error outputs.
    \definecolor{outerrorbackground}{HTML}{FFDFDF}

    % commands and environments needed by pandoc snippets
    % extracted from the output of `pandoc -s`
    \providecommand{\tightlist}{%
      \setlength{\itemsep}{0pt}\setlength{\parskip}{0pt}}
    \DefineVerbatimEnvironment{Highlighting}{Verbatim}{commandchars=\\\{\}}
    % Add ',fontsize=\small' for more characters per line
    \newenvironment{Shaded}{}{}
    \newcommand{\KeywordTok}[1]{\textcolor[rgb]{0.00,0.44,0.13}{\textbf{{#1}}}}
    \newcommand{\DataTypeTok}[1]{\textcolor[rgb]{0.56,0.13,0.00}{{#1}}}
    \newcommand{\DecValTok}[1]{\textcolor[rgb]{0.25,0.63,0.44}{{#1}}}
    \newcommand{\BaseNTok}[1]{\textcolor[rgb]{0.25,0.63,0.44}{{#1}}}
    \newcommand{\FloatTok}[1]{\textcolor[rgb]{0.25,0.63,0.44}{{#1}}}
    \newcommand{\CharTok}[1]{\textcolor[rgb]{0.25,0.44,0.63}{{#1}}}
    \newcommand{\StringTok}[1]{\textcolor[rgb]{0.25,0.44,0.63}{{#1}}}
    \newcommand{\CommentTok}[1]{\textcolor[rgb]{0.38,0.63,0.69}{\textit{{#1}}}}
    \newcommand{\OtherTok}[1]{\textcolor[rgb]{0.00,0.44,0.13}{{#1}}}
    \newcommand{\AlertTok}[1]{\textcolor[rgb]{1.00,0.00,0.00}{\textbf{{#1}}}}
    \newcommand{\FunctionTok}[1]{\textcolor[rgb]{0.02,0.16,0.49}{{#1}}}
    \newcommand{\RegionMarkerTok}[1]{{#1}}
    \newcommand{\ErrorTok}[1]{\textcolor[rgb]{1.00,0.00,0.00}{\textbf{{#1}}}}
    \newcommand{\NormalTok}[1]{{#1}}

    % Additional commands for more recent versions of Pandoc
    \newcommand{\ConstantTok}[1]{\textcolor[rgb]{0.53,0.00,0.00}{{#1}}}
    \newcommand{\SpecialCharTok}[1]{\textcolor[rgb]{0.25,0.44,0.63}{{#1}}}
    \newcommand{\VerbatimStringTok}[1]{\textcolor[rgb]{0.25,0.44,0.63}{{#1}}}
    \newcommand{\SpecialStringTok}[1]{\textcolor[rgb]{0.73,0.40,0.53}{{#1}}}
    \newcommand{\ImportTok}[1]{{#1}}
    \newcommand{\DocumentationTok}[1]{\textcolor[rgb]{0.73,0.13,0.13}{\textit{{#1}}}}
    \newcommand{\AnnotationTok}[1]{\textcolor[rgb]{0.38,0.63,0.69}{\textbf{\textit{{#1}}}}}
    \newcommand{\CommentVarTok}[1]{\textcolor[rgb]{0.38,0.63,0.69}{\textbf{\textit{{#1}}}}}
    \newcommand{\VariableTok}[1]{\textcolor[rgb]{0.10,0.09,0.49}{{#1}}}
    \newcommand{\ControlFlowTok}[1]{\textcolor[rgb]{0.00,0.44,0.13}{\textbf{{#1}}}}
    \newcommand{\OperatorTok}[1]{\textcolor[rgb]{0.40,0.40,0.40}{{#1}}}
    \newcommand{\BuiltInTok}[1]{{#1}}
    \newcommand{\ExtensionTok}[1]{{#1}}
    \newcommand{\PreprocessorTok}[1]{\textcolor[rgb]{0.74,0.48,0.00}{{#1}}}
    \newcommand{\AttributeTok}[1]{\textcolor[rgb]{0.49,0.56,0.16}{{#1}}}
    \newcommand{\InformationTok}[1]{\textcolor[rgb]{0.38,0.63,0.69}{\textbf{\textit{{#1}}}}}
    \newcommand{\WarningTok}[1]{\textcolor[rgb]{0.38,0.63,0.69}{\textbf{\textit{{#1}}}}}


    % Define a nice break command that doesn't care if a line doesn't already
    % exist.
    \def\br{\hspace*{\fill} \\* }
    % Math Jax compatibility definitions
    \def\gt{>}
    \def\lt{<}
    \let\Oldtex\TeX
    \let\Oldlatex\LaTeX
    \renewcommand{\TeX}{\textrm{\Oldtex}}
    \renewcommand{\LaTeX}{\textrm{\Oldlatex}}
    % Document parameters
    % Document title
    \title{02-RehaussementVisualisationImages}
    
    
    
    
    
    
    
% Pygments definitions
\makeatletter
\def\PY@reset{\let\PY@it=\relax \let\PY@bf=\relax%
    \let\PY@ul=\relax \let\PY@tc=\relax%
    \let\PY@bc=\relax \let\PY@ff=\relax}
\def\PY@tok#1{\csname PY@tok@#1\endcsname}
\def\PY@toks#1+{\ifx\relax#1\empty\else%
    \PY@tok{#1}\expandafter\PY@toks\fi}
\def\PY@do#1{\PY@bc{\PY@tc{\PY@ul{%
    \PY@it{\PY@bf{\PY@ff{#1}}}}}}}
\def\PY#1#2{\PY@reset\PY@toks#1+\relax+\PY@do{#2}}

\@namedef{PY@tok@w}{\def\PY@tc##1{\textcolor[rgb]{0.73,0.73,0.73}{##1}}}
\@namedef{PY@tok@c}{\let\PY@it=\textit\def\PY@tc##1{\textcolor[rgb]{0.24,0.48,0.48}{##1}}}
\@namedef{PY@tok@cp}{\def\PY@tc##1{\textcolor[rgb]{0.61,0.40,0.00}{##1}}}
\@namedef{PY@tok@k}{\let\PY@bf=\textbf\def\PY@tc##1{\textcolor[rgb]{0.00,0.50,0.00}{##1}}}
\@namedef{PY@tok@kp}{\def\PY@tc##1{\textcolor[rgb]{0.00,0.50,0.00}{##1}}}
\@namedef{PY@tok@kt}{\def\PY@tc##1{\textcolor[rgb]{0.69,0.00,0.25}{##1}}}
\@namedef{PY@tok@o}{\def\PY@tc##1{\textcolor[rgb]{0.40,0.40,0.40}{##1}}}
\@namedef{PY@tok@ow}{\let\PY@bf=\textbf\def\PY@tc##1{\textcolor[rgb]{0.67,0.13,1.00}{##1}}}
\@namedef{PY@tok@nb}{\def\PY@tc##1{\textcolor[rgb]{0.00,0.50,0.00}{##1}}}
\@namedef{PY@tok@nf}{\def\PY@tc##1{\textcolor[rgb]{0.00,0.00,1.00}{##1}}}
\@namedef{PY@tok@nc}{\let\PY@bf=\textbf\def\PY@tc##1{\textcolor[rgb]{0.00,0.00,1.00}{##1}}}
\@namedef{PY@tok@nn}{\let\PY@bf=\textbf\def\PY@tc##1{\textcolor[rgb]{0.00,0.00,1.00}{##1}}}
\@namedef{PY@tok@ne}{\let\PY@bf=\textbf\def\PY@tc##1{\textcolor[rgb]{0.80,0.25,0.22}{##1}}}
\@namedef{PY@tok@nv}{\def\PY@tc##1{\textcolor[rgb]{0.10,0.09,0.49}{##1}}}
\@namedef{PY@tok@no}{\def\PY@tc##1{\textcolor[rgb]{0.53,0.00,0.00}{##1}}}
\@namedef{PY@tok@nl}{\def\PY@tc##1{\textcolor[rgb]{0.46,0.46,0.00}{##1}}}
\@namedef{PY@tok@ni}{\let\PY@bf=\textbf\def\PY@tc##1{\textcolor[rgb]{0.44,0.44,0.44}{##1}}}
\@namedef{PY@tok@na}{\def\PY@tc##1{\textcolor[rgb]{0.41,0.47,0.13}{##1}}}
\@namedef{PY@tok@nt}{\let\PY@bf=\textbf\def\PY@tc##1{\textcolor[rgb]{0.00,0.50,0.00}{##1}}}
\@namedef{PY@tok@nd}{\def\PY@tc##1{\textcolor[rgb]{0.67,0.13,1.00}{##1}}}
\@namedef{PY@tok@s}{\def\PY@tc##1{\textcolor[rgb]{0.73,0.13,0.13}{##1}}}
\@namedef{PY@tok@sd}{\let\PY@it=\textit\def\PY@tc##1{\textcolor[rgb]{0.73,0.13,0.13}{##1}}}
\@namedef{PY@tok@si}{\let\PY@bf=\textbf\def\PY@tc##1{\textcolor[rgb]{0.64,0.35,0.47}{##1}}}
\@namedef{PY@tok@se}{\let\PY@bf=\textbf\def\PY@tc##1{\textcolor[rgb]{0.67,0.36,0.12}{##1}}}
\@namedef{PY@tok@sr}{\def\PY@tc##1{\textcolor[rgb]{0.64,0.35,0.47}{##1}}}
\@namedef{PY@tok@ss}{\def\PY@tc##1{\textcolor[rgb]{0.10,0.09,0.49}{##1}}}
\@namedef{PY@tok@sx}{\def\PY@tc##1{\textcolor[rgb]{0.00,0.50,0.00}{##1}}}
\@namedef{PY@tok@m}{\def\PY@tc##1{\textcolor[rgb]{0.40,0.40,0.40}{##1}}}
\@namedef{PY@tok@gh}{\let\PY@bf=\textbf\def\PY@tc##1{\textcolor[rgb]{0.00,0.00,0.50}{##1}}}
\@namedef{PY@tok@gu}{\let\PY@bf=\textbf\def\PY@tc##1{\textcolor[rgb]{0.50,0.00,0.50}{##1}}}
\@namedef{PY@tok@gd}{\def\PY@tc##1{\textcolor[rgb]{0.63,0.00,0.00}{##1}}}
\@namedef{PY@tok@gi}{\def\PY@tc##1{\textcolor[rgb]{0.00,0.52,0.00}{##1}}}
\@namedef{PY@tok@gr}{\def\PY@tc##1{\textcolor[rgb]{0.89,0.00,0.00}{##1}}}
\@namedef{PY@tok@ge}{\let\PY@it=\textit}
\@namedef{PY@tok@gs}{\let\PY@bf=\textbf}
\@namedef{PY@tok@gp}{\let\PY@bf=\textbf\def\PY@tc##1{\textcolor[rgb]{0.00,0.00,0.50}{##1}}}
\@namedef{PY@tok@go}{\def\PY@tc##1{\textcolor[rgb]{0.44,0.44,0.44}{##1}}}
\@namedef{PY@tok@gt}{\def\PY@tc##1{\textcolor[rgb]{0.00,0.27,0.87}{##1}}}
\@namedef{PY@tok@err}{\def\PY@bc##1{{\setlength{\fboxsep}{\string -\fboxrule}\fcolorbox[rgb]{1.00,0.00,0.00}{1,1,1}{\strut ##1}}}}
\@namedef{PY@tok@kc}{\let\PY@bf=\textbf\def\PY@tc##1{\textcolor[rgb]{0.00,0.50,0.00}{##1}}}
\@namedef{PY@tok@kd}{\let\PY@bf=\textbf\def\PY@tc##1{\textcolor[rgb]{0.00,0.50,0.00}{##1}}}
\@namedef{PY@tok@kn}{\let\PY@bf=\textbf\def\PY@tc##1{\textcolor[rgb]{0.00,0.50,0.00}{##1}}}
\@namedef{PY@tok@kr}{\let\PY@bf=\textbf\def\PY@tc##1{\textcolor[rgb]{0.00,0.50,0.00}{##1}}}
\@namedef{PY@tok@bp}{\def\PY@tc##1{\textcolor[rgb]{0.00,0.50,0.00}{##1}}}
\@namedef{PY@tok@fm}{\def\PY@tc##1{\textcolor[rgb]{0.00,0.00,1.00}{##1}}}
\@namedef{PY@tok@vc}{\def\PY@tc##1{\textcolor[rgb]{0.10,0.09,0.49}{##1}}}
\@namedef{PY@tok@vg}{\def\PY@tc##1{\textcolor[rgb]{0.10,0.09,0.49}{##1}}}
\@namedef{PY@tok@vi}{\def\PY@tc##1{\textcolor[rgb]{0.10,0.09,0.49}{##1}}}
\@namedef{PY@tok@vm}{\def\PY@tc##1{\textcolor[rgb]{0.10,0.09,0.49}{##1}}}
\@namedef{PY@tok@sa}{\def\PY@tc##1{\textcolor[rgb]{0.73,0.13,0.13}{##1}}}
\@namedef{PY@tok@sb}{\def\PY@tc##1{\textcolor[rgb]{0.73,0.13,0.13}{##1}}}
\@namedef{PY@tok@sc}{\def\PY@tc##1{\textcolor[rgb]{0.73,0.13,0.13}{##1}}}
\@namedef{PY@tok@dl}{\def\PY@tc##1{\textcolor[rgb]{0.73,0.13,0.13}{##1}}}
\@namedef{PY@tok@s2}{\def\PY@tc##1{\textcolor[rgb]{0.73,0.13,0.13}{##1}}}
\@namedef{PY@tok@sh}{\def\PY@tc##1{\textcolor[rgb]{0.73,0.13,0.13}{##1}}}
\@namedef{PY@tok@s1}{\def\PY@tc##1{\textcolor[rgb]{0.73,0.13,0.13}{##1}}}
\@namedef{PY@tok@mb}{\def\PY@tc##1{\textcolor[rgb]{0.40,0.40,0.40}{##1}}}
\@namedef{PY@tok@mf}{\def\PY@tc##1{\textcolor[rgb]{0.40,0.40,0.40}{##1}}}
\@namedef{PY@tok@mh}{\def\PY@tc##1{\textcolor[rgb]{0.40,0.40,0.40}{##1}}}
\@namedef{PY@tok@mi}{\def\PY@tc##1{\textcolor[rgb]{0.40,0.40,0.40}{##1}}}
\@namedef{PY@tok@il}{\def\PY@tc##1{\textcolor[rgb]{0.40,0.40,0.40}{##1}}}
\@namedef{PY@tok@mo}{\def\PY@tc##1{\textcolor[rgb]{0.40,0.40,0.40}{##1}}}
\@namedef{PY@tok@ch}{\let\PY@it=\textit\def\PY@tc##1{\textcolor[rgb]{0.24,0.48,0.48}{##1}}}
\@namedef{PY@tok@cm}{\let\PY@it=\textit\def\PY@tc##1{\textcolor[rgb]{0.24,0.48,0.48}{##1}}}
\@namedef{PY@tok@cpf}{\let\PY@it=\textit\def\PY@tc##1{\textcolor[rgb]{0.24,0.48,0.48}{##1}}}
\@namedef{PY@tok@c1}{\let\PY@it=\textit\def\PY@tc##1{\textcolor[rgb]{0.24,0.48,0.48}{##1}}}
\@namedef{PY@tok@cs}{\let\PY@it=\textit\def\PY@tc##1{\textcolor[rgb]{0.24,0.48,0.48}{##1}}}

\def\PYZbs{\char`\\}
\def\PYZus{\char`\_}
\def\PYZob{\char`\{}
\def\PYZcb{\char`\}}
\def\PYZca{\char`\^}
\def\PYZam{\char`\&}
\def\PYZlt{\char`\<}
\def\PYZgt{\char`\>}
\def\PYZsh{\char`\#}
\def\PYZpc{\char`\%}
\def\PYZdl{\char`\$}
\def\PYZhy{\char`\-}
\def\PYZsq{\char`\'}
\def\PYZdq{\char`\"}
\def\PYZti{\char`\~}
% for compatibility with earlier versions
\def\PYZat{@}
\def\PYZlb{[}
\def\PYZrb{]}
\makeatother


    % For linebreaks inside Verbatim environment from package fancyvrb.
    \makeatletter
        \newbox\Wrappedcontinuationbox
        \newbox\Wrappedvisiblespacebox
        \newcommand*\Wrappedvisiblespace {\textcolor{red}{\textvisiblespace}}
        \newcommand*\Wrappedcontinuationsymbol {\textcolor{red}{\llap{\tiny$\m@th\hookrightarrow$}}}
        \newcommand*\Wrappedcontinuationindent {3ex }
        \newcommand*\Wrappedafterbreak {\kern\Wrappedcontinuationindent\copy\Wrappedcontinuationbox}
        % Take advantage of the already applied Pygments mark-up to insert
        % potential linebreaks for TeX processing.
        %        {, <, #, %, $, ' and ": go to next line.
        %        _, }, ^, &, >, - and ~: stay at end of broken line.
        % Use of \textquotesingle for straight quote.
        \newcommand*\Wrappedbreaksatspecials {%
            \def\PYGZus{\discretionary{\char`\_}{\Wrappedafterbreak}{\char`\_}}%
            \def\PYGZob{\discretionary{}{\Wrappedafterbreak\char`\{}{\char`\{}}%
            \def\PYGZcb{\discretionary{\char`\}}{\Wrappedafterbreak}{\char`\}}}%
            \def\PYGZca{\discretionary{\char`\^}{\Wrappedafterbreak}{\char`\^}}%
            \def\PYGZam{\discretionary{\char`\&}{\Wrappedafterbreak}{\char`\&}}%
            \def\PYGZlt{\discretionary{}{\Wrappedafterbreak\char`\<}{\char`\<}}%
            \def\PYGZgt{\discretionary{\char`\>}{\Wrappedafterbreak}{\char`\>}}%
            \def\PYGZsh{\discretionary{}{\Wrappedafterbreak\char`\#}{\char`\#}}%
            \def\PYGZpc{\discretionary{}{\Wrappedafterbreak\char`\%}{\char`\%}}%
            \def\PYGZdl{\discretionary{}{\Wrappedafterbreak\char`\$}{\char`\$}}%
            \def\PYGZhy{\discretionary{\char`\-}{\Wrappedafterbreak}{\char`\-}}%
            \def\PYGZsq{\discretionary{}{\Wrappedafterbreak\textquotesingle}{\textquotesingle}}%
            \def\PYGZdq{\discretionary{}{\Wrappedafterbreak\char`\"}{\char`\"}}%
            \def\PYGZti{\discretionary{\char`\~}{\Wrappedafterbreak}{\char`\~}}%
        }
        % Some characters . , ; ? ! / are not pygmentized.
        % This macro makes them "active" and they will insert potential linebreaks
        \newcommand*\Wrappedbreaksatpunct {%
            \lccode`\~`\.\lowercase{\def~}{\discretionary{\hbox{\char`\.}}{\Wrappedafterbreak}{\hbox{\char`\.}}}%
            \lccode`\~`\,\lowercase{\def~}{\discretionary{\hbox{\char`\,}}{\Wrappedafterbreak}{\hbox{\char`\,}}}%
            \lccode`\~`\;\lowercase{\def~}{\discretionary{\hbox{\char`\;}}{\Wrappedafterbreak}{\hbox{\char`\;}}}%
            \lccode`\~`\:\lowercase{\def~}{\discretionary{\hbox{\char`\:}}{\Wrappedafterbreak}{\hbox{\char`\:}}}%
            \lccode`\~`\?\lowercase{\def~}{\discretionary{\hbox{\char`\?}}{\Wrappedafterbreak}{\hbox{\char`\?}}}%
            \lccode`\~`\!\lowercase{\def~}{\discretionary{\hbox{\char`\!}}{\Wrappedafterbreak}{\hbox{\char`\!}}}%
            \lccode`\~`\/\lowercase{\def~}{\discretionary{\hbox{\char`\/}}{\Wrappedafterbreak}{\hbox{\char`\/}}}%
            \catcode`\.\active
            \catcode`\,\active
            \catcode`\;\active
            \catcode`\:\active
            \catcode`\?\active
            \catcode`\!\active
            \catcode`\/\active
            \lccode`\~`\~
        }
    \makeatother

    \let\OriginalVerbatim=\Verbatim
    \makeatletter
    \renewcommand{\Verbatim}[1][1]{%
        %\parskip\z@skip
        \sbox\Wrappedcontinuationbox {\Wrappedcontinuationsymbol}%
        \sbox\Wrappedvisiblespacebox {\FV@SetupFont\Wrappedvisiblespace}%
        \def\FancyVerbFormatLine ##1{\hsize\linewidth
            \vtop{\raggedright\hyphenpenalty\z@\exhyphenpenalty\z@
                \doublehyphendemerits\z@\finalhyphendemerits\z@
                \strut ##1\strut}%
        }%
        % If the linebreak is at a space, the latter will be displayed as visible
        % space at end of first line, and a continuation symbol starts next line.
        % Stretch/shrink are however usually zero for typewriter font.
        \def\FV@Space {%
            \nobreak\hskip\z@ plus\fontdimen3\font minus\fontdimen4\font
            \discretionary{\copy\Wrappedvisiblespacebox}{\Wrappedafterbreak}
            {\kern\fontdimen2\font}%
        }%

        % Allow breaks at special characters using \PYG... macros.
        \Wrappedbreaksatspecials
        % Breaks at punctuation characters . , ; ? ! and / need catcode=\active
        \OriginalVerbatim[#1,codes*=\Wrappedbreaksatpunct]%
    }
    \makeatother

    % Exact colors from NB
    \definecolor{incolor}{HTML}{303F9F}
    \definecolor{outcolor}{HTML}{D84315}
    \definecolor{cellborder}{HTML}{CFCFCF}
    \definecolor{cellbackground}{HTML}{F7F7F7}

    % prompt
    \makeatletter
    \newcommand{\boxspacing}{\kern\kvtcb@left@rule\kern\kvtcb@boxsep}
    \makeatother
    \newcommand{\prompt}[4]{
        {\ttfamily\llap{{\color{#2}[#3]:\hspace{3pt}#4}}\vspace{-\baselineskip}}
    }
    

    
    % Prevent overflowing lines due to hard-to-break entities
    \sloppy
    % Setup hyperref package
    \hypersetup{
      breaklinks=true,  % so long urls are correctly broken across lines
      colorlinks=true,
      urlcolor=urlcolor,
      linkcolor=linkcolor,
      citecolor=citecolor,
      }
    % Slightly bigger margins than the latex defaults
    
    \geometry{verbose,tmargin=1in,bmargin=1in,lmargin=1in,rmargin=1in}
    
    

\begin{document}
    
    \maketitle
    
    

    
    \hypertarget{sec-chap02}{%
\section{Réhaussement et visualisation d'images}\label{sec-chap02}}

Assurez-vous de lire ce préambule avant d'exécutez le reste du notebook.

\hypertarget{rocket-pruxe9ambule}{%
\subsection{:rocket: Préambule}\label{rocket-pruxe9ambule}}

\hypertarget{dart-objectifs}{%
\subsubsection{:dart: Objectifs}\label{dart-objectifs}}

Dans ce chapitre, nous abordons quelques techniques de réhaussement et
de visualisation d'images. Ce chapitre est aussi disponible sous la
forme d'un notebook Python:

\href{https://colab.research.google.com/github/sfoucher/TraitementImagesPythonVol1/blob/main/notebooks/02-RehaussementVisualisationImages.ipynb}{\includegraphics{images/colab-badge.svg}}

\hypertarget{librairies}{%
\subsubsection{Librairies}\label{librairies}}

Les librairies qui vont être explorées dans ce chapitre sont les
suivantes:

\begin{itemize}
\item
  \href{https://scipy.org/}{SciPy -}
\item
  \href{https://numpy.org/}{NumPy -}
\item
  \href{https://pypi.org/project/opencv-python/}{opencv-python · PyPI}
\item
  \href{https://scikit-image.org/}{scikit-image}
\item
  \href{https://rasterio.readthedocs.io/en/stable/}{Rasterio}
\item
  \href{https://docs.xarray.dev/en/stable/}{Xarray}
\item
  \href{https://corteva.github.io/rioxarray/stable/index.html}{rioxarray}
\end{itemize}

Dans l'environnement Google Colab, seul \texttt{rioxarray} et GDAL
doivent être installés:

    \begin{tcolorbox}[breakable, size=fbox, boxrule=1pt, pad at break*=1mm,colback=cellbackground, colframe=cellborder]
\prompt{In}{incolor}{1}{\boxspacing}
\begin{Verbatim}[commandchars=\\\{\}]
\PY{o}{\PYZpc{}\PYZpc{}capture}
\PY{err}{!}\PY{n}{apt}\PY{o}{\PYZhy{}}\PY{n}{get} \PY{n}{update}
\PY{err}{!}\PY{n}{apt}\PY{o}{\PYZhy{}}\PY{n}{get} \PY{n}{install} \PY{n}{gdal}\PY{o}{\PYZhy{}}\PY{n+nb}{bin} \PY{n}{libgdal}\PY{o}{\PYZhy{}}\PY{n}{dev}
\PY{err}{!}\PY{n}{pip} \PY{n}{install} \PY{o}{\PYZhy{}}\PY{n}{q} \PY{n}{rioxarray}
\PY{err}{!}\PY{n}{pip} \PY{n}{install} \PY{o}{\PYZhy{}}\PY{n}{qU} \PY{l+s+s2}{\PYZdq{}}\PY{l+s+s2}{geemap[workshop]}\PY{l+s+s2}{\PYZdq{}}
\end{Verbatim}
\end{tcolorbox}

    Vérifier les importations:

    \begin{tcolorbox}[breakable, size=fbox, boxrule=1pt, pad at break*=1mm,colback=cellbackground, colframe=cellborder]
\prompt{In}{incolor}{2}{\boxspacing}
\begin{Verbatim}[commandchars=\\\{\}]
\PY{c+c1}{\PYZsh{}| eval: true}
\PY{k+kn}{import} \PY{n+nn}{numpy} \PY{k}{as} \PY{n+nn}{np}
\PY{k+kn}{import} \PY{n+nn}{rioxarray} \PY{k}{as} \PY{n+nn}{rxr}
\PY{k+kn}{from} \PY{n+nn}{scipy} \PY{k+kn}{import} \PY{n}{signal}
\PY{k+kn}{import} \PY{n+nn}{xarray} \PY{k}{as} \PY{n+nn}{xr}
\PY{k+kn}{import} \PY{n+nn}{xrscipy}
\PY{k+kn}{import} \PY{n+nn}{matplotlib}\PY{n+nn}{.}\PY{n+nn}{pyplot} \PY{k}{as} \PY{n+nn}{plt}
\end{Verbatim}
\end{tcolorbox}

    \hypertarget{donnuxe9es}{%
\subsubsection{Données}\label{donnuxe9es}}

Nous allons utilisez les images suivantes dans ce chapitre:

    \begin{tcolorbox}[breakable, size=fbox, boxrule=1pt, pad at break*=1mm,colback=cellbackground, colframe=cellborder]
\prompt{In}{incolor}{3}{\boxspacing}
\begin{Verbatim}[commandchars=\\\{\}]
\PY{o}{\PYZpc{}\PYZpc{}capture}
\PY{err}{!}\PY{n}{wget} \PY{n}{https}\PY{p}{:}\PY{o}{/}\PY{o}{/}\PY{n}{github}\PY{o}{.}\PY{n}{com}\PY{o}{/}\PY{n}{sfoucher}\PY{o}{/}\PY{n}{TraitementImagesPythonVol1}\PY{o}{/}\PY{n}{raw}\PY{o}{/}\PY{n}{refs}\PY{o}{/}\PY{n}{heads}\PY{o}{/}\PY{n}{main}\PY{o}{/}\PY{n}{data}\PY{o}{/}\PY{n}{chapitre01}\PY{o}{/}\PY{n}{subset\PYZus{}RGBNIR\PYZus{}of\PYZus{}S2A\PYZus{}MSIL2A\PYZus{}20240625T153941\PYZus{}N0510\PYZus{}R011\PYZus{}T18TYR\PYZus{}20240625T221903}\PY{o}{.}\PY{n}{tif} \PY{o}{\PYZhy{}}\PY{n}{O} \PY{n}{RGBNIR\PYZus{}of\PYZus{}S2A}\PY{o}{.}\PY{n}{tif}
\PY{err}{!}\PY{n}{wget} \PY{n}{https}\PY{p}{:}\PY{o}{/}\PY{o}{/}\PY{n}{github}\PY{o}{.}\PY{n}{com}\PY{o}{/}\PY{n}{sfoucher}\PY{o}{/}\PY{n}{opengeos}\PY{o}{\PYZhy{}}\PY{n}{data}\PY{o}{/}\PY{n}{raw}\PY{o}{/}\PY{n}{refs}\PY{o}{/}\PY{n}{heads}\PY{o}{/}\PY{n}{main}\PY{o}{/}\PY{n}{raster}\PY{o}{/}\PY{n}{landsat7}\PY{o}{.}\PY{n}{tif} \PY{o}{\PYZhy{}}\PY{n}{O} \PY{n}{landsat7}\PY{o}{.}\PY{n}{tif}
\PY{err}{!}\PY{n}{wget} \PY{n}{https}\PY{p}{:}\PY{o}{/}\PY{o}{/}\PY{n}{github}\PY{o}{.}\PY{n}{com}\PY{o}{/}\PY{n}{sfoucher}\PY{o}{/}\PY{n}{opengeos}\PY{o}{\PYZhy{}}\PY{n}{data}\PY{o}{/}\PY{n}{raw}\PY{o}{/}\PY{n}{refs}\PY{o}{/}\PY{n}{heads}\PY{o}{/}\PY{n}{main}\PY{o}{/}\PY{n}{images}\PY{o}{/}\PY{n}{berkeley}\PY{o}{.}\PY{n}{jpg} \PY{o}{\PYZhy{}}\PY{n}{O} \PY{n}{berkeley}\PY{o}{.}\PY{n}{jpg}
\PY{err}{!}\PY{n}{wget} \PY{n}{https}\PY{p}{:}\PY{o}{/}\PY{o}{/}\PY{n}{github}\PY{o}{.}\PY{n}{com}\PY{o}{/}\PY{n}{sfoucher}\PY{o}{/}\PY{n}{TraitementImagesPythonVol1}\PY{o}{/}\PY{n}{raw}\PY{o}{/}\PY{n}{refs}\PY{o}{/}\PY{n}{heads}\PY{o}{/}\PY{n}{main}\PY{o}{/}\PY{n}{data}\PY{o}{/}\PY{n}{chapitre01}\PY{o}{/}\PY{n}{subset\PYZus{}0\PYZus{}of\PYZus{}S1A\PYZus{}split\PYZus{}NR\PYZus{}Cal\PYZus{}Deb\PYZus{}ML\PYZus{}Spk\PYZus{}SRGR}\PY{o}{.}\PY{n}{tif} \PY{o}{\PYZhy{}}\PY{n}{O} \PY{n}{SAR}\PY{o}{.}\PY{n}{tif}
\end{Verbatim}
\end{tcolorbox}

    Vérifiez que vous êtes capable de les lire :

    \begin{tcolorbox}[breakable, size=fbox, boxrule=1pt, pad at break*=1mm,colback=cellbackground, colframe=cellborder]
\prompt{In}{incolor}{4}{\boxspacing}
\begin{Verbatim}[commandchars=\\\{\}]
\PY{c+c1}{\PYZsh{}| eval: true}
\PY{c+c1}{\PYZsh{}| output: false}

\PY{k}{with} \PY{n}{rxr}\PY{o}{.}\PY{n}{open\PYZus{}rasterio}\PY{p}{(}\PY{l+s+s1}{\PYZsq{}}\PY{l+s+s1}{berkeley.jpg}\PY{l+s+s1}{\PYZsq{}}\PY{p}{,} \PY{n}{mask\PYZus{}and\PYZus{}scale}\PY{o}{=} \PY{k+kc}{True}\PY{p}{)} \PY{k}{as} \PY{n}{img\PYZus{}rgb}\PY{p}{:}
    \PY{n+nb}{print}\PY{p}{(}\PY{n}{img\PYZus{}rgb}\PY{p}{)}
\PY{k}{with} \PY{n}{rxr}\PY{o}{.}\PY{n}{open\PYZus{}rasterio}\PY{p}{(}\PY{l+s+s1}{\PYZsq{}}\PY{l+s+s1}{RGBNIR\PYZus{}of\PYZus{}S2A.tif}\PY{l+s+s1}{\PYZsq{}}\PY{p}{,} \PY{n}{mask\PYZus{}and\PYZus{}scale}\PY{o}{=} \PY{k+kc}{True}\PY{p}{)} \PY{k}{as} \PY{n}{img\PYZus{}rgbnir}\PY{p}{:}
    \PY{n+nb}{print}\PY{p}{(}\PY{n}{img\PYZus{}rgbnir}\PY{p}{)}
\PY{k}{with} \PY{n}{rxr}\PY{o}{.}\PY{n}{open\PYZus{}rasterio}\PY{p}{(}\PY{l+s+s1}{\PYZsq{}}\PY{l+s+s1}{SAR.tif}\PY{l+s+s1}{\PYZsq{}}\PY{p}{,} \PY{n}{mask\PYZus{}and\PYZus{}scale}\PY{o}{=} \PY{k+kc}{True}\PY{p}{)} \PY{k}{as} \PY{n}{img\PYZus{}SAR}\PY{p}{:}
    \PY{n+nb}{print}\PY{p}{(}\PY{n}{img\PYZus{}SAR}\PY{p}{)}
\end{Verbatim}
\end{tcolorbox}

    \begin{Verbatim}[commandchars=\\\{\}]
<xarray.DataArray (band: 3, y: 771, x: 1311)> Size: 12MB
[3032343 values with dtype=float32]
Coordinates:
  * band         (band) int64 24B 1 2 3
  * x            (x) float64 10kB 0.5 1.5 2.5 {\ldots} 1.308e+03 1.31e+03 1.31e+03
  * y            (y) float64 6kB 0.5 1.5 2.5 3.5 4.5 {\ldots} 767.5 768.5 769.5 770.5
    spatial\_ref  int64 8B 0
<xarray.DataArray (band: 4, y: 1926, x: 2074)> Size: 64MB
[15978096 values with dtype=float32]
Coordinates:
  * band         (band) int64 32B 1 2 3 4
  * x            (x) float64 17kB 7.318e+05 7.318e+05 {\ldots} 7.525e+05 7.525e+05
  * y            (y) float64 15kB 5.041e+06 5.041e+06 {\ldots} 5.022e+06 5.022e+06
    spatial\_ref  int64 8B 0
Attributes:
    TIFFTAG\_IMAGEDESCRIPTION:  subset\_RGBNIR\_of\_S2A\_MSIL2A\_20240625T153941\_N0{\ldots}
    TIFFTAG\_XRESOLUTION:       1
    TIFFTAG\_YRESOLUTION:       1
    TIFFTAG\_RESOLUTIONUNIT:    1 (unitless)
    AREA\_OR\_POINT:             Area
    STATISTICS\_MAXIMUM:        15104
    STATISTICS\_MEAN:           1426.6252674912
    STATISTICS\_MINIMUM:        86
    STATISTICS\_STDDEV:         306.56427126942
    STATISTICS\_VALID\_PERCENT:  100
<xarray.DataArray (band: 2, y: 1188, x: 1599)> Size: 15MB
[3799224 values with dtype=float32]
Coordinates:
  * band         (band) int64 16B 1 2
  * x            (x) float64 13kB -73.98 -73.98 -73.98 {\ldots} -73.51 -73.51 -73.51
  * y            (y) float64 10kB 45.27 45.27 45.27 45.27 {\ldots} 44.93 44.93 44.93
    spatial\_ref  int64 8B 0
Attributes:
    TIFFTAG\_XRESOLUTION:     1
    TIFFTAG\_YRESOLUTION:     1
    TIFFTAG\_RESOLUTIONUNIT:  1 (unitless)
    AREA\_OR\_POINT:           Area
    \end{Verbatim}

    \begin{Verbatim}[commandchars=\\\{\}]
/home/sfoucher/miniconda3/lib/python3.10/site-packages/rioxarray/\_io.py:1143:
NotGeoreferencedWarning: Dataset has no geotransform, gcps, or rpcs. The
identity matrix will be returned.
  warnings.warn(str(rio\_warning.message), type(rio\_warning.message))  \# type:
ignore
Warning 1: TIFFReadDirectory:Sum of Photometric type-related color channels and
ExtraSamples doesn't match SamplesPerPixel. Defining non-color channels as
ExtraSamples.
Warning 1: TIFFReadDirectory:Sum of Photometric type-related color channels and
ExtraSamples doesn't match SamplesPerPixel. Defining non-color channels as
ExtraSamples.
    \end{Verbatim}

    \hypertarget{ruxe9haussements-visuels}{%
\subsection{Réhaussements visuels}\label{ruxe9haussements-visuels}}

Le but du réhaussement visuel d'une image vise principalement à
améliorer la qualité visuelle d'une image en améliorant le contraste, la
dynamique ou la texture d'une image. De manière générale, ce
réhaussement ne modifie pas la donnée d'origine mais est plutôt
appliquée dynamiquement à l'affichage pour des fins d'inspection
visuelle.

\hypertarget{statistiques-dune-image}{%
\subsubsection{Statistiques d'une image}\label{statistiques-dune-image}}

On peut considérer un ensemble de statistique globales pour chacune des
bandes d'une image: - valeurs minimales et maximales - valeurs moyennes,
médianes et quantiles - écart-types, skewness et kurtosis Ces
statistiques doivent être calculées pour chaque bande d'une image
multispectrale.

En ligne de commande, \texttt{gdalinfo} permet d'interroger rapidement
un fichier image pour connaitre les statistiques de base:

    \begin{tcolorbox}[breakable, size=fbox, boxrule=1pt, pad at break*=1mm,colback=cellbackground, colframe=cellborder]
\prompt{In}{incolor}{5}{\boxspacing}
\begin{Verbatim}[commandchars=\\\{\}]
\PY{c+c1}{\PYZsh{}| lst\PYZhy{}label: lst\PYZhy{}gdalstats}
\PY{c+c1}{\PYZsh{}| lst\PYZhy{}cap: Statistiques d\PYZsq{}une image avec gdal}
\PY{c+c1}{\PYZsh{}| eval: true}

\PY{o}{!}gdalinfo\PY{+w}{ }\PYZhy{}stats\PY{+w}{ }landsat7.tif
\end{Verbatim}
\end{tcolorbox}

    \begin{Verbatim}[commandchars=\\\{\}]
Driver: GTiff/GeoTIFF
Files: landsat7.tif
       landsat7.tif.aux.xml
Size is 2181, 1917
Coordinate System is:
PROJCS["WGS 84 / Pseudo-Mercator",
    GEOGCS["WGS 84",
        DATUM["WGS\_1984",
            SPHEROID["WGS 84",6378137,298.257223563,
                AUTHORITY["EPSG","7030"]],
            AUTHORITY["EPSG","6326"]],
        PRIMEM["Greenwich",0,
            AUTHORITY["EPSG","8901"]],
        UNIT["degree",0.0174532925199433,
            AUTHORITY["EPSG","9122"]],
        AUTHORITY["EPSG","4326"]],
    PROJECTION["Mercator\_1SP"],
    PARAMETER["central\_meridian",0],
    PARAMETER["scale\_factor",1],
    PARAMETER["false\_easting",0],
    PARAMETER["false\_northing",0],
    UNIT["metre",1,
        AUTHORITY["EPSG","9001"]],
    AXIS["X",EAST],
    AXIS["Y",NORTH],
    EXTENSION["PROJ4","+proj=merc +a=6378137 +b=6378137 +lat\_ts=0.0 +lon\_0=0.0
+x\_0=0.0 +y\_0=0 +k=1.0 +units=m +nadgrids=@null +wktext +no\_defs"],
    AUTHORITY["EPSG","3857"]]
Origin = (-13651650.000000000000000,4576290.000000000000000)
Pixel Size = (30.000000000000000,-30.000000000000000)
Metadata:
  AREA\_OR\_POINT=Area
  OVR\_RESAMPLING\_ALG=NEAREST
  TIFFTAG\_RESOLUTIONUNIT=1 (unitless)
  TIFFTAG\_XRESOLUTION=1
  TIFFTAG\_YRESOLUTION=1
Image Structure Metadata:
  COMPRESSION=DEFLATE
  INTERLEAVE=PIXEL
Corner Coordinates:
Upper Left  (-13651650.000, 4576290.000) (122d38' 5.49"W, 37d58'40.08"N)
Lower Left  (-13651650.000, 4518780.000) (122d38' 5.49"W, 37d34'10.00"N)
Upper Right (-13586220.000, 4576290.000) (122d 2'49.53"W, 37d58'40.08"N)
Lower Right (-13586220.000, 4518780.000) (122d 2'49.53"W, 37d34'10.00"N)
Center      (-13618935.000, 4547535.000) (122d20'27.51"W, 37d46'26.05"N)
Band 1 Block=512x512 Type=Byte, ColorInterp=Red
  Min=19.000 Max=233.000
  Minimum=19.000, Maximum=233.000, Mean=98.433, StdDev=21.164
  NoData Value=0
  Overviews: 1091x959, 546x480
  Metadata:
    STATISTICS\_MAXIMUM=233
    STATISTICS\_MEAN=98.433096940153
    STATISTICS\_MINIMUM=19
    STATISTICS\_STDDEV=21.164021026458
Band 2 Block=512x512 Type=Byte, ColorInterp=Green
  Min=19.000 Max=178.000
  Minimum=19.000, Maximum=178.000, Mean=55.068, StdDev=22.204
  NoData Value=0
  Overviews: 1091x959, 546x480
  Metadata:
    STATISTICS\_MAXIMUM=178
    STATISTICS\_MEAN=55.067787534804
    STATISTICS\_MINIMUM=19
    STATISTICS\_STDDEV=22.203571974581
Band 3 Block=512x512 Type=Byte, ColorInterp=Blue
  Min=19.000 Max=187.000
  Minimum=19.000, Maximum=187.000, Mean=43.341, StdDev=20.330
  NoData Value=0
  Overviews: 1091x959, 546x480
  Metadata:
    STATISTICS\_MAXIMUM=187
    STATISTICS\_MEAN=43.340507443056
    STATISTICS\_MINIMUM=19
    STATISTICS\_STDDEV=20.32987736339
    \end{Verbatim}

    Les librairies de base comme \texttt{xarray} et \texttt{numpy} peuvent
facilement produire des statistiques comme avec la fonction
\href{https://rasterio.readthedocs.io/en/stable/api/rasterio.io.html\#rasterio.io.BufferedDatasetWriter.stats}{stats}:

    \begin{tcolorbox}[breakable, size=fbox, boxrule=1pt, pad at break*=1mm,colback=cellbackground, colframe=cellborder]
\prompt{In}{incolor}{6}{\boxspacing}
\begin{Verbatim}[commandchars=\\\{\}]
\PY{c+c1}{\PYZsh{}| eval: false}

\PY{k+kn}{import} \PY{n+nn}{rasterio} \PY{k}{as} \PY{n+nn}{rio}
\PY{k+kn}{import} \PY{n+nn}{numpy} \PY{k}{as} \PY{n+nn}{np}
\PY{k}{with} \PY{n}{rio}\PY{o}{.}\PY{n}{open}\PY{p}{(}\PY{l+s+s1}{\PYZsq{}}\PY{l+s+s1}{landsat7.tif}\PY{l+s+s1}{\PYZsq{}}\PY{p}{)} \PY{k}{as} \PY{n}{src}\PY{p}{:}
    \PY{n}{stats}\PY{o}{=} \PY{n}{src}\PY{o}{.}\PY{n}{stats}\PY{p}{(}\PY{p}{)}
    \PY{n+nb}{print}\PY{p}{(}\PY{n}{stats}\PY{p}{)}
\end{Verbatim}
\end{tcolorbox}

    \begin{Verbatim}[commandchars=\\\{\}]
[Statistics(min=19.0, max=233.0, mean=98.433096940153, std=21.164021026458),
Statistics(min=19.0, max=178.0, mean=55.067787534804, std=22.203571974581),
Statistics(min=19.0, max=187.0, mean=43.340507443056, std=20.32987736339)]
    \end{Verbatim}

    La librairie \texttt{xarray} donne accès à des fonctionnalités plus
sophistiquées comme le calcul des quantiles:

    \begin{tcolorbox}[breakable, size=fbox, boxrule=1pt, pad at break*=1mm,colback=cellbackground, colframe=cellborder]
\prompt{In}{incolor}{7}{\boxspacing}
\begin{Verbatim}[commandchars=\\\{\}]
\PY{c+c1}{\PYZsh{}| eval: true}

\PY{k+kn}{import} \PY{n+nn}{rioxarray} \PY{k}{as} \PY{n+nn}{riox}
\PY{k}{with} \PY{n}{riox}\PY{o}{.}\PY{n}{open\PYZus{}rasterio}\PY{p}{(}\PY{l+s+s1}{\PYZsq{}}\PY{l+s+s1}{landsat7.tif}\PY{l+s+s1}{\PYZsq{}}\PY{p}{,} \PY{n}{masked}\PY{o}{=} \PY{k+kc}{True}\PY{p}{)} \PY{k}{as} \PY{n}{src}\PY{p}{:}
    \PY{n+nb}{print}\PY{p}{(}\PY{n}{src}\PY{p}{)}
\PY{n}{quantiles} \PY{o}{=} \PY{n}{src}\PY{o}{.}\PY{n}{quantile}\PY{p}{(}\PY{n}{dim}\PY{o}{=}\PY{p}{[}\PY{l+s+s1}{\PYZsq{}}\PY{l+s+s1}{x}\PY{l+s+s1}{\PYZsq{}}\PY{p}{,}\PY{l+s+s1}{\PYZsq{}}\PY{l+s+s1}{y}\PY{l+s+s1}{\PYZsq{}}\PY{p}{]}\PY{p}{,} \PY{n}{q}\PY{o}{=}\PY{p}{[}\PY{l+m+mf}{.025}\PY{p}{,}\PY{l+m+mf}{.25}\PY{p}{,}\PY{l+m+mf}{.5}\PY{p}{,}\PY{l+m+mf}{.75}\PY{p}{,}\PY{l+m+mf}{.975}\PY{p}{]}\PY{p}{)}
\PY{n}{quantiles}
\end{Verbatim}
\end{tcolorbox}

    \begin{Verbatim}[commandchars=\\\{\}]
<xarray.DataArray (band: 3, y: 1917, x: 2181)> Size: 50MB
[12542931 values with dtype=float32]
Coordinates:
  * band         (band) int64 24B 1 2 3
  * x            (x) float64 17kB -1.365e+07 -1.365e+07 {\ldots} -1.359e+07
  * y            (y) float64 15kB 4.576e+06 4.576e+06 {\ldots} 4.519e+06 4.519e+06
    spatial\_ref  int64 8B 0
Attributes:
    AREA\_OR\_POINT:           Area
    OVR\_RESAMPLING\_ALG:      NEAREST
    TIFFTAG\_RESOLUTIONUNIT:  1 (unitless)
    TIFFTAG\_XRESOLUTION:     1
    TIFFTAG\_YRESOLUTION:     1
    STATISTICS\_MAXIMUM:      233
    STATISTICS\_MEAN:         98.433096940153
    STATISTICS\_MINIMUM:      19
    STATISTICS\_STDDEV:       21.164021026458
    scale\_factor:            1.0
    add\_offset:              0.0
    \end{Verbatim}

            \begin{tcolorbox}[breakable, size=fbox, boxrule=.5pt, pad at break*=1mm, opacityfill=0]
\prompt{Out}{outcolor}{7}{\boxspacing}
\begin{Verbatim}[commandchars=\\\{\}]
<xarray.DataArray (quantile: 5, band: 3)> Size: 120B
array([[ 54.,  19.,  19.],
       [ 85.,  38.,  27.],
       [ 99.,  54.,  38.],
       [111.,  69.,  57.],
       [140., 102.,  89.]])
Coordinates:
  * band      (band) int64 24B 1 2 3
  * quantile  (quantile) float64 40B 0.025 0.25 0.5 0.75 0.975
\end{Verbatim}
\end{tcolorbox}
        
    \hypertarget{calcul-de-lhistogramme}{%
\paragraph{Calcul de l'histogramme}\label{calcul-de-lhistogramme}}

Le calcul d'un histogramme pour une image (une bande) permet d'avoir une
vue plus détaillée de la répartition des valeurs radiométriques. Le
calcul d'un histogramme nécessite minimalement de faire le choix d'une
valeur du nombre de \emph{bins} (ou de la largeur). Un \emph{bin} est un
intervalle de valeurs pour lequel on peut calculer le nombre de valeurs
observées dans l'image. La fonction de base pour ce type de calcul est
la fonction \texttt{numpy.histogram()}:

    \begin{tcolorbox}[breakable, size=fbox, boxrule=1pt, pad at break*=1mm,colback=cellbackground, colframe=cellborder]
\prompt{In}{incolor}{8}{\boxspacing}
\begin{Verbatim}[commandchars=\\\{\}]
\PY{c+c1}{\PYZsh{}| eval: true}
\PY{k+kn}{import} \PY{n+nn}{numpy} \PY{k}{as} \PY{n+nn}{np}
\PY{n}{array} \PY{o}{=} \PY{n}{np}\PY{o}{.}\PY{n}{random}\PY{o}{.}\PY{n}{randint}\PY{p}{(}\PY{l+m+mi}{0}\PY{p}{,}\PY{l+m+mi}{10}\PY{p}{,}\PY{l+m+mi}{100}\PY{p}{)} \PY{c+c1}{\PYZsh{} 100 valeurs aléatoires entre 0 et 10}
\PY{n}{hist}\PY{p}{,} \PY{n}{bin\PYZus{}limites} \PY{o}{=} \PY{n}{np}\PY{o}{.}\PY{n}{histogram}\PY{p}{(}\PY{n}{array}\PY{p}{,} \PY{n}{density}\PY{o}{=}\PY{k+kc}{True}\PY{p}{)}
\PY{n+nb}{print}\PY{p}{(}\PY{l+s+s1}{\PYZsq{}}\PY{l+s+s1}{valeurs :}\PY{l+s+s1}{\PYZsq{}}\PY{p}{,}\PY{n}{hist}\PY{p}{)}
\PY{n+nb}{print}\PY{p}{(}\PY{l+s+s1}{\PYZsq{}}\PY{l+s+s1}{;imites :}\PY{l+s+s1}{\PYZsq{}}\PY{p}{,}\PY{n}{bin\PYZus{}limites}\PY{p}{)}
\end{Verbatim}
\end{tcolorbox}

    \begin{Verbatim}[commandchars=\\\{\}]
valeurs : [0.15555556 0.05555556 0.07777778 0.13333333 0.11111111 0.12222222
 0.1        0.07777778 0.12222222 0.15555556]
;imites : [0.  0.9 1.8 2.7 3.6 4.5 5.4 6.3 7.2 8.1 9. ]
    \end{Verbatim}

    Le calcul se fait avec 10 intervalles par défaut.

Pour des besoins de visualisation, le calcul des valeurs extrêmes de
l'histogramme peut aussi se faire via les quantiles comme discutés
auparavant.

\hypertarget{visualisation-des-histogrammes}{%
\subparagraph{Visualisation des
histogrammes}\label{visualisation-des-histogrammes}}

La librarie \texttt{rasterio} est probablement l'outil le plus simples
pour visualiser rapidement des histogrammes sur une image
multi-spectrale:

    \begin{tcolorbox}[breakable, size=fbox, boxrule=1pt, pad at break*=1mm,colback=cellbackground, colframe=cellborder]
\prompt{In}{incolor}{9}{\boxspacing}
\begin{Verbatim}[commandchars=\\\{\}]
\PY{c+c1}{\PYZsh{}| eval: true}
\PY{k+kn}{import} \PY{n+nn}{rasterio} \PY{k}{as} \PY{n+nn}{rio}
\PY{k+kn}{from} \PY{n+nn}{rasterio}\PY{n+nn}{.}\PY{n+nn}{plot} \PY{k+kn}{import} \PY{n}{show\PYZus{}hist}
\PY{k}{with} \PY{n}{rio}\PY{o}{.}\PY{n}{open}\PY{p}{(}\PY{l+s+s1}{\PYZsq{}}\PY{l+s+s1}{RGBNIR\PYZus{}of\PYZus{}S2A.tif}\PY{l+s+s1}{\PYZsq{}}\PY{p}{)} \PY{k}{as} \PY{n}{src}\PY{p}{:}
  \PY{n}{show\PYZus{}hist}\PY{p}{(}\PY{n}{src}\PY{p}{,} \PY{n}{bins}\PY{o}{=}\PY{l+m+mi}{50}\PY{p}{,} \PY{n}{lw}\PY{o}{=}\PY{l+m+mf}{0.0}\PY{p}{,} \PY{n}{stacked}\PY{o}{=}\PY{k+kc}{False}\PY{p}{,} \PY{n}{alpha}\PY{o}{=}\PY{l+m+mf}{0.3}\PY{p}{,}\PY{n}{histtype}\PY{o}{=}\PY{l+s+s1}{\PYZsq{}}\PY{l+s+s1}{stepfilled}\PY{l+s+s1}{\PYZsq{}}\PY{p}{,} \PY{n}{title}\PY{o}{=}\PY{l+s+s2}{\PYZdq{}}\PY{l+s+s2}{Histogram}\PY{l+s+s2}{\PYZdq{}}\PY{p}{)}
\end{Verbatim}
\end{tcolorbox}

    \begin{center}
    \adjustimage{max size={0.9\linewidth}{0.9\paperheight}}{02-RehaussementVisualisationImages_files/02-RehaussementVisualisationImages_17_0.png}
    \end{center}
    { \hspace*{\fill} \\}
    
    \hypertarget{ruxe9haussements-linuxe9aires}{%
\subsubsection{Réhaussements
linéaires}\label{ruxe9haussements-linuxe9aires}}

Le réhaussement linéaire d'une image est la forme la plus simple de
réhaussement, elle consiste 1) à optimiser les valeurs des pixels d'une
image afin de maximiser la dynamique disponibles à l'affichage, ou 2)
changer le format de stockage des valeurs (e.g.~de 8 bit à 16 bit):

\[ \text{nouvelle valeur d'un pixel} = \frac{\text{valeur d'un pixel} - min_0}{max_0 - min_0}\times (max_1 - min_1)+min_1\]\{\#eq-rehauss-lin\}

Par cette opération, on passe de la dynamique de départ
(\(max_0 - min_0\)) vers la dynamique cible (\(max_1 - min_1\)). Bien
que cette opération semble triviale, il est important d'être conscient
des trois contraintes suivantes: 1. \textbf{Faire attention à la
dynamique cible}, ainsi, pour sauvegarder une image en format 8 bit, on
utilisera alors \(max_1=255\) et \(min_1=0\). 2. \textbf{Préservation de
la valeur de no data} : il faut faire attention à la valeur \(min_1\)
dans le cas d'une valeur présente pour \emph{no\_data}. Par exemple, si
\emph{no\_data=0} alors il faut s'assurer que \(min_1>0\). 3.
\textbf{Précision du calcul} : si possible réaliser la division
ci-dessus en format \emph{float}

\hypertarget{ruxe9haussements-non-linuxe9aires}{%
\subsubsection{Réhaussements non
linéaires}\label{ruxe9haussements-non-linuxe9aires}}

Calcul d'histogrammes, étirement, égalisation, styling

\hypertarget{composuxe9s-couleurs}{%
\subsubsection{Composés couleurs}\label{composuxe9s-couleurs}}

Le système visuel humain est sensible seulement à la partie visible du
spectre électromagnétique qui compose les couleurs de l'arc-en-ciel du
bleu au rouge. L'ensemble des couleurs du spectre visible peut être
obtenu à partir du mélange de trois couleurs primaires (rouge, vert et
bleu). Ce système de décomposition à trois couleurs est à la base de la
plupart des systèmes de visualisation ou de représentation de
l'information de couleur. On peut trouver des variantes comme le système
HSV (\emph{Hue-Saturation-Value}) utilisé en encodage de données vidéos.

\hypertarget{visualisation}{%
\subsection{Visualisation}\label{visualisation}}

\hypertarget{visualisation-en-python}{%
\subsubsection{Visualisation en Python}\label{visualisation-en-python}}

Il faut d'entrée mentionner que Python n'est pas vraiment fait pour
visualiser de la donnée de grande taille, le niveau d'interactivité est
aussi plus limité. Néanmoins, il est possible de visualiser de petites
images avec la librairie Matplotlib.

\hypertarget{outils-de-visualisation}{%
\subsubsection{Outils de visualisation}\label{outils-de-visualisation}}

Il existe plusieurs outils gratuits de visualisation d'une image
satellite, on peut mentionner les deux principaux: - QGIS - ESA Snap

\hypertarget{visualisation-sur-le-web}{%
\subsubsection{Visualisation sur le
Web}\label{visualisation-sur-le-web}}

Une des meilleures pratiques pour visualiser une image de grande taille
est d'utiliser un service de type Web Mapping Service (WMS). Cependant,
type de service nécessite une architecture client-serveur qui est plus
complexe à mettre en place.

Google Earth Engine offre des moyens de visualiser de la donnée locale:
\emph{Working with Local Geospatial Data} --- via
\href{https://geog-312.gishub.org/book/geospatial/geemap.html\#working-with-local-geospatial-data}{17.
Geemap --- Introduction to GIS Programming}

via \href{https://github.com/opengeos/data/tree/main/raster}{data/raster
at main · opengeos/data}

\hypertarget{visualisation-3d}{%
\subsubsection{Visualisation 3D}\label{visualisation-3d}}

drapper une image satellite sur un DEM

\hypertarget{exercices-de-ruxe9vision}{%
\subsection{Exercices de révision}\label{exercices-de-ruxe9vision}}


    % Add a bibliography block to the postdoc
    
    
    
\end{document}
